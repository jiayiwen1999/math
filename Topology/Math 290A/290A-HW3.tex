\documentclass[12pt]{amsart}
\usepackage{amsmath,epsfig,fancyhdr,amssymb,subfigure,setspace,fullpage,mathrsfs,mathtools,graphicx,wrapfig}
\graphicspath{ {./image/} }
\usepackage[utf8]{inputenc}

\newcommand{\R}{\mathbb{R}}
\newcommand{\C}{\mathbb{C}}
\newcommand{\Z}{\mathbb{Z}}
\newcommand{\N}{\mathbb{N}}
\newcommand{\G}{\mathcal{N}}
\newcommand{\A}{\mathcal{A}}
\newcommand{\sB}{\mathscr{B}}
\newcommand{\sd}{{\Sigma\Delta}}
\newcommand{\sC}{\mathscr{C}}

\begin{document}
\title{Homework2 - 290A}
\maketitle
\begin{center}
    Jiayi Wen\\
    A15157596
\end{center}
For the first problem, I discussed with Prof. Lin during the office hour, and he gave me a hint on how to relate covering space and null-homotopic.\\
\textbf{Problem 1:} Let $p(t)=e^{2\pi i t}$ be the universal covering mapping from $\R$ to $S^1$. Then we can define $p^n(t_1,\dots,t_n)=(e^{2\pi i t_1},\dots,e^{2\pi i t_n})$ be a covering map from $\R^n$ to $\Pi_{i=1}^n S^1$. Since $\mathbb{T}^n=\Pi_{i=1}^n S^1$ and $\pi_1(\R^n)=0$, we have $(\R^n,p^n)$ is the universal covering space of $\mathbb{T}^n$. \\
Next, we want to show $f_\ast$ is a trivial homomorphism. Suppose $\pi_1(X)\neq ker f_\ast$ towards contradiction. Then there exists some $\alpha \in \pi_(X)$ such that $f_\ast (\alpha)\neq (0,0,\dots,0)$. Since $\pi_1(X)$ is a finite set, $o(\alpha)$ is finite. Then we have 
\[(0,\dots,0)=f(1_{\pi_1(X)})=f(\alpha^{o(\alpha)})=f(\alpha)^{o(\alpha)}\]
However, $\pi_1(\mathbb{T}^n)=\Pi_{i=1}^n\Z$ is an infinite group with no element of finite order except the identity. Hence, $f(\alpha)=(0,\dots,0)$. It contradicts. So $\pi_1(X)=ker f_\ast$. Therefore, we have $Im(f_\ast)=Im(p^n_\ast)=0$. Then by map lifting theorem, there exists a unique lift $\tilde{f}$ of $f$ such that $p\circ \tilde{f}=f$. Since $\R^n$ is contractible, $\tilde{f}$ is null-homotopic. Hence, we have $f=p\circ \tilde{f}\simeq p\circ c_a$, where $c_a$ is a constant mapping into $\R^n$. But any continuous map composition after a constant map is still constant. Hence, $p\circ c_a$ is a constant map into $\mathbb{T}^n$. So $f$ is null-homotopic by definition.
\\\phantom{qed}\hfill$\square$\\
\textbf{Problem 2:} Since $S^1$ is path-connected, the fundamental group of $X$ is determined by the attaching map $f$. Also, $f(z)=z^n$ is a loop with winding number $n$ in $S^1$. Hence, we have $\pi_1(X)=\pi_1(S^1)/n\Z=\Z/n\Z$ because $n\Z$ is the normal subgroup generated by $f$.\\
We will construct the universal covering space $\tilde{X}$ by constructing a CW-complex first and then descibe the covering map. We start with one 0-cell, then attach a 1-cell on the 0-cell. So we have $SK_1(\tilde{X})=S^1$ with a base point identified. Next, we attach a family of disks $\{D^2_k\}_{k=1}^n$ with the attaching map $f_k(z)=ze^{\frac{2\pi k}{n}}$. Next, we define the covering map as $$p:\tilde{X}\to X$$
\[p(\tilde{x})=\begin{cases}
    \tilde{x}^n & \text{ if } \tilde{x}\in SK_1(\tilde{X})\\
    \tilde{x} & \text{ if }\tilde{x}\in e^2_k \text{ for some }1\leq k\leq n\\ 
\end{cases}\]
where $e^2_k$ is the interior of the disk $D^2_k$. The covering map is continuous because if we restrict to any disk $D^2_k$, $p$ is the identity map, and the pasting lemma follows because the set of preimages of $x\in S^1\subseteq X$ is an invariant under rotations by $\frac{2\pi k}{n}$ for any $k$. Hence, we obtain a map that is locally homeomorphism between $\tilde{X}$ and $X$ because restricting to a small neighborhood, we get an identity map. So $(\tilde{X},p)$ is indeed a covering space of $X$. Also, we have $\pi_1(\tilde{X})=0$ since once we attach $D^2_n$ to $SK_1(\tilde{X})$, we have $SK_1(\tilde{X})\cup_{f_n} D^2_n\cong D^2_n$ and $\pi_1(D^2_n)=0$. So $(\tilde{X},p)$ is a universal covering space.
\\\phantom{qed}\hfill$\square$\\
\textbf{Problem 3:} Since $\tilde{g}$ can be viewed as inclusion map from the fiber product to $Y\times Z$ composite with a projection map to the first coordinate, we have $\tilde{g}$ is continuous. Same for $\tilde{f}$.\\
Next, we want to show for any $y\in Y$, there exists an evenly covered open neighborhood $N_y$ of $y$ in $Y$. Let $x=f(y)$. Since $g$ is a covering map, then there exists an open neighborhood $U$ of $x$ such that $g^{-1}(U)=\sqcup_\alpha V_\alpha$ where $V_\alpha$ are disjoint open sets in $Z$. And $g|_{V_\alpha}$ is a homeomorphism for all $\alpha$. Let $N_y=f^{-1}(U)$. Then $N_y$ is open since $U$ is open and $f$ is continuous, and $y\in N_y$ since $f(y)=x\in U$. So $N_y$ is an open neighborhood of $y$ in $Y$.\\ 
We claim that $\tilde{g}^{-1}(N_y)=\sqcup_\alpha \big((N_y,V_\alpha)\cap Y\times_X Z\big)$. If $(y_0,z_0)\in \tilde{g}^{-1}(N_y)$, then we have $y_0\in N_y$; hence, we have $g(z_0)=f(y_0)\in U$. So $z_0\in g^{-1}(U)=\sqcup_\alpha V_\alpha$. Hence, we have 
\[(y_0,z_0)\in \sqcup_\alpha \big((N_y,V_\alpha)\cap Y\times_X Z\big)\]
Conversely, if $(y_0,z_0)\in \sqcup_\alpha \big((N_y,V_\alpha)\cap Y\times_X Z\big)$, then $y_0\in N_y$. Hence, $\tilde{g}(y_0,z_0)=y_0\in N_y$. So $(y_0,z_0)\in \tilde{g}^{-1}(N_y)$. So we have 
\[\tilde{g}^{-1}(N_y)=\sqcup_\alpha \big((N_y,V_\alpha)\cap Y\times_X Z\big)\] Notice that $\{V_\alpha\}$ is a family of disjoint open sets, we have $\sqcup_\alpha \big((N_y,V_\alpha)\cap Y\times_X Z\big)$ is a disjoint union of open sets.\\
Let's restrict $\tilde{g}$ to any sheet $(N_y,V_\alpha)\cap Y\times_X Z$. For any $y_0\in N_y$, we have $f(y_0)=x_0$ for some $x_0$ in $U$. Since $g|_{V_\alpha}$ is a homeomorphism onto $U$, we have $f(y_0)=x_0=g(z_0)$ for some $z_0\in V_\alpha$. So we have a continuous mapping $(id_{N_y},g^{-1}|_{V_\alpha}\circ f)$. So we have 
\[(id_{N_y},g^{-1}|_{V_\alpha}\circ f)\circ \tilde{g}(y_0,z_0)=(y_0,g^{-1}|_{V_\alpha}\circ f(y_0))=(y_0,g^{-1}|_{V_\alpha}(x_0))=(y_0,z_0)\]
\[\tilde{g}\circ (id_{N_y},g^{-1}|_{V_\alpha}\circ f)(y_0)=\tilde{g}(y_0,z_0)=y_0\]
So $(id_{N_y},g^{-1}|_{V_\alpha}\circ f)$ is an continuous inverse of $\tilde{g}$ to the sheet $(N_y,V_\alpha)\cap Y\times_X Z$. So $\tilde{g}$ to any sheet $(N_y,V_\alpha)\cap Y\times_X Z$ is a homeomorphism onto $N_y$. So $\tilde{g}$ is a covering map.
\\\phantom{qed}\hfill$\square$\\
\textbf{Problem 4:} Suppose $x\in \tilde{X}$ and $h(x)=y$. Since $p\circ h=p$, we have $p(x)=p(h(x))=p(y)$. If we view one covering map $p$ as a basepoint preserving map from $(\tilde{X},y)$ to $(X,p(y))$ and consider the covering map as $p:(\tilde{X},x)\to (X,p(y))$. Then we have 
\[p_\ast(\pi_1(\tilde{X},y))=0\subseteq p_\ast(\pi_1(\tilde{X},x))\]
By the map lifting theorem, the mapping $p:(\tilde{X},y)\to (X,p(y))$ has a lift $\tilde{p}: (\tilde{X},y)\to (\tilde{X},x)$.
Next, we want to show $\tilde{p}$ is an inverse of $h$. Notice that $id_{\tilde{X}}$ is a lift of $p$ as well since $p\ \circ \ id_{\tilde{X}}=p$. Furthermore, both $h\circ \tilde{p}$ and $\tilde{p}\circ h$ are lifts because 
\[p\circ (h\circ \tilde{p})=p\circ h\circ \tilde{p}=p\circ \tilde{p}=p\]
\[p\circ (\tilde{p}\circ h)=(p\circ \tilde{p})\circ h=p\circ h=p\]
Since $\tilde{X}$ is path-connected, we have $\tilde{X}$ is connected. And we know two lifts are the same if they agree on some point in $\tilde{X}$. And we have 
\[(\tilde{p}\circ h)(x)=\tilde{p}(y)=x=id_{\tilde{X}}(x)\]
\[(h\circ \tilde{p})(y)=h(x)=y=id_{\tilde{X}}(y)\]
So we have $\tilde{p}\circ h=id_{\tilde{X}}=h\circ \tilde{p}$. So $h$ is a homeomorphism and hence a deck transformation.
\\\phantom{qed}\hfill$\square$\\
\textbf{Problem 5:} Since $\tilde{X}$ and $\tilde{Y}$ are simply connected, they are the universal coverings for $X$ and $Y$, respectively. We will construct a homotopy equivalence between the universal covering space.\\
Let's define some notation first. Let's $(\tilde{X},p_x)$ and $(\tilde{Y},p_y)$ be the universal covering space for $X$ and $Y$, respectively. Also, we want to specify the basepoint for $p_x:(\tilde{X},\tilde{x_0})\to (X,x_0)$ and $p_y:(\tilde{Y},\tilde{y_0})\to (Y,y_0) $. And let $f:(X,x_0)\to (Y,y_0)$ and $g:(Y,y_0)\to (X,x_0)$ be homotopy equivalences such that $f\circ g\simeq 1_Y$ and $g\circ f\simeq 1_X$. Then we obtain a continuous mapping $f\circ p_x:\tilde{X}\to Y$. Since the $\tilde{X}$ and $\tilde{Y}$ have trivial fundamental group, $p_y$ and $f\circ p_x$ induce trivial homomorphisms. Hence, there exists a lifting $\widetilde{f\circ p_x}$ such that $p_y\circ \widetilde{f\circ p_x}=f\circ p_x$ by lifting criterion, and the lifting is unique since we specified the basepoint. Same argument for $g\circ p_y$. So we have a lifting $\widetilde{g\circ p_y}$ for $g\circ p_y$. \\
Next, we want to show $\widetilde{f\circ p_x}$ is homotopy equivalence. 
\[p_y\circ (\widetilde{f\circ p_x}\circ \widetilde{g\circ p_y})=(p_y\circ \widetilde{f\circ p_x})\circ \widetilde{g\circ p_y}=f\circ (p_x\circ \widetilde{g\circ p_y})=f\circ g\circ p_y\tag{1}\]
Notice that (1) implies $\widetilde{f\circ p_x}\circ \widetilde{g\circ p_y}$ is a unique lifting of $f\circ g\circ p_y$. Also, we have a homotopy $H:\tilde{Y}\times I\to Y$ such that $H|_{\tilde{Y}\times \{0\}}=f\circ g\circ p_y$ and $H|_{\tilde{Y}\times \{1\}}=1_Y\circ p_y=p_y$. Then by unique homotopy lifting property, we have a unique lift $\tilde{H}$ of the homotopy such that $p_y\circ \tilde{H}=H$. Also, if $in_{\tilde{Y}}:\tilde{Y}\to \tilde{Y}\times \{0\}$ is the canonical mapping, then we have \[\tilde{H}\circ in_{\tilde{Y}}=\tilde{H}|_{\tilde{Y}\times \{0\}}=\widetilde{f\circ p_x}\circ \widetilde{g\circ p_y}\]
And $\tilde{H}|_{\tilde{Y}\times \{1\}}$ is a lift for $p_y$. Since $p_y$ is a mapping with basepoint specified, so it has a unique lift. Also, we know $p_y\circ 1_{\tilde{Y}}=p_y$ is a basepoint preserving lift. So $\tilde{H}|_{\tilde{Y}\times \{1\}}=1_{\tilde{Y}}$. So $\tilde{H}$ gives the homotopy for $\widetilde{f\circ p_x}\circ \widetilde{g\circ p_y}\simeq 1_{\tilde{Y}}$. If we switch the notation for $X$ and $Y$, we get the argument for $\widetilde{g\circ p_y}\circ \widetilde{f\circ p_x}\simeq 1_{\tilde{X}}$. The proof is complete. 
\\\phantom{qed}\hfill$\square$\\
\textbf{Problem 6:} First, we can realize $F_n$ as the fundamental group of the wedge product of $n$-copies of $S^1$ at the same point. In order to obtain a subgroup with index $m$, we want to find a $m$-fold covering space of $\vee_n S^1$. The argument follows simply from counting.\\
Notice that $\vee_n S^1$ is a graph with 1 vertice and $n$ edges. If $X$ is a $m-$fold covering space of the graph, then it has $m$ vertices and $mn$ edges. Let $T$ be a maximal tree of $X$. By the theroem we proved in lecture 10, the fundamental group of a path-connected graph is isomorphic to $F_S$ where $S$ is the set of parametrized edges in $X\setminus T$. So we have $|S|=mn-(m-1)=mn-m+1$. So we have $\pi_1(X)\cong F_{mn-m+1}\leq \pi_1(\vee_n S^1)=F_n$. Let $G=\pi_1(X)$, then we have $|F_n:G|=m$ since $X$ is a $m$-fold covering space.
\\\phantom{qed}\hfill$\square$\\
\textbf{Problem 7:} First, let's realize $G$ as a fundamental group of a CW-complex. Because attaching a higher dimensional cell($n\geq 3$) doesn't change the fundamental group, we can always realize it as a 2-dimensional CW-complex. Let $X$ be a CW-complex and $SK_0X=\{0\}$. Then we attach three 1-cells and get $SK_1X=S^1\vee S^1\vee S^1$. We can parametrized three $S^1$ by $a,b,c$. Then we want to attach one 2-cell where the attaching map is $f(z)=z^2$ on $a$. Another 2-cell is attaching along $bcb^{-1}c^{-1}$. So $X$ has one 0-cell, three 1-cells, and two 2-cells.
\[\pi(X)=\langle a,b,c\mid a^2,bcb^{-1}c^{-1}\rangle=\Z/2\Z\ast \Z\oplus \Z=G\]
By group theory, we know index-2 subgroups are normal subgroups. So we there exists a bijection between index-2 subgroups and $2$-fold covering spaces(up to isomorphism). So the problem becomes to classify all 2-fold covering spaces of $X$ up to isomorphism. \\
Suppose $p$ is a covering map, then $p^{-1}(\mathbb{RP}^2)$ is either $S^2$, which maps by antipodal map, or 2 copies of $\mathbb{RP}^2$, which maps by identity on each copy. And $p^{-1}({\mathbb{T}^2})$ is either two copies of $\mathbb{T}^2$, which maps by identity, or one copy of $\mathbb{T}^2$, which maps by $(z^2,z)$. So we have a combination of 4 possibilities in total. But it is impossible to have 2 copies of $\mathbb{RP}^2$ and two copies of $\mathbb{T}^2$. Since the covering space is connected, each copy of $\mathbb{T}^2$ must have a point in common with one copy of $\mathbb{RP}^2$, but these common point fixed the position of the other copy of $\mathbb{RP}^2$ since it is a 2-fold covering. Then $p$ is locally homeomorphism forces two copies of $\mathbb{RP}^2$ intersect at the boundary since $\mathbb{RP}^2$ (the one in $X$) is a 2-dim manifold with boundary and so does the covering space. But this implies a copy of $S^2$ in the covering space.\\
Case 1: One copy of $S^2$ and one copy of $\mathbb{T}^2$. The covering space is $S^2\cup \mathbb{T}^2/\sim$. Let $x\in S^2$ and $(a,b)\in \mathbb{T}^2$, then we identify $x\sim(a,b)$ and $-x\sim (-a,b)$  The covering map is given by the antipodal map on $S^2$ and $(z_1^2,z_2)$ on $\mathbb{T}^2$. It is locally homeomorphism because this is just quotient map if we restrict on $S^2$ or $\mathbb{T}^2$ and the image is the same for the two point in common(pasting lemma).\\
Case 2: One copy of $S^2$ and two copy of $\mathbb{T}^2$. The covering space is $S^2\cup \mathbb{T}^2/\sim$, but we identify $x\sim (a,b)$ and $x\sim (a',b')$.  And the covering map is antipodal map on $S^2$ and identity on each copy of $\mathbb{T}^2$, where $p(a,b)=p(a',b')=p(x)=p(-x)$. The map is locally homeomorphism for the same reason.\\
Case 3: Two copies of $\mathbb{RP}^2$ and one copy of $\mathbb{T}^2$. The covering space is $\mathbb{RP}^2\cup \mathbb{RP}^2\cup \mathbb{T}^2/\sim $. If $x,x'\in \mathbb{RP}^2$, then we want to identify $x\sim (a,b)$ and $x'\sim (-a,b)$. The covering map is given by $(z_1^2,z_2$ on $\mathbb{T}^2$ and identity on $\mathbb{RP}^2$, where $p(a,b)=p(-a,b)=p(x)=p(x')$. Locally homeomorphism for the same reason as above.\\
These three cases are not isomorphic because the space is not homeomorphic. It is easy to see if we remove a circle on $S^2$ in case 2 or remove two circle on $\mathbb{T}^2$(making it disconnected). Hence, we have 3 subgroups of index-2 in $G$.
\\\phantom{qed}\hfill$\square$\\
\textbf{Problem 8:} First, let's realize $G$ as a fundamental group of a CW-complex. Let $SK_0X=\{0\}$ and $SK_1X=S^1\vee S^1$. Then we attach two 2-cells. One is glued by $f(z)=z^2$ on one copy of $S^1$, and the other is glued by $g(z)=z^3$ on the other copy of $S^1$. Notice that the part with attaching map $f(z)=z^2$ is homeomorphic to $\mathbb{RP}^2$. The other part is the space in problem 2 with $n=3$. Hence, $X$ is just the wedge of two spaces. we have 
\[\pi_1(X)=\pi_1(\mathbb{RP}^2\ast \pi_1(S^1\cup_g D^2)=\Z/2\Z\ast\Z/3\Z=G\]
Next, we want to realize $X$ as an orbit of a CW-complex under group action by $\Z/2\Z\oplus \Z/3\Z$. Here is a picture that help us to visualize the space. The picture gives the one skeleton of the space, with $x_i$ are the 0-cells. And we have 5 circles in the one skeleton. Each of the two circles on the top and bottom pass through 3 points. Each of the three circles in the middle pass through 2 points. Then we want to construct 3 spheres on $S^1_1$, $S^1_2$ and $S^1_3$ by  attaching two 2-cells on each of the circles. On the top we want to attach 3 2-cells by identity map on $S^1_4$ with each rotates $\frac{2\pi}{3}$ on $S^1_4$. This is the same construction of the universal covering of $S_1\cup_g D^2$. Similarly, we do the same thing for the bottom circle $S^1_5$. We call this CW-complex by $Y$. Next, we define the action on $Y$. The action is done basically by rotations. If we embed $Y$ into $\R^3$, then we can do two things. One is to rotate $\frac{2\pi}{3}$ along the axis that is through the center of $S^1_4$ and $S^1_5$. So this sends each 2-cell in the universal covering of $S_1\cup_g D^2$ to another 2-cell, and rotate the boundary, which is $S^1_4$ or $S^1_5$ by $\frac{2\pi}{3}$. An send each sphere to another(i.e the corresponding sphere of $S^1_1$ to the one of $S^1_2$). The action of $(0,1)$ is defined by this rotation. On the other hand, we can also rotate by $\pi$ along the axis that through the center of $S^1_3$ and pointing into the paper. So this sends the corresponding sphere of $S^1_1$ to $S^1_2$ and defines an antipodal map on $S^1_3$. Also it take the whole universal covering of $S^1\cup_g D^2$ that based on $S^1_4$ to $S^1_5$. This rotation defines the action by ($1,0$). The orbit $Y/G$ is homeomorphic to $X$. Because each sphere will be identify by antipodal map through $(1,0),(1,1),$ or $(1,2)$. And all sphere can be identify as one by $\langle (0,1)\rangle$. Two copies of universal covering of $S^1\cup_g D^2$ is identify by $(1,0)$ and $S^1\cup_g D^2$ is identify by $\langle (0,1)\rangle$.\\
The only thing left is to understand the fundamental group of $Y$. A nice observation. Let's choose the basepoint as $x_1$( We are free to choose any basepoint because the space is path connected). For convenience, we will use $S^1_i$ to denote the space based on that circle(i.e. $S^1_1$ is the sphere with one skeleton as $S^1_1$ and $S^1_4$ is the copy of the universal covering space of $S^1\cup_g D^2$ base at the circle.). Because $Y$ is the wedge of several simply connected space, let's call these as "parts" of $Y$. Then if we restrict a loop to some interval $[a,b]\subseteq I$ and get another loop in some "part" of $Y$, then the restriction map is homotopic to constant since each "part" of $Y$ is simply-connected by our definition. So it is sufficient to discuss only the loop with distinct entry point and exit point on each "part". So the problem is the same as asking the fundamental group of a graph where we identify each "part" as a vertex and an edge exists if they are adjacent "parts", where adjacent means to have a point in common. So it is a graph with 6 edges and a maximal tree of the graph contains 4 edges. So the fundamental group is $F_2$ because $|E(G)\setminus T|=2$. Therefore, we have $\pi_1(Y)=F_2$. Notice that the action we defined above is a covering space action because for $(i,j)\neq (0,0)$, there must have some "part" of $Y$ that is not fixed (when $i=1$, $S^1_4$ is not fiex, when $i=0$, $S^1_1$ is not fixed). So we have 
\[\Z/2\Z\oplus \Z/3\Z\cong \pi_1(Y/\Z/2\Z\oplus \Z/3\Z)/\pi_1(Y)=\pi_1(X)/\pi_1(Y)=G/F_2\]
Notice that $F_2$ is the kernel of some surjective homomorphism $\sigma:G\to \Z/2\Z\oplus \Z/3\Z=G/[G,G]$. The by the universal property of quotient group, there exists an homomorphism $\bar{\sigma}$ that $\bar{\sigma}\circ i=\sigma$, where $i$ is the quotient map. Since $\sigma$ is surjective, $\bar{\sigma}$ is surjective as well. And surjective implies bijective for mapping between finite sets with same cardinality. So we get an isomorphism. Hence, we have $F_2=ker\sigma=ker i=[G,G]$.
\\\phantom{qed}\hfill$\square$\\

\end{document}