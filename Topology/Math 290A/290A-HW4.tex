\documentclass[12pt]{amsart}
\usepackage{amsmath,epsfig,fancyhdr,amssymb,subfigure,setspace,fullpage,mathrsfs,mathtools}
\usepackage[utf8]{inputenc}

\newcommand{\R}{\mathbb{R}}
\newcommand{\C}{\mathbb{C}}
\newcommand{\Z}{\mathbb{Z}}
\newcommand{\N}{\mathbb{N}}
\newcommand{\G}{\mathcal{N}}
\newcommand{\A}{\mathcal{A}}
\newcommand{\sB}{\mathscr{B}}
\newcommand{\sC}{\mathscr{C}}
\newcommand{\sd}{{\Sigma\Delta}}
\newcommand{\torus}{\mathbb{T}^2}
\newcommand{\rp}{\mathbb{RP}^2}

\begin{document}
\title{Homework 4 - 290A}
\maketitle
\begin{center}
    Jiayi Wen\\
    A15157596
\end{center}
For the construct of connected closed 2-surfaces by a polygon in problem 2, we refer the construction that was mentioned in lecture 18.\\
\textbf{Problem 1:} (1) Let's denote the polygon by $X$. Then $\Sigma$ is obtained by a quotient map $f$, which identifies the edges of $X$ pairwise as required. It is obvious that $X$ is a two dimensional manifold with boundary where the boundary of $X$ is just the union of all edges. Also if we restrict $f$ to $X\setminus \partial X$, we will get an identity map. So for any point $x\in\Sigma$ such that $f^{-1}(x)\subseteq X\setminus\partial X$, we have $f^{-1}(x)=\{y\}$ for some interior point $y$ of $X$. We choose a small neighborhood $U$ of $x$ such that the preimage of $U$ under $f$ contains no boundary point of $X$. Then if we restrict $f$ to $U$ we get a locally homeomorphism, which is the identity map actually, then it makes sense to talks about the inverse. And we have $f^{-1}(U)$ is an open neighborhood of $f^{-1}(x)=y$. Suppose $g: f^{-1}(U)\to V$ is a parametrization of $X$ at $y$ and $V$ is an open subset of $\R^2$. Then we have $g\circ f^{-1}|_{U}:U\to V$ gives a parametrization of $\Sigma$ at $x$.\\
If $x\in\Sigma$ is the image of a boundary point of $X$ but not a vertex, then it has two preimages $y_1,y_2$, and they are the corresponding points of the pair of directed edges ($l_i,l_i'$). So if we take an open small neighborhood $U$ of $x$ such that  $f^{-1}(U)$ contains no vertices of $X$, then $f^{-1}(U)=V_1\sqcup V_2$ where $V_i$ is an open neighborhood of $y_i$. Since $X$ is a manifold with boundary, we can obtain a locally homeomorphism from $V_i$ to the unit open half-disk $D_i$ on $H^2$, where $H^2=\{(x_1,x_2)\in\R^2\mid x_2\geq 0\}$ is the upper half-space in $\R^2$. Notice that if the locally homeomorphism preserve the direction of the edges, namely we want the direction of directed edge of $X$ to is from small $x_1$ value to large $x_1$ value in $H^2$, then we glue two half space together by the $x_2$ with the same direction. So we get an unit open disk $Y$ in $\R^2$. If we take the composition of all these map we get a locally homeomorphism from $Y$ to $U$.\\
For a vertex point, the argument is similar, but instead of identifies two edges, we might identify more. First observation is that whenenver we have $k$ vertices identified as the same point $x\in\Sigma$, we can take a small neighborhood $U$ of $x$ such that the preimage doesn't contain another vertices that is different from $x$ and have no overlap in $X$. Such choice is possible because we have finite many vertices in $X$. Then we denote $f^{-1}(x)=\{y_1,\dots,y_k\}$, and for each $y_i$, there are two adjacent edges $y_i$. After consider the direction on each edges, $y_i$ is either the starting point of a directed edge or an end point of a directed edge. By our choice of $U$, $f^{-1}(U)$ should contains 2$k$ pieces of directed edges. So, we distinguished a piece not just by the edges in $X$ but also by whether it contains the starting point or the end point of the edge. If we start with some vertex and one of the piece that is adjacent to the vertex, then we go to the corresponding piece that attached together in $\Sigma$. Next, we go to the other piece that is adjacent to the same vertex, and repeat the previous process. For example, if we denote $a_e$ as the end piece of $a$ and $a_s$ as starting piece of $a$. Also denote T,B,L,R as top, bottom, left,right, respectively. We can have an ordered sequence starting from the vertex of $a_e(B)$ and $b_e(B)$ and the piece $a_e(B)$.
\[a_e(B),a_e(T),d_e(T),d_e(B),a_s(B),a_s(T),c_e(R),c_e(L),b_e(T),b_e(B)\]
Because $\Sigma$ is obtained by an equivalence relation, the ordered sequence is unique up to action by $\langle(1,2,\dots,2k)\rangle$, where $k$ is number of preimage of the vertex. This actually gives us the order to glue the open half disks into an open disk in $\R^2$. It makes sense to glue them together because each open neighborhood of a vertex is homeomorphic to the unit open half disk, which is homeomorphic to any sector of the open disk by a homeomorphism that changes angle of the sector. And we make the angle of the sector to be $\frac{2\pi}{m}$ if $m$ vertices are identified as the same point. Then the quotient map makes this into a open disk in $\R^2$. So $\Sigma$ is a 2-dim manifold without the boundary. Notice that $\Sigma=f(X)$ is compact because $X$ is compact and $f$ is continuous. So $\Sigma$ is a closed surface.\\
(2) $\Sigma$ is not orientable because the pair of $a$ are both counterclockwise. So if we draw a path from some point on $a$ at the bottom to the corresponding point on $a$ at the top, we get a loop that doesn't preserve the orientation. From part 1, we actually showed that 5 vertices of $X$ are identified as the same point in $\Sigma$. The rest of 3 vertices of $X$ are identified as the same point as well if we check the direction. So $\Sigma$ as two 0-cells, four 1-cells, one 2-cell. Hence, we have $\chi(\Sigma)=2-4+1=-1$.\\
(3) By the corollary in the lecture, we know $\chi(\Sigma)=2-g(\Sigma)$ if $\Sigma$ is non-orientable. Hence $g(\Sigma)=3$. Then by the classification theorem of closed surfaces, we konw $\Sigma\cong \#^3\mathbb{RP}^2$.
\\\phantom{qed}\hfill$\square$\\
\textbf{Problem 2:}(1) Since we are removing finitely many open disks from the surface, we can do this by induction. Let $\Sigma_i$ denote the closed surface $\Sigma$ with $i$ open disk removed. Then, by the same argument in the lecture, we have $\Sigma=\Sigma_1\cup_{id} D^2$, where $D^2$ is attaching by the identity map of $S^1$. Then we have 
\[\chi(\Sigma)=\chi(\Sigma_1)+\chi(D^2)-\chi(S^1)=\chi(\Sigma_1)+1\]
Hence, $\chi(\Sigma_1)=\chi(\Sigma)-1$. Suppose $\chi(\Sigma_i)=\chi(\Sigma)-i$.\\
Then we have $\Sigma_i=\Sigma_{i+1}\cup_{id} D^2$, then by the same process, we have 
\[\chi(\Sigma_{i+1})=\chi(\Sigma_{i})-1=\chi(\Sigma)-i-1\]
The induction completes. So we have $\chi(\Sigma')=\chi(\Sigma)-k$.\\
(2) To prove the result, we want to split into 3 cases by our classification theorem of connected closed 2-surface. Also, to make the statement make senses, we must have $k\geq 1$ since none of $S^2,\mathbb{T}^2,\mathbb{RP}^2$ is homotopic to some wedge of $S^1$ (by looking at the fundamental group). Also, the complement of $k$ disjoint open disks can be obtained by a strong deformation retraction from the complement of $k$ distinct points. The argument is obvious. If we only need to consider the deformation retraction of $U_i\setminus \{x_i\}\cong D^2\setminus\{0\}$ to $S^1$ and the deformation retraction is given by $f(z)=\frac{z}{|z|}$ if we embed $D^2$ as the unit disk in $\C$. Because no open neighborhood is contained in an edge, we can choose all the points lying in the 2-cell. Because a manifold is locally homeomorphic to an open subset of an Euclidean space, which is path connected, the manifold is locally path-connected. Since $\Sigma$ is connected, we know $\Sigma$ is path connected because locally path connected implies same component and path component. We can construct a null homotopic map from $D^2$ to $D^2$ such that it fixes the boundary and move an interior point $x$ to another interior point $y$. The idea is to work on a small $\epsilon$ neighborhood of a path from $x$ to $y$. And the homotopy restrict to $(x,\dot)$ is the path we chose. So we are free to choose the locations, where to remove point from the surface.\\
\textbf{Case 1:} If $\Sigma\cong S^2$, then we have $\Sigma'\cong S^2\setminus\{k \text{ points}\}$. And we have 
\[S^2\setminus\{k \text{ points}\}\cong \R^2\setminus\{k-1 \text{ points}\}\simeq \vee_{k-1}S^1\]
This is a result that proved in the second homework. So we omit the details here. This completes the proof of case 1 since $1-\chi(\Sigma')=1-(\chi(\Sigma)-k)=1-2+k=k-1$.\\
\textbf{Case 2:} If $\Sigma \cong\#^n\torus$, then we want to construct $\Sigma$ by a polygon with $4n$ edges(refer to lecture 18). We want to remove $k$ points from the 2-cell. Let's consider the unit disk $D^2$. Then this is the same as case one because we can choose to remove $k$ points inside $D^2$. And we have a deformation retraction from $\R^2\cong \C$ to $D^2$ by $f(z)=\frac{z}{|z|}$ for all $z\in \R^2$ such that $|z|\geq 1$. So $D^2\setminus \{k\text{ points}\}\simeq \R^2\setminus\{k \text{ points}\}\simeq \vee_{k}S^1$. But it is useful to think one step backwards. As we deformation retracts $D^2\setminus \{k\text{ points}\}$, we deformation retracts to $S^1$ with $k-1$ non-intersecting paths that split $D^2$ into $k$ parts with one point removed from each part. Then $k-1$ copies of $S^1$ can be realized by a homotopy of attaching map of these paths to $\partial D^2=S^1$.\\
The last step is to show that the attaching map of $D^2$ to the one skeleton of $\Sigma$ divides the boundary into some wedge product of $S^1$. Notice that the construction gives us the wedge product naturally. If we count the Euler characteristic, we have 
\[2-2n=\chi(\Sigma)=\#(0-cell)-\#(1-cells)+\#(2-cell)=\#(0-cell)-2n+1\]
So we have one 0-cell and $2n$ distinct loops in $\Sigma$ and $\Sigma'$. If $f:\partial D^2\to S^1$ is the attaching map for $\Sigma$, then we have 
\[\Sigma'=SK_1(\Sigma)\cup_f D^2\setminus\{k \text{ points}\}\simeq Sk_1(\Sigma)\cup_f (\partial D^2 \vee_{k-1}S^1)=\vee_{2n}S^1\cup_f (\partial D^2 \vee_{k-1}S^1)\]
Since $\partial D^2$ is attaching on the wedge first $2n$ copies of $S^1$, we will finally get the wedge of $2n+k-1$ copies of $S^1$. The number agrees because 
\[1-\chi(\Sigma')=1-(\chi(\Sigma)-k)=k+1-\chi(\Sigma)=k+1-(2-2n)=k-1+2n\]
\textbf{Case 3:} The statement is similar. If we have $\Sigma\cong \#^n\rp$, then we realized it as a polygon with $2n$ edges(refer to lecture 18 as well). And we can assume that $k$ points are removed from the interior of $D^2$. This also gives us a deformation retract to $ \partial D^2 \vee_{k-1}S^1$. But if we look at the one skeleton of $\Sigma'$, we get one 0-cell and $n$ 1-cells because 
\[2-n=\chi(\Sigma)=\#(0-cell)-\#(1-cells)+\#(2-cell)=\#(0-cell)-n+1\]
If the attaching map of $D^2$ is $f$, then we have 
\[\chi(\Sigma')\simeq \vee_{n}S^1\cup_f (\partial D^2\vee_{k-1}S^1)=\vee_{n+k-1}S^1\]
And we have 
\[1-\chi(\Sigma')=k+1-\chi(\Sigma)=k+1-2+n=n+k-1\]
\\\phantom{qed}\hfill$\square$\\
\textbf{Problem 3:}
(1): Let's follow the hint. Since $\Sigma$ is a connected closed surface, we know $\Sigma$ has homeomorphism type of $S^2$, $\#^n \torus$, and $\#^n \rp$. By the assumption given in the problem, we have two cases $\#^n\torus$ or $\#^n\rp$, where $n\geq 1$ for $\torus $ and $n\geq 2$ for $\rp$. For both cases, we induct on $n$.\\
\textbf{Case 1:} If $\Sigma\cong \#^n\torus$. For $n=1$, we will use a standard way to build the covering space. We start with the standard construction of a torus. So we start with $I\times I\subseteq \R^2$ where $I=[0,1]$. Then we take the quotient given by $(x,0)\sim(x,1)$ and $(0,y)\sim(1,y)$. For a $m$-fold covering space, we start with $[0,m]\times I $, and the quotient is given by $(x,0)\sim(x,1)$ and $(0,y)\sim(m,y)$. Hence, we construct a $m$-fold covering of a torus by itself. The covering map is given by 
\[p:[0,m]\times I/\sim\  \to I\times I/\sim\]
\[ (x,y)\mapsto (x-\lfloor x \rfloor,y) \]
where $\lfloor x\rfloor$ is the floor function that gives the maximal integer that is no greater than $x$. The map is obviously continuous for any $(x,y)$ where $x$ is not an integer. Since $(0,y)$ and $(1,y)$ are identified as the same point on a torus, $p$ is continuous at $(x,y)$ when $x$ is an integer.\\
Next, we show $p$ is a covering map. For the interior point $(x,y)$ of $I\times I$, the argument is obvious. Since we can take a small neighborhood $U$ that is contained in the interior of $I\times I$, then the preimage of $U$ are just distinct union of $m$ open neighborhood $V_i\subseteq (i-1,i)\times (0,1)$ where $V_i=\{(x+i,y)\mid (x,y)\in U\}\cong U$. If $(x,y)$ is a boundary point on $I \times I$, the argument is basically the same but we need to deal with the boundaries and the equivalence relations. The following picture can illustrate them well enough.\\
So $p$ is a covering map. The covering is regular since covering transformation can be given by the group action of $\Z/m\Z$, where the generator $\bar{1}$ moves every point horizontally to right by 1. We call the covering space as $(\tilde{\torus},p)$\\
Next, we want to do connected sum. If we attach one more torus on, then we need to first identify an open neighborhood on the torus, then do the connected sum on the boundary of the open neighborhood we just identified. Notice that this open neighborhood has $m$ copies of homeomorphic open neighborhood on our covering space $(\tilde{\torus},p)$. So we can attach one torus on each copy of open neighborhood on the covering space. So the covering space is given by $\tilde{\torus}\#^m \torus$ and the covering map is identity if we restrict to each copy of torus and $p$ if we restrict to $\tilde{\torus}-U_1-U_2-\dots-U_m$ where $U_i$ is the open neighborhood we removed. This is indeed a covering map if you take a small neighborhood of each $U_i$ that doesn't intersect with any other $U_j$, we get a locally homeomorphism. If we are attaching more torus, we repeat the same process. If we have connected sum of $n$ torus, then for the $n-1$ torus we connected to the original one, we need to connect $m$ copies of torus for each. Hence, an $m$-fold covering space of $\Sigma$ is given by $\tilde{\torus}\#^{m(n-1)}\torus\cong ([0,m]\times I/\sim)\#^{m(n-1)}\torus$ and the covering map is the same as we describe on $\tilde{\torus}$ with $m(n-1)$ open disks removed and identity on each copy of torus.\\
\textbf{Case 2:} If $\Sigma\cong \#^n\rp $, then the construction is basically the same, but we start with klein bottle since $\rp\#\rp$ is homeomorphic to klein bottle. We also do the standard construction starting from $I\times I$. The equivalence relation is given by $(x,0)\sim(x,1)$ and $(0,y)\sim(1,1-y)$. Then a m-fold covering is given by $[0,m]\times I$, where the equivalence relation is given by $(x,0)\sim(x,1)$ and $(0,y)\sim(m,1-y)$ if $m$ is odd or $(0,y)\sim (m,y)$ if $m$ is even. And the covering map $p$ is given by 
\[p(x,y)=\begin{cases}
    (x-\lfloor x\rfloor,y) & \text{ if } \lfloor x\rfloor\equiv 0 \mod {2}\\
    (x-\lfloor x\rfloor,1-y) & \text{ if } \lfloor x\rfloor\equiv 1 \mod {2}
\end{cases}\]
The following picture illustrates $p$ is a covering map. The argument is basically the same as for torus but need to care about the location of preimage in each block and the parity of $m$. If we connect one more $\rp$, then we connect $m$ copies of $\rp$ on the $m-$fold covering space $\tilde{K}$ that we construct for the klein bottle $K$. So we have the m-fold covering space for $\Sigma$ is $\tilde{K}\#^{m(n-2)}\rp$ and the covering map is $p$ on $\tilde{K}$ and identity on each copy of $\rp$ we attached. The covering is regular since the deck transformation is also given by $\Z/m\Z$ where $\bar{1}$ is given by horizontal translation together with a reflection along $[0,m]\times \{\frac{1}{2}\}$. So $(x,y)$ will be sent to $(x+1,1-y)$, where the first coordinate is in $\Z/m\Z$.\\
(2): Since the orientability of the covering space is determined by the fundamental group of the covering space, we just need to show $p_\ast\big(\pi_1(\tilde{\Sigma})\big)$ is not a subgroup of $H$, where $H$ is the index 2 subgroup of $\pi_1(\Sigma)$ consists of all homotopy classes of orientation preserving loops. Also, we know $p_\ast\big(\pi_1(\tilde{\Sigma})\big)$ is an index $m$ subgroup of $\pi_1(\Sigma)$ by the one-to-one correspondence between covering spaces(up to isomorphism) and subgourps(up to conjugation). If $p_\ast\big(\pi_1(\tilde{\Sigma})\big)\leq H$, then we have 
\[m=[\pi_1(\Sigma): p_\ast\big(\pi_1(\tilde{\Sigma})\big)]=[\pi_1(\Sigma): H][H: p_\ast\big(\pi_1(\tilde{\Sigma})\big)]=2[H: p_\ast\big(\pi_1(\tilde{\Sigma})\big)]\]
It is a contradiction to the assumption that $m$ is odd. Hence, $\tilde{\Sigma}$ is not orientable.\\
(3): In part 1, we constructed a non-orientable covering. Since on each copy of $\rp$, we can find a orientation reserving loop based at some boundary point. And such loop always exists since $\rp\setminus U$ is path connected, where $U$ is the open disk on $\rp$ whose boundary is used for connected sum.\\
For the orientable covering, we first consider the corresponding covering space of $H$, where $H$ is the index 2 subgroup of $\pi_1(\Sigma)$ consists of all homotopy classes of orientation preserving loops. We call this covering $(\Sigma_o,p_o)$. Then we know $\Sigma_o$ is an orientable closed surface, hence any covering of its is also orientable. Since $m$ is even, then by part 1, we can construct a $\frac{m}{2}$-fold covering for $\Sigma_o$. We call this covering $(\tilde{\Sigma_o},\tilde{p})$. Now, we claim $(\tilde{\Sigma_o},p_o\circ \tilde{p})$ is an orientable $m$-fold covering space of $\Sigma$. It is obvious that every point $x\in\Sigma$ has $m$ preimage because $p_o^{-1}(x)=\{y_1,y_2\}$ and $\mid\tilde{p}^{-1}(y_i)\mid=\frac{m}{2}$ for all $i$. Hence, $\mid (p_o\circ \tilde{p})^{-1}(x)\mid =m$. \\
Also, there exists some open neighborhood $U$ of $x\in\Sigma$ such that $p^{-1}(U)=V_1\sqcup V_2$ and $p\mid_{V_i}$ is a homeomorphism onto $U$. Also, for $y_i$ we have an evenly covered open neighborhood $A_i$ of $y_i$ with respect to $\tilde{p}$. Then $A_i\cap V_i$ is an evenly covered neighborhood of $y_i$ with respect to $\tilde{p}$. Then we can take 
\[U'=p_o(A_1\cap V_1)\cap p_o(A_2\cap V_2)\subseteq U\] 
This is an evenly covered open neighborhood of $x$.
And let 
\[V_i'=p_o(U')\cap V_i\subseteq A_i\cap V_i\] And we have 
\[\tilde{p}^{-1}(V_i')=\sqcup_{j=1}^{\frac{m}{2}}B_ij\]
Hence, we have $\tilde{p}\mid_{B_j}$ is a homeomorphism onto $V_i'$. Hence, $p_o\circ \tilde{p}\mid_{B_ij}=p_o\mid_{V_i}$ is a homeomorphism onto $U'$. Hence, $p_o\circ \tilde{p}$ is a covering map.
\\\phantom{qed}\hfill$\square$\\
\textbf{Problem 4:} If $\Sigma\cong S^2$, then ant continuous map $f$ induce $\varphi$ because $\pi_1(S^2)=0$ is trivial. Hence, $\varphi$ and $f_\ast$ are trivial. If not, then we can use the standard CW structure of closed surface. So $\rp$ is constructed by two 0-cells, two 1-cells and one 2-cell, where the one skeleton forms $S^1$ and we identifies the antipodal point on $S^1$ to get $\rp$. And other homeomorphism types are constructed by a regular n-gon as we showed in lecture 18. So we have a CW structure for $\Sigma$ and there is only one 0-cell in the CW structure.\\
Let's specify the basepoint, and then construct a continuous function on the one skeleton. We pick $e^0=y\in \Sigma$ as the basepoint on the surface. Since $X$ is path-connected, we are free to pick any basepoint. Let $x\in X$ be any point in the space. Since any 1-cell in $\Sigma$ is attaching at $y$, so $SK_1(\Sigma)$ is just the wedge of some number of $S^1$. Then on each copy of $S^1_i$, this is a loop $\gamma $ on $\Sigma$. Hence, the homomorphism between fundamental groups gives as a natural mapping from this copy of $S^1_i$ to $X$. To be more explicitly, we suppose $\varphi([\gamma])$ has a representative $\beta$, then for any $y'\in S^1_i$, we have $\gamma(a)=y'$ for some $a\in I$, where we parametrize $\gamma$ by $a\mapsto e^{2\pi ia}$. So we have a unique inverse when $a\neq 0,1$. Then we can also parametrize $\beta$ by the same way and get $\beta(a)\in X$. So we defined as function $f'|_{S^1_i}(y')=\beta(\gamma^{-1}(y'))=\beta(a)$. And this is indeed continuous since $\beta,\gamma$ are loops. Extend this to all loops. Then we defined a continou $f':SK_1(\Sigma)\to X$. By the construction, we have $f'_\ast=\varphi\circ i_\ast$, where $i:SK_1(\Sigma)\to \Sigma$ is the inclusion map. All we need to do is to extend the function to the 2-cell.\\
Notice that $\Sigma=SK_1(\Sigma)\cup_g D^2$, where $g:S^1\to SK_1(\Sigma)$ is the attaching map for the 2-cell. Since we are doing the standard construction, so the image of $g$ is the whole one skeleton. Hence, we have $g(\partial D^2)=SK_1(\Sigma)$. So if we are able to extend $f'\circ g$ to the interior of $D^2$, then $f'$ can be extended to $\Sigma$ since $g$ is extended by identity map in the interior of $D^2$. However, $g$ denotes a loop in $SK_1(\Sigma)$ as well. So we have $f'_\ast(g)$ is a loop in $X$. But we have $i_\ast(g)=0$ in $\Sigma$ by a linear homotopy in $D^2$. Hence, we have $f'_\ast(g)=\varphi\circ i_\ast(g)=\varphi(0)=0$ is a null homotopic in $X$. This implies $f'\circ g$ can be extened by homework 1. Hence, there exists a extended map $f:\Sigma \to X$ such that $f\circ \bar{g}=\overline{f'\circ g}$, where $\bar{g}$ and $\overline{f'\circ g}$ are the extended map of $g$ and $f'\circ g$, respectively. $f_\ast=\varphi$ because any loops in $\Sigma$ is homotopic to some loop on the one skeleton and we have showed that the restriction of $f'$ to one skeleton induced $\varphi$ by construction.
\\\phantom{qed}\hfill$\square$\\
\textbf{Problem 5:}\\
(1) Let's denote the square as $Y=I\times I \setminus\{point\} $ and denote the quotient map $f:Y\to X$ as the picture in the homework. Let $\gamma$ be a loop on $Y$ such that $f\circ \gamma=\sigma$. Then we know $Y\cong D^2\setminus\{0\}$ can deformation retract to its boundary by normalizing the vectors on $D^2\setminus\{0\}$ (i.e. $r:D^2\setminus\{0\}\to S^1$, $r(x)=\frac{x}{|x|}$) Hence, $\gamma$ is homotopic to a loop $\gamma'$ on the boundary, which is the loop with the same winding number on the boundary. Suppose $\sigma$ has winding number $n$, then $\gamma\simeq\gamma'$ has winding number $n$. If we pick a base point(i.e. the bottom left vertex), then we can represent $\gamma'$ by $(bab^{-1}a^{-1})^n$. Hence $[\sigma]=(bab^{-1}a^{-1})^n\neq 1\in \pi_1(Y)\cong \pi_1(S^1\vee S^1)$.\\
(2) To make the picture more clear, we draw $\sigma$ as a square in the middle instead( we are allowed to do so because square is homeomorphic to a circle). In this part, we will use the notation in Miller's note. If $\varphi:\Delta^2\to X$ is a singular 2-simplex, then we will denote 0,1,2 as the image of $d_0\varphi,d_1\varphi,d_2\varphi$. And the number that denotes the boundary operator are written in the image of each singluar 2-simplex. Then the following picture well illustrate the construction of the singular n-chain $c$. We denote $c_i$ as the triangle region we showed in the picture. Then we have $c=\sum_{i=1}^8 c_i$ and $dc=\sum_{i=1}^8(d_0c_i-d_1c_i+d_2c_i)$. If we look at the picture, then we have $d_0c_1=d_1c_2$, $d_0c_2=d_1c_3$, $d_2c_3=d_1c_4$, $d_0c_4=d_1c_5$, $d_2c_5=d_1c_6$, $d_2c_6=d_1c_7$, $d_0c_7=d_1c_8$, and $d_2c_8=d_1c_1$. Also, by the quotient map $f$, we have $d_2c_1=-d_2c_4$, $d_2c_2=-d_2c_7$. Hence $dc=d_0c_8+d_0c_6+d_0c_5+d_0c_3 $ as indicated in the picture. They have the same image as the loop $\sigma$.\\
Last, we need a degenerated 2-simplex $\gamma:\Delta^2\xrightarrow{\gamma_1}\  \Delta^1\xrightarrow{\gamma_2} X$. The first map $\gamma_1$ is given by the projection to edge 0. And the second map $\gamma_2$ is defined to be $\sigma$. Then we have $d_0\gamma=d_1\gamma-d_2\gamma=dc$
Hence, we have 
\[d(c+\gamma)=dc+d\gamma=dc+d_0\gamma-d_1\gamma+d_2\gamma=d_0\gamma=\sigma\]
So $\sigma$ is null homologurous.\\
\textbf{Problem 6:}\\
(a) If we denote $i\in[m]$ by barycentric coordinates, then we have $i=(0,\dots,0,1,0,\dots,0)$, where the only nonzero entry is the $i$-th entry. Same for $[n]$. Then the map $\phi$ is simply defined as $\phi_{\phi(i)}(s_0,\dots, s_m)=s_i$ where $f_j$ denotes the $j$-th coordinate function of $\phi$. And if $j\notin Im(\phi)$ ( the order preserving map), then $\phi_j=0$. This gives $t_j=\phi_j$ as a function of $(s_0,\dots,s_m$).\\
(b) As very easy way to understand this is to build a combinatorical object. An injective function from $[n-k]$ to $[n]$ is the same as assigning $n-k+1$ indistinguishable balls into $n+1$ baskets and no basket contains more than one ball. To define an order on balls and baskets, we can distinguish balls and baskets by labeling with $[n-k]$ and $[n]$, respectively. As a result, we want to add a rule that the ball with label $i$ can only be put into the basket with label no smaller than $i$. This gives us order preserving. Then it is natural to define $j_i$ as the label of $i$-th empty basket. One way to think of $d^i:[n-1]\to [n]$ is to place $n$ balls in $n+1$ baskets and leaving the basket with label $i$ empty. So this tells us $\phi$ can be written as the composition of the form $d^{l_1}d^{l_2}\dots d^{l_k}$ for some sequence of number . Another way to think about $d^i$ is to move every ball in the basket with label $j\geq i$ to the next basket (i.e. basket with label $j$). So $d^i$ will no change the first $i$ basket. Hence, we have $d^id^j=d^{j+1}d^i$ for $i\leq j$ (The empty basket with label $j$ in $[n-1]$ is the same as the empty basket with $j+1$ in $[n]$ because $d^i$ moves the baskets with label greater than $i$). So this allow us to rearrange the composition and make it into decreasing order. To get strictly decreasing order, we have $d^id^j=d^{j+1}d^{i}$ by the relation we just proved. So we can get a strictly decreasing order. \\
(c) For any $\sigma:\Delta^n\to X$, by the combinatorical meaning, we can think of $d_i$ as $$d_i\sigma=\sigma(s_0,\dots,\hat{s_i},\dots,s_n)$$, where $\hat{s_i}=0$. 
Then part b, if $i\leq j$ we have 
\[d_id_j\sigma=\sigma(s_0,\dots, \hat{s_i},\dots,\hat{s_{j}},\dots,s_n)=d_{j+1}d_i\sigma\]
Hence, we have two ways to obtain $\sigma(s_0,\dots, \hat{s_i},\dots,\hat{s_{j}},\dots,s_n)$. And the coefficient before $d_id_j$ is $(-1)^{i+j}$. So we can reindex the terms in $d^2$ and put the subscript into strictly increasing order.
\[d^2\sigma=\sum_{i<j}(-1)^i(-1)^jd_id_j\sigma+\sum_{j>i}(-1)^i(-1)^{j-1}d_id_j\sigma=0\]
Because $i+j$ and $i+j-1$ has different parity.
\\\phantom{qed}\hfill$\square$\\
\textbf{Problem 7:}\\
(a) We construct the torus by the standard construction. Then we can split the torus into two singular 2-simplices as the following picture. So we have $c=\sigma_1+\sigma_2$ covers all of the torus exactly ones except on the boundary of $\sigma_1$ and $\sigma_2$, where the boundary is a measure zero set on $\torus$. Also, we have $c$ is a cycle because $d_2\sigma 1 =d_1\sigma_2$, $d_0\sigma_1=-d_0\sigma_2$, and $d_1\sigma_1=d_2\sigma_2$. So we have
\[dc=d_0\sigma_1-d_1\sigma_1+d_2\sigma_1+d_0\sigma_2-d_1\sigma_2+d_2\sigma_2=0\]
(b) Since $X,Y$ are disjoint, we have the image of $\sigma:\Delta^n\to X\sqcup Y$ lies entirely on $X$ or entirely on $Y$ by continuity of $\sigma$ and connectness of $\Delta^n$. Hence, we have $Sin_n(X\sqcup Y)=Sin_n(X)\sqcup Sin_n(Y)$. Therefore,
\[S_n(X\sqcup Y)=\Z Sin_n(X\sqcup Y)\cong S_n(X)\oplus S_n(Y)\]
And the isomorphism is naturally given by 
\[f:S_n(X)\oplus S_n(Y)\to S_n(X\sqcup Y)\]
\[(\sigma_1,\sigma_2)\mapsto \sigma_1+\sigma_2\tag{$\star$}\]
It is a bijection because of the structure of $\Z$-module. It is a homomorphism because 
\[f(\sigma_1,\sigma_2)+f(\sigma_3,\sigma_4)=\sum_{i=1}^4\sigma_i=f(\sigma_1+\sigma_3,\sigma_2+\sigma_4)\]
Therefore, $\star$ also defines isomorphisms between the kernel and image.
\[g:Z_n(X)\oplus Z_n(Y)\to Z_n(X\sqcup Y)\]
\[(\sigma_1,\sigma_2)\mapsto \sigma_1+\sigma_2\]
because $d(\sigma_1+\sigma_2)=d\sigma_1+d\sigma_2=0$. Similarly, we have 
\[h:B_n(X)\oplus B_n(Y)\to B_n(X\sqcup Y)\]
\[(\sigma_1,\sigma_2)\mapsto \sigma_1+\sigma_2\]
where if $d(c_1,c_2)=(\sigma_1,\sigma_2)$, then we have $d(c_1+c_2)=\sigma_1+\sigma_2$.
Hence, we have 
\[\varphi: H_n(X)\oplus H_n(Y)\to H_n(X\sqcup Y)\]
\[(\sigma_1 B_n(X),\sigma_2 B_n(Y))\mapsto (\sigma_1+\sigma_2)B_n(X\sqcup Y)\]
\\\phantom{qed}\hfill$\square$\\
\textbf{Problem 8:}\\
(a) The isomorphism follows naturally from the definition of 0th singualr homology group First, we have $Sin_1(X)$ is the set of all paths in $X$ and $Sin_0(X)$ is the set of all points in $X$. And $d:Sin_1(X)\to S_0(X)$ is a map that takes all paths to their end points. Or in other word, $B_0(X)=Im(d)$ defines an equivalence relation between two points $x,y\in X$ if there exists a path $\sigma$ such that $\sigma(0)=x$ and $\sigma(1)=y$. Also, we have $ker(d:S_0(X)\to S_{-1}(X))=S_0(X)$ because $S_{-1}(X)=0$. So $H_0(X)=S_0(X)/B_0(X)$. Hence, the homomorphism is naturally given by 
\[\varphi:\Z\pi_0(X)\to H_0(X)\]
\[x\mapsto x\]
where $x$ is a point of $X$ that represent some path connected component of $X$. The intepretation of the homology group shows that $\varphi$ is well-defined. The construction shows $\varphi$ is injective and surjective.\\
(b)Being a natural homomorphism/isomorphism means we didn't actually construct the morphism, but we found these morphisms from the definition of the objects. For example, in part a, we found the morphism from the definition of $H_0(X)$. In 2.b, we found the morphism $f$ by the definition of $S_n(X\sqcup Y)$ and the module structure of it. Then we define homomorphism $g,h$ by $f$. And, $\varphi$ is defined by $g,h$.
\\\phantom{qed}\hfill$\square$
\end{document}