\documentclass[12pt]{amsart}
\usepackage{amsmath,epsfig,fancyhdr,amssymb,subfigure,setspace,fullpage,mathrsfs}
\usepackage[utf8]{inputenc}

\newcommand{\R}{\mathbb{R}}
\newcommand{\C}{\mathbb{C}}
\newcommand{\Z}{\mathbb{Z}}
\newcommand{\N}{\mathbb{N}}
\newcommand{\G}{\mathcal{N}}
\newcommand{\A}{\mathcal{A}}
\newcommand{\sB}{\mathscr{B}}
\newcommand{\sC}{\mathscr{C}}
\newcommand{\sd}{{\Sigma\Delta}}
\newcommand{\bigO}{\mathcal{O}}
\newcommand{\df}{\textbf{Definition: }}
\newcommand{\thm}{\textbf{Theorem }}
\newcommand{\rmk}{\textbf{Remark: }}
\newcommand{\lem}{\textbf{Lemma: }}
\newcommand{\cor}{\textbf{Corollay: }}
\newcommand{\x}{$x$ }
\newcommand{\X}{$X$ }

\title{Topology Note - Munkres}
\begin{document}
\maketitle

\section*{Section 29: Local Compactness}
\df A space $X$ is said to be locally compact at $x$ if there is some compact subspace $C$ of $X$ that contains a neighborhood of $x$. If $X$ is locally compact at each of its points, $X$ is said simply to be locally compact.
\smallbreak
\thm \textbf{29.1: } Let $X$ be a space. Then $X$ is locally compact Hausdorff if and only if there exists a space $Y$ satisfying the following conditions:
\begin{enumerate}
    \item $X$ is a subspace of $Y$.
    \item The set $Y - X$ consists of a single point.
    \item  $Y$ is a compact Hausdorff space.
\end{enumerate}
If $Y$ and $Y'$ are two spaces satisfying these conditions, then there is a homeomorphism of $Y$ with $Y'$ that equals to the identity on $X$.
\smallbreak
\df If $Y$ is a compact Hausdorff space and $X$ is a proper subspace of $Y$ whose closure equals $Y$, then $Y$ is said to be a compactification of $X$. If $Y-X$ equals a single point, then $Y$ is called the one-point compactification of $X$.
\smallbreak
\thm \textbf{29.2: } Let $X$ be a Hausdorff space. Then $X$ is locally compact if and only if given $x$ in $X$, and given neighborhood $U$ of $x$, there is a neighborhood $V$ of $x$ such that $\bar{V}$ is compact and $\bar{V}\subset U$.

\smallbreak
\cor Let $X$ be locally compact Hausdorff; let $A$ be a subspace of $X$. If $A$ is closed in $X$ or open in $X$, then $A$ is locally compact.
\smallbreak
\cor A space $X$ is homemorphic to an open subspace of a compact Hausdorff space if and only if $X$ is locally compact Hausdorff.
\pagebreak
\section*{Section 30: Countability Axioms}
\df A space $X$ is said to have a countable basis at $x$ if there is a countable collection $\mathscr{B}$ of neighborhood of $x$ such that each neighborhood of $x$ contains at least one of the elements of $\mathscr{B}$. A space that has a countable basis at each of its points is said to satisfy the first Countability Axiom, or to be first-countable.
\smallbreak
\thm \textbf{30.1} Let $X$ be a topological space.\\
(a) Let $A$ be a subset of $X$. If there is a sequence of points of $A$ convergening to $x$, then $x\in A$; the converse holds if $X$ is first-countable.\\
(b) Let $f: X\rightarrow Y$. If $f$ is continuous, then for every convergent sequence $x_n\rightarrow x$ in $X$, the sequence $f(x_n)$ converges to $f(x)$. The converse holds if $X$ is first-countable.
\smallbreak
\df If a space $X$ has a countable basis for the topology, then $X$ is said to satisfy the second countability axiom, or to be second-countable.
\smallbreak
\thm \textbf{30.2:} A subspace of a first-countable space(second-countable) is first-countable(second-countable). A countable product of first-countable(second-countable) spaces is frist-countable(second-countable).
\smallbreak
\df A subset $A$ of a space $X$ is dense if $\bar{A}=X$.
\smallbreak
\thm\textbf{30.3: }Suppose that $X$ has a countable basis. Then:\\
\indent (a) Every open covering of $X$ contains a countable subcollection covering $X$.\\
\indent (b) There exists a countable subset of $X$ that is dense in $X$.
\pagebreak
\section*{Section 31: The Separation Axioms}
\df Suppose that one-point sets are closed in $X$. Then $X$ is said to be regular if for each pair consisting of a point $x$ and a closed set $B$ disjoint from $x$, there exists disjoint open sets containing $x$ and $B$, respectively. The space $X$ is said to be normal if for each pair $A,B$ of disjoint closed sets of $X$, there exists disjoint open sets containing $A$ and $B$, respectively.
\smallbreak
\lem Let $X$ be a topological space. Let one-point sets in $X$ be closed.\\
\indent (a) $X$ is regular if and only if given a point \x of \X and a neighborhood $U$ containing $A$, there is an open set $V$ containing $A$ such that $\bar{V}\subset U$.\\
\indent (b) \X is normal if and only if given a closed set $A$ and an open set $U$ containing $A$, there is an open set $V$ containing $A$ such that $\bar{V} \subset U$.\\
Analogs of subspace and products:\\
\indent \thm \textbf{31.2: } A subspace of a Hausdorff space is Hausdorff; a product of Hausdorff spaces is Hausdorff. Same for regular. But normal space doesn't have such analogs.
\section*{Section 32: Normal Spaces}
\thm \textbf{1:} Every regular space with a countable basis is normal.\\
\indent \thm \textbf{2:} Every metrizable space is normal.\\
\indent \thm \textbf{3:} Every compact Hausdorff space is normal.\\
\indent \thm \textbf{4:} Every well-ordered set $X$ is normal in the order topology.\\
\pagebreak
\section*{Section 33: Urysohn Lemma}
\indent \thm \textbf{: (Urysohn Lemma)} Let $X$ be a normal space,; let $A$, $B$ be disjoint closed subsets of $X$. Let $[a,b]$ be a closed interval in the real line,. Then there exists a continuous map $$f:X \rightarrow [a,b]$$ such that $f(x)=a$ for every $x$ in $A$ and $f(x)=b$ for every $x$ in $B$.\\
\indent \df If $A$ and $B$ are subsets of a topological space $X$, and if there is a continuous function $f:X \rightarrow [0,1]$ such that $f(A)=\{0\}$ and $f(B)=\{1\}$, then we say that $A$ and $B$ can be separated by a continuous function.
\indent \df A space $X$ is said to be completely regular if one-points sets of $X$ are closed in $X$ and if for each $x_0$ and each closed set $A$ not containing $x_0$, there is a continuous function $f:X \rightarrow [0,1]$ such that $f(x_0)=1$ and $f(A)=\{0\}$.\\
\indent \thm \textbf{2:} A subspace of completely regular space is completely regular. A product of completely regular spaces is completely regular.
\section*{Section 34: The Urysohn Metrization Theorem}
\indent \thm \textbf{1: (Urysohn Metrization Theorem)} Every regular space $X$ with a countable basis is metrizable. \\
\indent \thm \textbf{2: (Imbedding Theorem)} Let $X$ be a space in which one-point sets are closed. Suppose that $\{f_\alpha\}_{\alpha\in J}$ is an indexed family of continuous functions $f_\alpha: X \rightarrow \R $ satisfying the requirement that for each point $x_0$ of $X$ and each neighborhood of $U$ of $x_0$, there is an index $\alpha$ such that $f_\alpha$ is positive at $x_0$ and vanishes outside $U$. Then the function $F:X\rightarrow \R^J$ defined by 
\[F(x)= (f_\alpha(x))_{\alpha\in J}\]
is an embedding of $X$ in $\R^J$. If $f_\alpha$ maps $X$ into $[0,1]$ for each $\alpha$, then $F$ embeds $X$ into $[0,1]^J$.\\
\indent \thm \textbf{3:} A space is completely regular if and only if it is a subspace of $[0,1]^J$ for some $J$.

\pagebreak
\section*{Section 37: The Tychonoff Theorem}
\indent \lem Let $X$ be a set; let $\mathcal{A}$ be a collection of subsets of $X$ having the finite intersection property. Then there is a collection $\mathcal{D}$ of subsets of $X$ such that $\mathcal{D}$ contains $\mathcal{A}$, and $\mathcal{D}$ has the finite intersection property, and no collection of subsets of $X$ that properly contains $\mathcal{D}$ has this property.\\
\indent \lem Let $X$ be a set; let $\mathcal{D}$ be a collection of subsets of $X$ that is a maximal with respect to finite intersection property. Then:\\
\begin{itemize}
    \item[(a)] Any finite intersection of elements in $\mathcal{D}$ is an element in $\mathcal{D}$.
    \item[(b)] If $A$ is a subset of $X$ that intersects every element of $\mathcal{D}$, then $A$ is an element of $\mathcal{D}$.
\end{itemize}\phantom{a}\\
\indent \thm \textbf{(Tychonoff Theorem):} An arbitrary product of compact sets is compact in the product topology.
\end{document}

