\documentclass[12pt]{amsart}
\usepackage{amsmath,epsfig,fancyhdr,amssymb,subfigure,setspace,fullpage,mathrsfs,upgreek,tikz-cd,todonotes}
\usepackage[utf8]{inputenc}

\newcommand{\R}{\mathbb{R}}
\newcommand{\Q}{\mathbb{Q}}
\newcommand{\C}{\mathbb{C}}
\newcommand{\Z}{\mathbb{Z}}
\newcommand{\N}{\mathbb{N}}
\newcommand{\F}{\mathbb{F}}
\newcommand{\G}{\mathcal{N}}
\newcommand{\A}{\mathcal{A}}
\newcommand{\sB}{\mathscr{B}}
\newcommand{\sC}{\mathscr{C}}
\newcommand{\Orbit}{\mathcal{O}}
\newcommand{\normal}{\triangleleft}
\newcommand{\Hom}{\operatorname{Hom}}
\newcommand{\Ext}{\operatorname{Ext}}
\newcommand{\Tor}{\operatorname{Tor}}
\newcommand{\If}{\text{ if }}
\newcommand{\Else}{\text{ else}}
\begin{document}
\title{Homework 6 - 290B}
\maketitle
\begin{center}
    Jiayi Wen\\
    A15157596
\end{center}
Source Consulted: We refer to Prof. Lin's lecture notes, Miller's notes and Hatcher's book. \\
\section*{Problem 1.}
\textbf{(1):} By cellular cohomology, we have
\[H^n(S^2;\Z)=\begin{cases}
        \Z & \If n=0,2 \\
        0  & \Else
    \end{cases}\]
Then by Kunneth formula, we have
\[H^n(S^2\times S^2;\Z )\cong \bigoplus_{p+q=n}H^p(S^2;\Z)\otimes_\Z  H^q(S^2;\Z )=\begin{cases}
        \Z          & \If n=0,4 \\
        \Z\oplus \Z & \If n=2   \\
        0           & \Else
    \end{cases}\]
since the torsion term vanishes. Since the isomorphism in the Kunneth formula is induced by the cross product, if we pick $\alpha,\beta$ as the generators of $H^2(S^2;\Z)$ (one for each copy of $S^2$), then we have the cohomology ring is
\[H^\ast(S^2\times S^2;\Z)\cong \Z\langle 1,\alpha,\beta,\alpha\smile \beta\rangle\]
Now, we want to compute the cup product.
$$\alpha\smile\beta=(\alpha\times 1)\smile(1\times \beta)=(-1)^4(\alpha\smile 1)\times (1\smile \beta)=\alpha\times \beta$$
This is just the cross product
\[\times : H^2(S^2;\Z)\otimes H^2(S^2;\Z)\to H^4(S^2\times S^2;\Z)\]
If $\sigma:\Delta^4\to S^2\times S^2$ is a singualr chain, then we denote $\sigma_1=p_1\circ \sigma\circ i_1$ and $\sigma_2=p_2\circ \sigma\circ i_2$, where $i_1:\Delta^2\to \Delta^4$ is the front face inclusion and $i_2:\Delta^2\to \Delta^4$ is the back face inclusion, $p_1,p_2$ are the projection map to the first coordinate and the second coordinate, respectively. Then we have
\[\alpha\times \beta(\sigma)=(-1)^{|\alpha||\beta|}\alpha(\sigma_1)\beta(\sigma_2)=(-1)^4\alpha(\sigma_1)\beta(\sigma_2)=\alpha(\sigma_1)\beta(\sigma_2)\]
So we have $\alpha\smile\beta(\sigma)=\alpha(\sigma_1)\beta(\sigma_2)$ is nontrivial since $\alpha$ and $\beta$ is nontrivial.
\qed\\
\textbf{(2):} Consider the Mayer-Vietoris sequence (on reduced cohomology), we have
\[0=\tilde{H}^{m-1}(\ast;\Z)\to \tilde{H}^{m}(X\vee Y;\Z)\xrightarrow[]{\cong} \tilde{H}^{m}(X;\Z)\oplus \tilde{H}^{m}(Y;\Z)\to \tilde{H}^{m}(\ast;\Z)=0\]
where the middle map is induced by the inclusion map. So we have the inclusion map $(i_1^\ast,i_2^\ast)$ is an isomorphism. Now, we denote $i^{\ast}_m=(i_1^\ast,i_2^\ast)$ be the induced isomorphism at dimension $m$. Since $\smile=\Delta^\ast \circ \times$, we have the following diagram
\begin{center}
    \begin{tikzcd}
        H^m(X\vee Y;\Z)\otimes H^n(X\vee Y;\Z)\arrow[r,"i_m^\ast\otimes i_n^\ast"]\arrow[d,"\times"] & (H^m(X;\Z)\oplus H^m(Y;\Z))\otimes (H^n(X;\Z)\oplus H^n( Y;\Z)) \arrow[d,"\times"]\\
        H^{m+n}(X\vee Y\times X\vee Y;\Z) \arrow[d,"\Delta^\ast"] \arrow[r,"i_{m+n}^\ast "] &
        H^{m+n}(X\times X;\Z)\oplus H^{m+n}(Y\times  Y;\Z)\arrow[d,"\Delta^\ast"]\\
        H^{m+n}(X\vee Y;\Z) \arrow[r,"i_{m+n}^\ast"]& H^{m+n}(X;\Z) \oplus H^{m+n}(Y;\Z)
    \end{tikzcd}
\end{center}
To prove the isomorphism preserves the cup product, it is equivalent to say the following diagram commutes. By the naturality of the cross product, we have the top square commutes.
By the naturality of Mayer-Vietoris sequence, we have the bottom square commutes. So the whole diagram commutes.
\\\qed\\
\textbf{(3):} If we consider the CW-structure of $S^2$ as one 0-cell (identify as $\{p\}$) and one 2-cell, then $Y$ is a CW-subcomplex of $S^2\times S^2$. In particular, $Y$ is the 2-skeleton of $S^2\times S^2$. Notice that $S^2\times S^2$ consists of one 4-cell, two 2-cells, and one 0-cell. So $S^2\times S^2/Y$ has one 0-cell and one 4-cell. By cellular cohomology, we have
\[H^2(S^2\times S^2,Y;\Z)\cong H^2(S^2\times S^2/Y;\Z)=0\]
\[H^3(S^2\times S^2,Y;\Z)\cong H^3(S^2\times S^2/Y;\Z)=0\]
Now, putting into the long exact sequence of relative cohomology, we have
\[0\to H^(S^2\times S^2;\Z )\xrightarrow[]{i^\ast}H^2(Y;\Z)\to 0\]
So the inclusion map $i:Y\to S^2\times S^2$ induces an isomorphism on dimension 2. Suppose there is a retraction $r:S^2\times S^2\to Y$, then we have $r\circ i =1_Y$. So we have
\[ i^\ast \circ r^\ast=1_{H^2(Y;\Z)}\]
Hence, $r^\ast$ must be an isomorphism as well. By naturality of the cup product, we have
\begin{center}
    \begin{tikzcd}
        H^2(Y;\Z)\otimes H^2(Y;\Z)\arrow[r,"\smile"] \arrow[d,"r^\ast"] & H^4(Y;\Z)\arrow[d,"r^\ast"]\\
        H^2(S^2\times S^2;\Z)\otimes H^2(S^2\times S^2;\Z) \arrow[r,"\smile"]& H^4(S^2\times S^2;\Z)
    \end{tikzcd}
\end{center}
But since $Y$ has dimension 2, so $H^4(Y;\Z)=0$. If we pick an nonzero element $\alpha\otimes \beta\in H^2(Y;\Z)\otimes H^2(Y;\Z)$, then we go along the induced map first, then the cup product, we get something nonzero since the induced map on $H^2$ is isomorphism and by part 1, we know the cup product is nontrivial. But if we go along the cup product first, then we will have $\alpha\smile \beta=0$. So we have $r^\ast(\alpha\smile\beta)=r^\ast(0)=0$. It contradicts. So such retraction doesn't exist.\\
\qed\\
\section*{Problem 2.} Suppose $g_i=\# genus(\Sigma_i)$, then we have $H^1(\Sigma_1;\F_2)\cong \F_2^{2g_1}$ and $H^1(\Sigma_2;\F_2)\cong \F_2^{2g_2}$. So we have $f^\ast:H^1(\Sigma_2;\F_2)\to H^1(\Sigma_1;\F_2)$ is not injective because $g_2>g_1$. Choose $\alpha\in H^1(\Sigma;\F_2)$ such that $\alpha\in \ker(f^\ast)$. Then P.D., we know $\alpha$ is represented by a loop on a copy of $T^2$ in $\Sigma_2$, where $\Sigma_2=\#_{2g_2}T^2$ is orientable. Then we can pick $\beta\in H^1(\Sigma_2;\F_2)$ such that $\beta$ another loop on the same copy of $T^2$, which intersect with $\alpha$. Then we have the intersection $\alpha\cdot\beta=\#(\alpha\cap\beta)=1$. Now, by the naturality of P.D., we have
\[\begin{tikzcd}
        H^1(\Sigma_2;\F_2)\times H^1(\Sigma_2;\F_2)\arrow[r,"\cdot"]\arrow[d,"f^\ast"]&H^2(\Sigma_2;\F_2)\arrow[d,"f^\ast"]\\
        H^1(\Sigma_1;\F_2)\times H^1(\Sigma_1;\F_2)\arrow[r,"\cdot"]& H^2(\Sigma_1;\F_2)
    \end{tikzcd}\]
Notice that if we go along $f^\ast$ first, then the intersection pairing, we have
\[f^\ast(\alpha)\cdot f^\ast(\beta)=0\cdot f^\ast(\beta)=0\]
But if we go along the intersection pairing first, we have
\[f^\ast(\alpha\cdot \beta)=f^\ast(1)=0\]
So we have $f^\ast:H^2(\Sigma_2;\F_2)\to H^2(\Sigma_1;\F_2)$ is trivial.
\\\qed\\
\section*{Problem 3.}
\textbf{(1):} Consider the relative homology, we have
\[\cdots \to H^l(\R P^n,\R P^k;\F_2)\to H^l(\R P^n;\F_2)\xrightarrow[]{f^\ast}H^l(\R P^k;\F_2)\xrightarrow[]{\partial}H^{l+1}(\R P^n,\R P^k;\F_2)\to \cdots \]
Since $(\R P^n,\R P^k)$ is a CW-pair, we have
\[H^l(\R P^n,\R P^k;\F_2)=H^l(\R P^n/\R P^k;\F_2)=\begin{cases}
        \Z/2 & \If l=0, or\  k+1\leq l \leq n \\
        0    & \Else
    \end{cases}\]
Also, we know
\[H^l(\R P^n;\F_2)\cong H^l(\R P^k;\F_2)\cong \F_2,\ \forall 0\leq l\leq k\]
So if $l=0$, then we have $f^\ast$ is surjective because $H^1(\R P^n,\R P^k;\F_2)=0$. But there is only two endomorphism of $\F_2$, namely the trivial map and the identity map. So $f^\ast$ is the identity map if it is surjective. Similarly, if $l=k$, then we have $f^\ast$ is injective because $H^k(\R P^n,\R P^k;\F_2)=0$, which implies $f^\ast$ is the identity map. If $1\leq l\leq k-1$, then we have $f^\ast$ is an isomorphism because $H^l(\R P^n,\R P^k;\F_2)=H^{l+1}(\R P^n,\R P^k;\F_2)=0$.
\\\qed\\
\textbf{(2):} We induct on $n$. It is obvious that when $n=1$, we have $H^\ast(\R P^1;\F_2)\cong \F_2[x]/(x^2)$ since $H^2(\R P^1;\F_2)=0$. Now, we suppose it is true for $n=k$. Then we have $x^k$ is a generator of $H^k(\R P^{k+1};\F_2)$ since $H^k(\R P^{k+1};\F_2)\cong H^k(\R P^{k};\F_2)\cong \F_2$ by part 1 and any nonzero element generates the subspace because we have coefficient mod 2.\\
For $n=k+1$, we consider the intersection pairing
\[x\cdot x^k=\langle x\smile x^k,[\R P^{k+1}]\rangle\]
Since the pairing is perfect, we know the mapping
\[x\mapsto \langle x\smile -,[\R P^{k+1}]\rangle\]
is an isomorphism. Hence, we have
\[x\cdot x^k=\langle x\smile x^k,[\R P^{k+1}]\rangle\neq 0\]
because $x\neq 0$ implies the homomorphism $\langle x\smile -,[\R P^{k+1}]\rangle$ is nontrivial and $x^k$ is the generator of $H^k(\R P^{k+1};\F_2)$, which must have nontrivial image.
So we have $x^{k+1}\neq 0$. But we have $x^{k+2}=0$ since $\R P^{k+1}$ has no $k+2$-cell, which implies trivial cohomology at k+2 dimension. The induction completes.
\\\qed\\
\section*{Problem 4.}
\textbf{(1): }Suppose $n>1$. Then we have $H_1(\R P^n;\Z)\cong \Z/2\Z$. And the generator is a loop $\alpha$ with winding number 1. Since $f$ is odd, we have $f\circ \alpha$ has odd winding number. Hence, we have $f_\ast(1)=1$. So $f_\ast$ is nontrivial.\\
If $n=1$, then $H_1(\R P^n;\Z)\cong  \Z$. The induced map is nontrivial since $f\circ \alpha$ cannot be null-homotopic given $f$ is odd.\\\qed\\
\textbf{(2):} Since $f^\ast$ is defined by precomposing $f_\ast$, so for a generator $\alpha\in H^1(\R P^n;\Z/2\Z)$, we have $\alpha: H_1(\R P^n;\Z )\to \Z/2\Z$ is nontrivial. Then we have
\[f^\ast(\alpha)=\alpha\circ f_\ast\]
This is a nontrivial map since $f_\ast$ is nontrivial by part 1. So we have $f^\ast: H^1(\R P^n;\Z/2\Z)\to H^1(\R P^m;\Z/2\Z)$ is nontrivial.
\\\qed\\
\textbf{(3)} Suppose there is such an odd map towards contradiction. By part 2, we know the induced map of $f^\ast$ sends the generator to the generator since the only nontrivial endomorphism $\Z/2\Z$ is the identity map. By the naturality of cup product, we know $f^\ast(x^k)=f^\ast(x)^k=x^k$, where $x$ is the generator of $H^1(\R P^n;\F_2)$. So the induced map on the cohomology ring is
\[f^\ast: \F_2[x]/(x^{n+1})\to \F_2[x]/(x^{m+1})\]
\[x\mapsto x\]
We have $0=f^\ast(x^m)$ since $x^m=0$ in $H^1(\R P^n;\F_2)$ by the assumption $n<m$. But we also have $f^\ast(x^m)=f^\ast(x)^m=x^m\neq 0$ in $H^1(\R P^m;\F_2)$. It contradicts.
\\\qed\\ 
\end{document}