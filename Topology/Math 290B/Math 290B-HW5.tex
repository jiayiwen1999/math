\documentclass[12pt]{amsart}
\usepackage{amsmath,epsfig,fancyhdr,amssymb,subfigure,setspace,fullpage,mathrsfs,upgreek,tikz-cd,todonotes}
\usepackage[utf8]{inputenc}

\newcommand{\R}{\mathbb{R}}
\newcommand{\Q}{\mathbb{Q}}
\newcommand{\C}{\mathbb{C}}
\newcommand{\Z}{\mathbb{Z}}
\newcommand{\N}{\mathbb{N}}
\newcommand{\G}{\mathcal{N}}
\newcommand{\A}{\mathcal{A}}
\newcommand{\sB}{\mathscr{B}}
\newcommand{\sC}{\mathscr{C}}
\newcommand{\sd}{{\Sigma\Delta}}
\newcommand{\Orbit}{\mathcal{O}}
\newcommand{\normal}{\triangleleft}

\begin{document}
\title{Homework 5 - 290B}
\maketitle
\begin{center}
    Jiayi Wen\\
    A15157596
\end{center}
Source Consulted: We refer to Prof. Lin's lecture notes, Miller's notes and Hatcher's book. \\
\textbf{Problem 1:} Consider the following short exact sequence
\begin{center}
    \begin{tikzcd}
        0 \arrow[r] & \mathbb{Z} \arrow[r, hook] & \mathbb{Q} \arrow[r, two heads] & \mathbb{Q}/\mathbb{Z} \arrow[r] & 0
        \end{tikzcd}
\end{center}
where the first map is the inclusion and the second map is the quotient map. Then by prop 3A.5 (6) in Hatcher's book, we have a naturally associated long exact sequence
\begin{center}
    \begin{tikzcd}
        0 \to {Tor(A,\mathbb{Z})} \to Tor(A,\mathbb{Q}) \to  Tor(A,\mathbb{Q}/\mathbb{Z}) \arrow[r] & A\otimes_\Z \mathbb{Z} \arrow[r] & A\otimes_\Z \mathbb{Q} \to A\otimes_\Z \mathbb{Q}/\mathbb{Z} \to 0
        \end{tikzcd}
\end{center}
But since $\Z$ and $\Q$ are torsion-free, the first two term in the sequence is trivial. So we can just consider the last four terms.
\begin{center}
    \begin{tikzcd}
        0 \to Tor(A,\mathbb{Q}/\mathbb{Z}) \arrow[r,"f"] & A\otimes_\Z \mathbb{Z} \arrow[r,"g"] & A\otimes_\Z \mathbb{Q} \to A\otimes_\Z \mathbb{Q}/\mathbb{Z} \to 0
        \end{tikzcd}
\end{center}
Hence, we have $\ker(g)=Im(f)\cong Tor(A,\Q/\Z)$ by exactness. All we need to know is the map $g$. Since $g$ is the induced map on homology of the first map
\[0\to F_0\otimes_\Z \Z \to F_0\otimes_\Z \Q \to F_0\otimes_\Z \Q/\Z\to 0\]
where $0\to F_1\to F_0\to A\to 0$ is a free resolution of $A$. Since $F_0$ is free, the first map is the direct sum of the inclusion map from $\Z$ to $\Q$, which is the same as $1_{F_0}\otimes i$ where $i:\Z\to \Q$ is the inclusion. Hence, the induced map on homology is $g=1_A\otimes i$. Hence, we have $g(a\otimes n)=a\otimes n=ma\otimes \frac{n}{m}$ for any $m\in \Z\setminus \{0\}$. Hence, $a\otimes n\in \ker(g)$ if and only if $a$ has finite order, which is the same as saying $a$ is a torsion element in $A$. Also, we have $A\otimes_\Z \Z=A$ by the property of tensor product. So we have $\ker(g)=Tors(A)\cong Tor(A,\Q/\Z)$.\\
So if we have $Tor(A,B)=0$ for all abelian groups $B$, then we can set $B=\Q/\Z$, then we have $Tors(A)\cong Tor(A,\Q/\Z)=0$, which implies $A$ is torsion-free.\\
Conversely, if $A$ is torsion-free, then by prop 3A.5(3), we have $Tor(A,B)=0$ for all $B$.
\qed\\
\textbf{Problem 2:} By prop 3A.5(4), we have 
\[Tor(A,B)=Tor(Tors(A),B)=Tor(Tors(A),Tors(B))\]
So without loss of generality, we can assume $A,B$ are torsion $\Z$-modules. Let $0\to F_1\to F_0\to A\to 0$ be a free resolution of $A$. Then $Tor(A,B)=\ker(F_1\otimes_\Z B\to F_0\otimes_\Z B)\leq F_1\otimes_\Z B$ is a subgroup. Since $F_1$ is free, we have $F_1\cong \oplus_I \Z$ for some index set $I$ and $F_1\otimes_\Z B\cong \oplus_I (\Z\otimes_\Z B)\cong \oplus_I B$. For any element $(\alpha_i)_{i\in I}\in \oplus_I B$, there are only finitely many $\alpha_i$ are nonzero, which is torsion by some element $n_i\in Z$. Let's denote the index set of nonzero coordinates of $(\alpha_i)_{i\in I}$ by $I_\alpha$. Then we have $\prod_{i\in I_\alpha}n_i\neq 0$ since $\Z$ is an integral domain and  $(\prod_{i\in I_\alpha}n_i)(\alpha_i)_{i\in I}=0$. So $\oplus_I B$ is torsion. Hence, any subgroup of it is also torsion. In particular, $Tor(A,B)$ is torsion.\\
If $Tor(A,B)$ has an element of order $n$, by previous argument, we see $Tor(A,B)$ is a subgroup of $\oplus_I B$. By assumption, there exists $(\alpha_i)_{i\in I}\in Tor(A,B)$ has order $n$. So we know $(\alpha_i)_{i\in I}\neq 0$, which implies not all $\alpha_i$ are zeros. And $n(\alpha_i)=0$ implies $n$ is the lcm of the order of nonzero coordinates. Now if we let $I_\alpha$ be the index set of all nonzero coordinates, then we have $\prod_{i\in I_\alpha} \alpha_i\in B$ has order $n$. Similarly, if we switch the notation, then $Tor(A,B)$ is the direct sum of $A$'s and we can do the same proof. So both $A$ and $B$ have elements of order $n$.\\
If both $A$ and $B$ have elements of order $n$, we need to understand more explicit about $Tor(A,B)$. By the explicity construction of free resolution given in the lecture, we take $F_0\cong \Z[A]\cong \oplus_A \Z$, the free abelian group generated by $A$ (as a set). Then we have $F_1'=\ker(F_0\to A)$ and $F_1=\oplus_{F_1'}\Z$, where $g:F_1\to F_0$ by sending $g((n_\alpha)_{\alpha\in F_1'})=\sum_{\alpha\in F_1'}n_\alpha\alpha$.
So if $a\in A$ has order $n$, then we take $(n_i)_A\in F_0$ such that $n_i=0$ for all $i\neq a$ and $n_a=n$. Then we have $(n_i)_A\in F_1'\neq \emptyset$. So $F_1$ has at least one coordinate. We denote $\beta=(n_i)_A$. Let $(l_j)_{j\in F_1'}\in F_1$ be an element such that $l_j=0$ for all coordinate except $j=\beta$. Then we have $g((l_j)_{j\in F_1'})=\beta=(n_i)_A$.\\
Now, we assume $b\in B$ also has order $n$. Then we have 
\[(g\otimes id_B)\Big((l_j)_{j\in F_1'}\otimes b\Big)=\beta\otimes b=(n_i)_A\otimes b=(n_i')_A\otimes nb=0\]
where $n_i'=0$ if $i\neq a$ and $n_a'=1$ if $i=a$.
So we have $\Big((l_j)_{j\in F_1'}\otimes b\Big)\in \ker(g\otimes id_B)=Tor(A,B)$. Notice that $\Big((l_j)_{j\in F_1'}\otimes b\Big)\in F_1\otimes B\cong (\oplus_{F_1}\Z)\otimes B\cong \oplus_{F_1'}B$ and under these isomorphisms, $\Big((l_j)_{j\in F_1'}\otimes b\Big)$ is cooresponding to an element with all coordinates are zero except one coordinate is equal to $b$. So the order of this element is the same as the order of $b$, which is $n$. So we find an element of order $n$ in $Tor(A,B)$.
\\\qed\\
\textbf{Problem 3:}\\
\textbf{(1):} We prove by induction on $k$. If $k=0$, then we have $T_0=X_0\times [0,1]/\sim$. Notice that $\sim$ doesn't identify any two points. So we have $T_0=X_0\times [0,1]$, which is homotopic to $X_0\times \{1\}$.\\
Now, we suppose $X_{k-1}\times\{k\} $ is a deformation retracts of $T_{k-1}$. By the construction, we have $T_k=(T_{k-1}\sqcup X_k\times [k,k+1])/\sim$. If $H:T_{k-1}\times I\to T_{k-1}$ is the deformation retracts of $T_{k-1}$ onto $X_{k-1}\times \{k\}$, then we define 
\[H':T_k\times I\to X_{k-1}\times \{k\}\sqcup X_k\times [k,k+1]\]
\[H_t'(\alpha)=\begin{cases}
    \alpha &\text{ if } \alpha\in X_k\times [k,k+1]\\
    H_t(\alpha) &\text{ if }\alpha\in T_{k-1}
\end{cases}\]
This is well-defined since the equivalence relation in $T_k$ identifies points in $X_{k-1}\times\{k\}$ with points in $X_k\times [k,k+1]$. By the construction of $H'$, we know $X_k\times [k,k+1]$ is fixed. Since $H$ is a deformation retract onto $X_{k-1}\times \{k\}$, we have $X_{k-1}\times\{k\}$ is also fixed. Hence, it is well-defined. The continuity follows by pasting lemma. So we have 
\[T_k\simeq (X_{k-1}\times\{k\}\sqcup X_k\times [k,k+1])/\sim\ =X_k\times [k,k+1]\simeq X_k\times \{k+1\}\]
So the induction completes.
\\\qed\\
\textbf{(2):} Let $H:T_k\times I \to T_k$ be the homotopy of the identify map (at $t=0$) and deformation retracts onto $X_k\times \{k+1\}$ (at $t=1$). Then we let $r:T_k\to X_k\times \{k+1\}= H_1$ be the retraction onto $H_k$ and $i':X_{k+1}\times \{k+1\}\to T_{k+1}$ be the inclusion. Now, denote $\phi=i'\circ f_k\circ r: T_k\to T_{k+1}$. Here, we extend the map $f_k$ to $X_k\times \{k+1\}\to X_{k+1}\times \{k+1\}$ as $f_k(x,k+1)=(f_k(x),k+1)$. We want to show $\phi\simeq i$. We define 
\[L:T_k\times I\xrightarrow[]{H}T_k\xrightarrow[]{i} T_{k+1}\]
This is a homotopy since $H$ is a homotopy. If $t=0$, then we have $L_0=i\circ H_0=i\circ id_{T_k}=i$. If $t=1$, then we have $L_1=i\circ H_1=i\circ r$. But notice that $H_1(T_k)=X_k\times \{k+1\}$, which is identified as $(x,k+1)\sim(f_k,k+1)$. So if $\alpha\in T_k$ such that $r(\alpha)=(x,k+1)$, then we have 
$L_1=i\circ r(\alpha)=i(x,k+1)=(f_k(x),k+1)=i'\circ f_k(x,k+1)=\phi$. So the following diagram of homologies commutes
\begin{center}
    \begin{tikzcd}
        H_\ast(T_k) \arrow[d,"\cong"] \arrow[r, "i_\ast"] & H_\ast(T_{k+1})           \\
        H_\ast(X_k) \arrow[r, "f_{k,\ast}"]         & H_\ast(X_{k+1}) \arrow[u,"\cong"]
        \end{tikzcd}
\end{center}
where the vertical arrows are isomorphisms given by the deformation retract in $(1)$. So they induce the same homomorphism.
\\\qed\\
\textbf{(3):} Consider the $\N$-directed system $T_0\hookrightarrow T_1\hookrightarrow \cdots $. We want to show the direct limit is $T_\infty$, where the morphisms are $\{i_j:T_j\hookrightarrow T_\infty\}$ inclusions. By the construction of $T_\infty$, we have the commutative diagram for all $j\in\N$.
\begin{center}
    \begin{tikzcd}
        T_j \arrow[rr,"i"] \arrow[rd, "i_j"] &          & T_{j+1} \arrow[ld, "i_{j+1}"] \\
                                         & T_\infty &                              
        \end{tikzcd}
\end{center}
So we have $i_j(x,s)=(x,s)$ for any $k\leq j$, $x\in X_k$ and $s\in[k,k+1]$.\\
For any topological space $A$ and morphisms $\{h_j:T_j\to A\}$, we define $h:T_\infty\to A$ by 
\[h(x,s)=h_j(x,s) \ \text{ if }x\in X_j, \ s\in [j+j+1]\]
Hence, $h\circ i_j(x,s)=h_j$. By the way we define $h$, it is unique. So $T_\infty$ is the direct limit of the $\N$-directed system.
Hence, we have 
\[H_\ast(T_\infty;\Z)\cong H_\ast(\varinjlim T_i;\Z)\cong \varinjlim H_\ast(T_i;\Z)\cong \varinjlim H_\ast(X_i;\Z) \]
where the first two isomorphisms are given by the fact that direct limit commutes with homology and the last one is given by part 1 (isomorphisms between the objects) and part 2 (the induced map on homology by inclusion are the same as induced by $f_k$).\\
\qed\\
\textbf{(4):} Consider the following $\N$-directed system
\[S^1\xrightarrow[]{z^2}S^1\xrightarrow[]{z^3}S^1\xrightarrow[]{z^4}\cdots\]
where we realize $S^1=\{z\in\C\mid |z|=1\}$. Now, if we take the direct limit, we have $T=\varinjlim S^1=S^1/\sim$, where $x\sim y$ if $y=x^{n!}$ for some $n\in \N$. Since the induced map on homology is 
\[H_1(S^1;\Z)\xrightarrow[]{\cdot 2}H_1(S^1;\Z)\xrightarrow[]{\cdot 3}\cdots\]
We have 
\[H_1(T;\Z)\cong \varinjlim (H_1(S^1;\Z)\xrightarrow[]{\cdot 2}H_1(S^1;\Z)\xrightarrow[]{\cdot 3}\cdots)\cong \varinjlim (\Z\xrightarrow[]{\cdot 2}\Z\xrightarrow[]{\cdot 3}\cdots)\cong \Q\]
And $H_n(T;\Z)\cong \varinjlim (0\xrightarrow[]{}0\to\cdots)\cong 0$ for all $n>1$.
\\\qed\\
\textbf{Problem 4:}
By cellular homology, we have 
\[H_n(\R P^2;\Z)=\begin{cases}
    \Z/2\Z & \ ,n=1\\
    \Z & \ ,n=0\\
0 & \text{ , else}
\end{cases}\]
So we have $H_i(X;\Z)\otimes H_0(\R P^2;\Z)\cong \Z/i\Z\otimes \Z\cong \Z/i\Z$ and $H_i(X;\Z)\otimes H_1(\R P^2;\Z)\cong \Z/i\Z\otimes \Z/2\cong \Z/\gcd(i,2)\Z$. Also, we have $Tor(H_i(X;\Z),H_0(\R P^2;\Z))\cong 0$ and $Tor(H_i(X;\Z),H_1(\R P^2;\Z))\cong\Z/i\Z\otimes \Z/2\Z\cong \Z/\gcd(i,2)\Z $
So by Kunneth formula, we have 
\[H_n(X\times \R P^2;\Z)\cong\begin{cases}
    \Z/n\Z\oplus \Z/2\Z\oplus \Z/2\Z & \text{ if $n$ is odd}\\  
    \Z/n\Z & \text{ if $n$ is even}\\  
\end{cases} \]
\qed\\
\textbf{Problem 5:} We identify the CW-structure on $S^2$ by one 0-cell and one 2-cell. By cellular homology, we have 
\[\cdots\to 0\to \Z\xrightarrow[]{\cdot 4}\Z\to 0\to\Z\to 0\to \cdots\]
So the homology of $X$ is 
\[H_n(X;\Z)\cong \begin{cases}
    4\Z&,\ n=3\\
    \Z/4\Z&, \ n=2\\
    \Z& , \ n=0\\
    0 &, \text{ else}
\end{cases}\]
And we know the homology of $\R P^3$ is 
\[H_n(\R P^2;\Z)=\begin{cases}
    \Z/2\Z & \ ,n=1,3\\
    \Z & \ ,n=0\\
0 & \text{ , else}
\end{cases}\]
Now, by Kunneth formula, we have 
\[H_6(X\times \R P^3;\Z)\cong 4\Z\otimes \Z/2\Z\oplus Tor(4\Z,\Z/2\Z)=0\oplus 0=0\]
The first tensor is zero since $4n\otimes i=2n\otimes 2i=2n\otimes 0=0$ and the second one is zero since $4\Z$ is torsion free.
\[H_5(X\times \R P^3;\Z)\cong \Z/4\Z\otimes \Z/2\Z\oplus Tor(\Z/4\Z,\Z/2\Z)\cong \Z/2\Z\oplus \Z/2\Z\]
\begin{align*}
    H_3(X\times \R P^3;\Z)&\cong 4\Z\otimes \Z\oplus \Z/4\Z\otimes \Z/2\Z\oplus \Z\otimes \Z/2\Z\oplus Tor(4\Z,\Z)\oplus\\
    &\  Tor(\Z/4\Z,\Z/2\Z)\oplus   Tor(\Z,\Z/2\Z)\\
    &\cong 4\Z\oplus \Z/2\Z\oplus \Z/2\Z\oplus \Z/2\Z
\end{align*}
\[H_2(X\times \R P^3;\Z)\cong \Z/4\Z\otimes \Z\oplus Tor(\Z/4\Z,\Z)\cong \Z/4\Z\]
\[H_1(X\times \R P^3;\Z)\cong \Z\otimes \Z/2\Z\oplus Tor(\Z,\Z/2\Z)\cong \Z/2\Z\]
\[H_0(X\times \R P^3;\Z)\cong \Z\otimes \Z\oplus Tor(\Z,\Z)\cong \Z\]
And $H_k(X\times \R P^3;\Z)=0$ for all $k>6$ or $k=4$
\qed\\
\textbf{Problem 6:} By cellular homology, we have 
\[H_k(M(\Z_m,n);\Z)=\begin{cases}
    m\Z & \ ,k=n+1\\
    \Z/m\Z & \ ,k=n\\
    \Z& \ ,k=0\\
0 & \text{ , else}
\end{cases}
    \] Let' focus on dimension-$(n+1)$ and by the naturality of Kunneth formula, we have
    \begin{center}
        \begin{tikzcd}[every arrow/.append style={shift left}]
            0 \arrow[r] & {m\Z\oplus m\Z} \arrow[d] \arrow[r] & {m\Z\oplus m\Z} \arrow[r] \arrow[d,"(f\times 1)_\ast"] &  0 \arrow[r] \arrow[d] & 0 \\
            0    \arrow[r]       & {\Z\oplus m\Z} \arrow[r]           & \Z\oplus m\Z \arrow[r]           & 0 \arrow[l, dashed]\arrow[r]           & 0
            \end{tikzcd}
                
    \end{center}
We denote the split by the dashed arrow. If it is natural, then the diagram commutes. Let's chase the diagram for the right square.\\
Since $f$ collapses the $n$-skeleton, but do nothing to the $n+1$ cell, so it has no effect on $n+1$ dimension homology. And notice on the middle term, where the $m\Z$ is corresponding to the homology of $M(\Z_m,n)$, which was fixed by $f\times 1$. So the induced map of $(f\times 1)_\ast$ is not trivial. Now, if we chase the diagram on the right square clockwise, we see $(m,m)\mapsto 0\mapsto 0\mapsto 0$. But if we chase along the vertical arrow, $(f\times 1)_\ast$, we have $(f\times 1)\ast(m,m)\neq 0$. So the split is not natural.
\qed\\




\end{document}