\documentclass[12pt]{amsart}
\usepackage{amsmath,epsfig,fancyhdr,amssymb,subfigure,setspace,fullpage,mathrsfs,upgreek}
\usepackage[utf8]{inputenc}

\newcommand{\R}{\mathbb{R}}
\newcommand{\Q}{\mathbb{Q}}
\newcommand{\C}{\mathbb{C}}
\newcommand{\Z}{\mathbb{Z}}
\newcommand{\N}{\mathbb{N}}
\newcommand{\G}{\mathcal{N}}
\newcommand{\A}{\mathcal{A}}
\newcommand{\sB}{\mathscr{B}}
\newcommand{\sC}{\mathscr{C}}
\newcommand{\sd}{{\Sigma\Delta}}
\newcommand{\Orbit}{\mathcal{O}}
\newcommand{\normal}{\triangleleft}

\begin{document}
\title{Homework 1 - 200B}
\maketitle
\begin{center}
    Jiayi Wen\\
    A15157596
\end{center}
\textbf{Problem 1:}\\
To prove the greatest common divsiors are equal is equivalent to prove the ideal generated by $a,b$ and the ideal generated by $b,r$ are the same. To prove this two ideal are the same, we can show the generators of one is contained in another.\\
$(a,b)\subseteq (b,r)$: It is obvious that $b\in (b,r)$, so we just need to show $a\in (b,r)$. However, we have $a=qb+r$ for some $q,r\in\Z$. By the properties of ideals, we have $qb\in (b,r)$ (closed under multiplication with elements in the ring) and $a=qb+r\in (b,r)$ (as an additive subgroup of $(\Z,+)$). So we have $(a,b)\subseteq (b,r)$ since $a,b\in (b,r)$.\\
$(b,r)\subseteq (a,b)$: Conversely, we have $b\in (a,b)$ implies $qb\in (a,b)$ and $a=qb+r$ implies $r=a-qb$. Since $(a,b)$ is an additive subgroup, it is closed under addition. So we have $r=a-qb\in (a,b)$ since $-qb\in (a,b)$ and $a\in (a,b)$. So we have $(b,r)\subseteq (a,b)$.
\\\phantom{qed}\hfill$\square$\\
\textbf{Problem 2:} From $a=qb+r$ where $q\neq 0$ and $0\leq r<b$, we have $(a,b)=(b,r)$ by Problem 1. From $b=q_1r+r_1$ where $q_1\neq 0$ and $0\leq r_1<r$, we have $(b,r)=(r,r_1)$ by Problem 1 again. Hence, we have $(a,b)=(b,r)=(r,r_1)$.\\
This must end in finitely many step because if we look at the value of $r_i$, we have an inequality of 
\[0\leq \dots < r_2 < r_1 < r <b\]
So every time we proceed the algorithm, we decrease the remainder by at least 1. Since $b$ is some fixed nonzero integer, the process can be done at most $b$ times. So it must end in finitely many times.\\
For the last result, we can apply Problem 1 $k+2$ times, then we have 
\[(a,b)=(b,r)\]
\[(b,r)=(r,r_1)\]
\[(r,r_1)=(r_1,r_2)\]
\[\vdots\]
\[(r_{k-1},r_k)=(r_k,r_{k+1})\]
\[(r_k,r_{k+1})=(r_{k+1},0)\]
If we combine them, we have $(a,b)=(b,r)=\cdots=(r_k,r_{k+1})=(r_{k+1},0)=(r_{k+1})$. By lemma 4, we know $r_{k+1}$ the greatest common divisor of $a$ and $b$. So we have $(a,b)=r_{k+1}$.
\\\phantom{qed}\hfill$\square$\\
\textbf{Problem 3:}
Let's follow the Euclidean algorithm. For $(187,221)$, we have 
\[187=(-1)\cdot 221+34\]
\[221=6\cdot 34+17\]
\[34=2\cdot 17\]
So we have $(187,221)=17$.
\\For $(6188,4709)$, we have 
\[6188=1\cdot 4709+1479\]
\[4079=3\cdot 1479 + 272\]
\[1479=5\cdot 272+ 119\]
\[272=2\cdot 119+34\]
\[119=3\cdot 34+17\]
\[34=2\cdot 17\]
So $(6188,4709)=17$.\\
For $(314,159)$, we have 
\[314=1\cdot 159+155\]
\[159=1\cdot 155+4\]
\[155=38\cdot 4+3\]
\[4=1\cdot 3+1\]
\[3=3\cdot 1\]
So we have $(314,159)=1$.
\\\phantom{qed}\hfill$\square$\\
\textbf{Problem 4:} For convenience, we want to label the equations in Problem 2 as the following
\[a=qb+r\tag{0}\]
\[b=q_1r+r_1\tag{1}\]
\[\vdots\]
\[r_{k-2}=q_{k}r_{k-1}+r_{k}\tag{k}\]
\[r_{k-1}=q_{k+1}r_k+r_{k+1}\tag{k+1}\]
Then by the $(k+1)$ equation, we have 
\[r_{k+1}=r_{k-1}-q_{k+1}r_k\]
If we replace $r_k$ by $(k)$ equation, we have 
\[r_{k+1}=r_{k-1}-q_{k+1}(r_{k-2}-q_kr_{k-1})=(1+q_{k+1}q_k)r_{k-1}-q_{k+1}r_{k-2}\]
Then we can replace $r_{k-1}$ by $(k-1)$ equation, etc. Since $r_i$ can be written as some linear combination of $r_{i-2}$ and $r_{i-1}$ by the $(i)$ equation, we will finally get some linear combination of $a,b$ at the end. Hence, one can use Euclidean algorithm to find a pair of $m,n$ such that $am+bn=d$.
\\\phantom{qed}\hfill$\square$\\
\textbf{Problem 8:} Suppose $x_1,y_1$ some solution of the Diophantine equation, $ax+by=c$. Then we can take the difference. So we have 
\[a(x_1-x_0)+b(y_1-y_0)=0\]
Hence, we have 
\[a(x_1-x_0)=-b(y_1-y_0)\tag{$\ast$}\]
Since $d=(a,b)$, then we know $\frac{b}{d}$ is coprime with $a$. Hence, $\frac{b}{d}\mid x_1-x_0$. So there exists some $t\in \Z$ such that $x_1-x_0=t\cdot \frac{b}{d}$. If we plug this into $(\ast)$, then we have 
\[-b(y_1-y_0)=\frac{atb}{d}\]
\[y_1-y_0=-\frac{ta}{d}\]
Hence, for any solution $(x,y)$ of $ax+by=c$, we can write $x=x_0+\frac{tb}{d}$ and $y=y_0+\frac{ta}{b}$.
\\\phantom{qed}\hfill$\square$\\
\textbf{Problem 16:} We want to use the unique factorization of integers. If either $u=1$ or $v=1$, then we are done since we will have $v=a^2$ or $u=a^2$ and $1=1^2$.\\
So we can suppose $u,v\neq 1$ and we suppose $u=p_1^{\alpha_1}p_2^{\alpha_2}\cdot p_r^{\alpha_r}$ and $v=q_1^{\beta_1}q_2^{\beta_2}\cdots q_s^{\beta_s}$, where $p_i,q_j$ are distinct primes with $\alpha_i,\beta_j\geq 1$. Notice that $p_i\neq q_j$ for any $i,j$ because $(u,v)=1$. Then we have 
\[uv=p_1^{\alpha_1}p_2^{\alpha_2}\cdot p_r^{\alpha_r}q_1^{\beta_1}q_2^{\beta_2}\cdots q_s^{\beta_s}=a^2\]
We can also factor $a=a_1^{\gamma_1}\cdots a_k^{\gamma_k}$, where $a_i$ are distinct primes and $\gamma_k\geq1$. So we have 
\[uv=a^2=a_1^{2\gamma_1}\cdots a_k^{2\gamma_k}\] 
By the uniqueness of the prime factorization, we should have $k=r+s$ and up to rearrangement, 
\[a_1=p_1, a_2=p_2, \cdots, a_r=p_r, a_{r+1}=q_1, \cdots, a_{r+s}=q_s\]
\[\alpha_1=2\gamma_1,\cdots , \alpha_r=2\gamma_r, \beta_1=2\gamma_{r+1},\cdots , \beta_s=2\gamma_{r+s}\]
Hence, we have 
\[u=(a_1^{\gamma_1}\cdots a_r^{\gamma_r})^2\]
\[v=(a_{r+1}^{\gamma_{r+1}}\cdots a_{r+s}^{\gamma_{r+s}})^2\]
So $u,v$ are squares.
\\\phantom{qed}\hfill$\square$\\
\textbf{Problem 18:}
We will prove by contradiction. Suppose $\sqrt[n]{m}$ is rational. Then there exists a pair of integers $a,b$ such that $(a,b)=1$, $b\neq 0$ and $  \sqrt[n]{m}=\frac{a}{b}$. If $b=1$, then we will have $m=(\sqrt[n]{m})^n=a^n$ which contradicts to the assumption that $m$ is not the $n$th power of an integer. So we can assume $b\neq 1$. Also, we have $a\neq 1$ because $(\frac{1}{b})^n<1$ is not an integer. By Unique prime factorization of integers, we can assume 
\[a= p_1^{\alpha_1}\cdots p_s^{\alpha_s}\] 
\[b= q_1^{\beta_1}\cdots q_t^{\beta_t}\] 
where $p_i,q_j$ are distinct primes and $\alpha_i,\beta_j\geq 1$ for all $1\leq i \leq s $ and $ 1\leq j \leq t$ because $a,b$ are coprimes.\\
Then we have 
\[m=(\sqrt[n]{m})^n=(\frac{p_1^{\alpha_1}\cdots p_s^{\alpha_s}}{q_1^{\beta_1}\cdots q_t^{\beta_t}})^n=\frac{p_1^{n\alpha_1}\cdots p_s^{n\alpha_s}}{q_1^{n\beta_1}\cdots q_t^{n\beta_t}}\]
Hence, we have
\[q_1^{n\beta_1}\cdots q_t^{n\beta_t}m=p_1^{n\alpha_1}\cdots p_s^{n\alpha_s}\]
By the uniqueness of the prime factorization, we should have $q_1=p_i$ for some $1\leq i \leq s $, but this contradicts to our assumption that $(a,b)$ are coprimes. Hence, $\sqrt[n]{m}$ is not rational. 
\\\phantom{qed}\hfill$\square$\\
\end{document}