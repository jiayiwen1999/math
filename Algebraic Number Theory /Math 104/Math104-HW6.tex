\documentclass[12pt]{amsart}
\usepackage{amsmath,epsfig,fancyhdr,amssymb,subfigure,setspace,fullpage,mathrsfs,upgreek,tikz-cd}
\usepackage[utf8]{inputenc}

\newcommand{\R}{\mathbb{R}}
\newcommand{\Q}{\mathbb{Q}}
\newcommand{\C}{\mathbb{C}}
\newcommand{\Z}{\mathbb{Z}}
\newcommand{\N}{\mathbb{N}}
\newcommand{\G}{\mathcal{N}}
\newcommand{\A}{\mathcal{A}}
\newcommand{\sB}{\mathscr{B}}
\newcommand{\sC}{\mathscr{C}}
\newcommand{\sd}{{\Sigma\Delta}}
\newcommand{\Orbit}{\mathcal{O}}
\newcommand{\normal}{\triangleleft}
\newcommand{\ord}{\mathrm{ord}}

\begin{document}
\title{Homework 6 - 104A}
\maketitle
\begin{center}
    Jiayi Wen\\
    A15157596
\end{center}
Notation: We will use $[a]_n$ to denote the congruence class of $a$ in $\Z/n\Z$. The subscript $n$ is omitted if the ring $\Z/n\Z$ is clear by the context.
\begin{center}
    \textbf{Chapter 4:}
\end{center}
\textbf{Problem 3:} If $a$ is a primitive root mod $p^n$, then we know $[a]_{p^n}$ has order $(p-1)p^{n-1}$. Also, we know $\gcd(a,p^n)=1$, hence $p\nmid a$. So $a\in (\Z/p\Z)^\times$. Since $[a]_p\in (\Z/p\Z)^\times$ has order $(p-1)$, we have $\ord([a]_p)\mid (p-1)$. If $\ord([a]_p)=d\mid (p-1)$, then we have 
\[a^d\equiv 1\pmod p\]
By Lemma 3, we have 
\[(a^d)^{p^{n-1}}\equiv 1^{p^{n-1}}\pmod{p^n}\]
So we have $(p-1)p^{n-1}\mid dp^{n-1}$. So we have $p-1\mid d$. Hence, $d=p-1$. So $[a]_p$ is a primitive root.
\\\qed\\
\textbf{Problem 4:} If $a^(p-1)=1\pmod p$, then we have 
\[(-a)^{p-1}=(-a)^{4t}=(-1)^{4t}a^{4t}=a^{4t}=1\pmod p\]
Suppose $\ord(-a)=d<4t$, then we have $d\mid 2t$. Hence, we have $1=(-a)^{2t}=a^{2t}\pmod p$. It contradicts since $a$ has order $4t=p-1$. Hence, $\ord(-a)=p-1$. So $-a$ is a primitive root mod $p$.\\
Conversely, if $-a$ is a primitive root, then we have $a^4t=(-a)^{4t}=1\pmod p$. Also, we have $a^{2t}=(-a)^{2t}\neq 1\pmod p$. So $\ord(a)=4t$. So $a$ is a primitive root mod $p$.\\
\textbf{Problem 5:} The proof is similar to last problem. If $a$ has order $p-1=4t+2$, we know $a^{2t+1}$ has order 2. Since $(\Z/p\Z)^\times$ is cyclic, there are only two elements are killed by 2. One is identity and the other one is $-1$. So we must have $a^{2t+1}=-1$ because $a^{2t+1}\neq 1$. Now, if we raise $-a$ to $2t+1$ power, then we have 
\[(-a)^{2t+1}=-a^{2t+1}=-(-1)=1\]
So $\ord(-a)\mid 2t+1$. If $\ord(-a)=d\mid 2t+1$, then we have $d$ is odd and $(-a)^d=-a^d=1$, which implies $a^d=-1$. But we already know that $a^{2t+1}=-1$ and $1\leq d\leq 2t+1$. So we must have $d=2t+1$ by the structure of cyclic group.\\
Conversely, if $(-a)$ has order $2t+1$, then we have $(-a)^{2t+1}=-a^{2t+1}=1$ implies $a^{2t+1}=-1$ and $a^{4t+2}=(-1)^2=1$.. Hence, we have $\ord(a)\nmid 2t+1$ but $\ord(a)\mid 4t+2$. So we must have $\ord(a)=4t+2=p-1$. So $a$ is a primitive root mod $p$.
\\\qed\\
\textbf{Problem 6:} Notice that $p=3$ is a Fermat prime, but 3 is not a primitive root mod $3$. Hence, we should assume $p\neq 3$. So we have $n>1$. So we have $p=4\cdot 2^{n-1}+1$. Now, by Wilson's theorem, we have 
\[-1=(p-1)!=\prod_{i=1}^{\frac{p-1}{2}}i\prod_{i=\frac{p-1}{2}+1}^{p-1}i=(-1)^{\frac{p-1}{2}}(\prod_{i=1}^{\frac{p-1}{2}}i)^2=(\prod_{i=1}^{\frac{p-1}{2}}i)^2\]
since $\frac{p-1}{2}+i=-i\pmod p$ for any $1\leq i\leq \frac{p-1}{2}$ and $\frac{p-1}{2}=2\cdot 2^{n-1}$ is even.\\
Now, we claim that $3$ is not a square. We suppose it is towards contradiction. Then, we have $3=a^2$ for some $a\in (\Z/p\Z)^\times $. Then we have $-3=(\prod_{i=1}^{\frac{p-1}{2}}i)^2a^2=b^2$, where $b=(\prod_{i=1}^{\frac{p-1}{2}}i)a$. If $b$ is even, then we let $b'=b+p$, we still have $b'^2=b^2=-3\pmod p$. So we can suppose $b$ is odd. Let $b=2m+1$ for some $m\in\Z$. Then we solve for $m$. 
\[(2m+1)^2\equiv -3\implies 4m^2+4m+4\equiv 0\pmod p\]
Since $p$ is odd prime, we have 
\[m^2+m+1\equiv 0\pmod p\]
Now multiply both side by $m-1$, we have 
\[m^3-1\equiv 0\pmod p\]
So we have $\ord(m)\mid 3$. But $m\neq 1$ since if $m=1$, then $-3\equiv(3)^2=9\pmod p$, which implies $p\mid 12$. But, we assume that $p\neq 3$ and $p>2$, so it is not possible. So we have $\ord(m)=3$. But this is weird since $\ord(m)\mid p$ implies $3\mid p$. It contradicts to the fact that $p$ is prime. So $3$ is not a square.\\
Now, suppose $g$ is a primitive root mod $p$. Then $3=g^{2k+1}$ for some $1\leq k\leq 2^{n-1}-1$. Suppose $\ord(3)=\alpha$. Then we have 
\[3^{\alpha}=(g^{2k+1})^\alpha=g^{\alpha(2k+1)}\equiv 1\pmod p\]
So we have $p-1\mid \alpha(2k+1)$. But since $p-1=2^n$, So we have $p-1\mid \alpha$ since $\gcd (2^n,2k+1)=1$. On the other hand, we have $\alpha=\ord(3)\mid (p-1)$, so we have $\alpha=p-1$. So $3$ is a primitive root. 
\\\qed\\
\textbf{Problem 12:}
Notice that when $p=2$, the result is trivial, we have nothing to prove. So we can assume $p>2$. So $p$ is odd. Now by the existence, we suppose the primitive root is $a$, the taking the product of all elements in the group is the same as taking the product of all power of a from $1$ to $p-1$.
\[(p-1)!=a^{\sum_{i=1}^pi}=a^{\frac{p(p-1)}{2}}\]
Since $p$ is odd, we have $2\nmid p$. So we have $p-1\nmid \frac{(p-1)p}{2}$ by multiply both side by $\frac{p-1}{2}$. So $a^{\frac{p(p-1)}{2}}\neq 1$. But we know 
\[(a^{\frac{p(p-1)}{2}})^2=a^{(p-1)p}=(a^{p-1})^p=1^p=1\]
So we know $a^{\frac{p(p-1)}{2}}$ has order 2. But there are exactly one element has order 2 in the cyclic group $(\Z/p\Z)^\times $, which is $-1$. Otherwise the group will have two distinct subgroups with order 2, which cannot happens in a cyclic group. So we have $a^{\frac{p(p-1)}{2}}=-1$. The proof completes.
\\\qed\\
\textbf{Problem 19:} Notice that if $g$ is a primitive root mod $p$, then we can write $a=g^k$ for some $1\leq k\leq p-1$. Then being solvable is the same as saying $3\mid k$.
If $p=7$, then we know the group of unity has order 6. So there is only one element is solvable, which is $a=g^3$, where $g$ is primitive root mod $7$.\\
If $p=11$, then the group of unity has order $10$. There are 3 elements making the equation solvable, $g^3,g^6,g^9$, where $g$ is primitive root mod $11$.\\
If $p=13$, then the group of unity has order 12. There are 3 elements making the equation solvable, $g^3,g^6,g^9$, where $g$ is primitive root mod $13$.\\
\qed\\
\textbf{Problem 21:} We can work in any finite cyclic group with order $p-1$ and let $g$ the a generator the group. Then we have $(g^{\frac{p-1}{d}})^d=g^{p-1}=1$. implies that $\ord(g^{\frac{p-1}{d}})\mid d$. If $\ord(g^{\frac{p-1}{d}})=d'< d$, then we have 
\[1\leq \frac{p-1}{d}d'<\frac{p-1}{d}d=p-1\]
So we have $(g^{\frac{p-1}{d}})^{d'}\neq 1$. It contradicts to the fact that $\ord(g^{\frac{p-1}{d}})=d'$. So $\ord(g^{\frac{p-1}{d}})=d$.\\
We shall suppose $a\neq 0\pmod p$; otherwise, $a$ can be any positive power of itself. If $a\neq 0$ is a $d$-th power, then there exists some element $n^d=a$. Suppose $n=g^k$, then 
we have $a=g^{kd}$.\\ Conversely, if $a= g^{kd }$, then we take $n=g^k$, we have $a=n^d$ is a dth power.\\\qed\\
\textbf{Problem 22:} Since $a^3\equiv 1\pmod p$, we have 
\[a^3-1=(a-1)(a^2+a+1)\equiv 0\pmod p\]
Since $a$ has order 3, $a-1\neq 0$. Since $\Z/p\Z$ is an integral domain, so $a^2+a+1=0$.
Then we have 
\[(1+a)^6=((1+a)^2)^3=(1+2a+a^2)^3\equiv a^3\equiv 1\pmod p\]
So we have $\ord(1+a)\mid 6$. If $\ord(1+a)=1$, then we have $1+a\equiv 1\pmod p$ implies $a\equiv 0$. It contradicts. If $\ord(1+a)=2$, then we have $(1+a)^2=1+2a+a^2\equiv a\equiv 1\pmod p$. It contradicts. Also, we know $\ord(1+a)\neq 3$, since we have 
$$(1+a)^3=(1+a)^2(1+a)=a(1+a)=a+a^2\equiv -1\pmod p$$ 
So we have $\ord(1+a)=6$.
\\\qed\\
\textbf{Problem 24:} We first construct a bijection between solutions in $(\Z/p\Z)^\times\times (\Z/p\Z)^\times$.\\
Since $m'=\gcd(m,p-1)$, then we have $m'd=m$ for some $d\in \Z$. Notice that $d=\frac{m}{\gcd(m,p-1)}$, we have $\gcd(d,p-1)=1$. Then we have $d\in (\Z/(p-1)\Z)^\times$. So we have $m'=d^{-1}m\pmod {p-1}$. Similarly, we can suppose $n'e=n$ and $n'=e^{-1}n\pmod{p-1}$. Now, we are ready to give a bijection between the solution sets. 
\[ax^m+by^n=c\tag{1}\] 
\[ax^{m'}+by^{n'}=c\tag{2}\] 
Since $p$ is a prime, there exists a primitive root of unity $g$ mod $p$. Suppose $(g^\alpha,g^\beta)$ is a solution for $(1)$. Then we have $(g^{d\alpha},g^{e\beta})$ is a solution for (2) since 
\[a(g^{d\alpha})^{m'}+b(g^{e\beta })^{n'}=a(g^{\alpha})^{m'd}+b(g^{\beta })^{n'e}=a(g^{\alpha})^{m}+b(g^{\beta })^{n}=c\]
We denote this function as 
\[f:Sol(1)\cap((\Z/p\Z)^\times\times (\Z/p\Z)^\times)\to Sol(2)\cap((\Z/p\Z)^\times\times (\Z/p\Z)^\times)\]
\[(g^\alpha,g^\beta)\mapsto(g^{d\alpha},g^{e\beta})\]
On the other hand, if $(g^{\alpha},g^{\beta})$ is a solution for $(2)$, then we can consider $(g^{d^{-1}\alpha},g^{e^{-1}\beta})$, where $d^{-1},e^{-1}$ makes senses because the power of $g$ is in $\Z/(p-1)\Z$ since $(\Z/p\Z)^\times$ is a cyclic group of order $p-1$.
Then we have 
\[a(g^{d^{-1}\alpha})^{m}+b(g^{e^{-1}\beta })^{n}=a(g^{\alpha})^{md^{-1}}+b(g^{\beta })^{ne^{-1}}=a(g^{\alpha})^{m'}+b(g^{\beta })^{n'}=c\]
is a solution for $(1)$.
We denote this function as 
\[h:Sol(2)\cap((\Z/p\Z)^\times\times (\Z/p\Z)^\times)\to Sol(1)\cap((\Z/p\Z)^\times\times (\Z/p\Z)^\times)\]
\[(g^\alpha,g^\beta)\mapsto(g^{d^{-1}\alpha},g^{e^{-1}\beta})\]
Now, we have 
\[f\circ h(g^{\alpha},g^{\beta})=f(g^{d^{-1}\alpha},g^{e^{-1}\beta})=(g^{dd^{-1}\alpha},g^{ee^{-1}\beta})=(g^\alpha,g^\beta)\]
\[h\circ f(g^{\alpha},g^{\beta})=h(g^{d\alpha},g^{e\beta})=(g^{d^{-1}d\alpha},g^{e^{-1}e\beta})=(g^\alpha,g^\beta)\]
So $f$ is a bijection.\\
Now, we consider the solutions such that $x=0$, and $y\in(\Z/p\Z)^\times$, then we also have a bijection 
$$f':Sol(1)\cap (0\times (\Z/p\Z)^\times)\to Sol(2)\cap (0\times (\Z/p\Z)^\times)$$
\[(0,g^\beta)\mapsto (0,g^{e\beta})\]
It is well-defined because 
\[b(g^{e\beta })^{n'}=b(g^{\beta })^{n'e}=b(g^{\beta })^{n}=c\]
Also, we have its inverse 
\[h':Sol(2)\cap(0\times (\Z/p\Z)^\times)\to Sol(1)\cap(0\times (\Z/p\Z)^\times)\]
\[(0,g^\beta)\mapsto(0,g^{e^{-1}\beta})\]
It is well-defined since 
\[b(g^{e^{-1}\beta })^{n}=b(g^{\beta })^{ne^{-1}}=b(g^{\beta })^{n'}=c\]
And we know $f'\circ h'=id$ and $h'\circ f'=id$ since $f'$ is just the second coordinate of $f$ and same for $h'$.
So we have a bijection here.\\
Similarly if we consider the solutions such that $x\in(\Z/p\Z)^\times$ and $y=0$. We have a bijection arise in the same way.\\
Now, if $x=0,y=0$ is a solution for $(1)$, then it is definitely a solution for (2) since this implies $c=0\pmod p$.
If it is not a solution, then we just ignore it.\\
Now, we can put these three bijective function together since they are functions whose domain are pairwise disjoint and whose image are pairwise disjoint. We will still get a bijective function. So $(1),(2)$ has same number of solutions.

\end{document}