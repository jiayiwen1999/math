\documentclass[12pt]{amsart}
\usepackage{amsmath,epsfig,fancyhdr,amssymb,subfigure,setspace,fullpage,mathrsfs,upgreek,tikz-cd}
\usepackage[utf8]{inputenc}

\newcommand{\R}{\mathbb{R}}
\newcommand{\Q}{\mathbb{Q}}
\newcommand{\C}{\mathbb{C}}
\newcommand{\Z}{\mathbb{Z}}
\newcommand{\N}{\mathbb{N}}
\newcommand{\G}{\mathcal{N}}
\newcommand{\A}{\mathcal{A}}
\newcommand{\sB}{\mathscr{B}}
\newcommand{\sC}{\mathscr{C}}
\newcommand{\sd}{{\Sigma\Delta}}
\newcommand{\Orbit}{\mathcal{O}}
\newcommand{\normal}{\triangleleft}
\newcommand{\ord}{\mathrm{ord}}

\begin{document}
\title{Homework 5 - 104A}
\maketitle
\begin{center}
    Jiayi Wen\\
    A15157596
\end{center}
Notation: We will use $[a]_n$ to denote the congruence class of $a$ in $\Z/n\Z$. The subscript $n$ is omitted if the ring $\Z/n\Z$ is clear by the context.
\begin{center}
    \textbf{Chapter 3}
\end{center}
\textbf{Problem 14:} By the Chinese Remainder Theorem, we have 
\[\Z/pq\Z\cong \Z/p\Z\times \Z/q\Z\]
So in order to prove $n^{q-1}\equiv 1 \pmod{pq}$, we can prove 
\[([1]_p,[1]_q)=([n^{q-1}]_p,[n^{q-1}]_q)\]
Since $\gcd(n,pq)=1$, we have $\gcd(n,p)=1$ and $\gcd(n,q)=1$. Hence, we have 
$[n^{q-1}]_q=1$ by Fermat's little theorem. All we need to show is $[n^{q-1}]_p=[1]_p$. Since $p-1\mid q-1$, and we know the order of $(\Z/p\Z)^\times$ is $p-1$, so we have $([n]_p)^{p-1}=[1]_p$ for all $n\in\Z$. Hence, we have 
\[[n^{q-1}]_p=([n]_p)^{q-1}=[1]_p\]
So we have 
\[([n^{q-1}]_p,[n^{q-1}]_q)=([1]_p,[1]_q)\]
Then by the Chinese Remainder Theorem, we have $[n^{q-1}]_{pq}=[1]_{pq}$.
\\\qed\\
\textbf{Problem 15:} It is obvious that the argument fails in $p=2$ since the summand will be $1$, whose numerator is also $1$ and not divisible by 2. If $p>2$. Notice that the numerator is 
\[\sum_{i=1}^{p-1}\prod_{j\neq i}j\]
Then by Wilson's Theorem, we have 
\[\prod_{j\neq i}j\equiv \frac{(p-1)!}{i}\equiv \frac{-1}{i}\pmod p\]
So the numerator is 
\[\sum_{i=1}^{p-1}\prod_{j\neq i}j\equiv -\sum_{i=1}^{p-1}\frac{1}{i}\pmod p\]
Since the inverse of every element is unique in a group, so we have a bijective function 
\[f:(\Z/p\Z)^\times \to (\Z/p\Z)^\times \]
\[i\mapsto i^{-1}\]
It is injective because $f(i)=f(j)$ if and only if $i^{-1}=j^{-1}$  if and only if $j=i$.\\
It is surjective since for any $i\in (\Z/p\Z)^\times$, $f(i^{-1})=(i^{-1})^{-1}=i$.\\
Hence, summing over the inverse of all elements in the multiplicative group is the same as summing over all the elements in the multiplicative group. So we have 
\[-\sum_{i=1}^{p-1}\frac{1}{i}\equiv -\sum_{i=1}^{p-1}i=-\frac{(1+p-1)(p-1)}{2}=-\frac{p(p-1)}{2}\pmod p\]
Since $p>2$ and $p$ is prime, we know $p-1$ is even. Hence $\frac{p-1}{2}$ is an integer. So we have 
\[-p\frac{p-1}{2}\equiv 0\pmod p\]
So the numerator is divisible by $p$.
\\\qed\\
\textbf{Problem 16:} By the proof, we have 
\[13\cdot 7+(-2)\cdot 45=1\]
\[4\cdot 9+(-1)\cdot 35=1\]
\[(-25)\cdot 5+2\cdot 63=1\]
So we have $x_0=1\cdot (-90)+4\cdot (-35)+3\cdot (126)=-90-140+378=148$ is a solution.
\\\qed\\
\textbf{Problem 18:} The idea is to use the Chinese Remainder Theorem. If $f(x_1)\equiv 0\pmod{p_1^{a_1}}$, $f(x_2)\equiv 0\pmod{p_2^{a_2}}$, $\cdots, \ f(x_t)\equiv 0\pmod{p_t^{a_t}}$, then we can find a solution $x$ for the system $x\equiv x_i\pmod{p_i^{a_i}}$, where $i=1,\cdots, t$ by the Chinese Remainder Theorem. Then we have 
\[f(x)\equiv f(x_i)\equiv 0\pmod{p_i^{a_i}}, \ \forall\  i=1,\cdots,t \]
So we have $p_i^{a_i}\mid f(x)$ for all $i$. Hence, we have 
\[n=p_1^{a_1}\cdots p_t^{a_t}\mid f(x)\]
So $x$ is a solution for $f(x)\equiv 0\pmod n$. Therefore, there are at least $N_1\cdots N_t$ solution for $f(x)\equiv0\pmod n$. In other words, we have $N\geq N_1\cdots N_t$.\\
On the other hand, if $x$ is a solution for $f(x)\equiv 0\pmod n$, then we have $n\mid f(x)$. Since $p_i^{a_i}\mid n$, we have $p_i^{a_i}\mid f(x)$. So $x$ is also a solution for $f(x)\equiv 0\pmod{p_i^{a_i}}$ for all $i$. So $N\leq N_1\cdots N_t$.\\
Hence, we have $N=N_1\cdots N_t$.
\\\qed\\
\textbf{Problem 20:} The problem is the same as asking how many element in $\Z/2^b\Z$ has multiplicative order 1 or 2. One observation is that the solution must be nonzero element since any power of 0 is 0. If $b=1$, then there are exactly 2 elements in $\Z/2\Z$ and we know $0^2\equiv 0\pmod 2$, so the only solution is $x=1$.\\
If $b=2$, then we have $2^2\equiv 0\pmod 4$ and $3^2\equiv 1\pmod 4$. So we have 2 solutions and they are $1,3$.\\
If $b>2$. Since $x^2=1$, then $x$ is a unit. so we have $x$ is odd. Suppose $x=2m+1$ is a solution, then we have 
\[x^2-1=(x+1)(x-1)=(2m+2)2m=4m(m+1)\equiv 0\pmod{2^b}\]
So we have $2^{b-2}\mid m(m+1)$. Since $m,m+1$ has different parity, we can two cases.\\
If $m$ is even, then we have $2^{b-2}\mid m$. Since $2m+1<2^b$, we have 
$2m<2^b-1$. Hence, we have $2m\leq 2^b-2$. So we have $m\leq 2^{b-1}-1$. Since $2^{b-2}\cdot 2=2^{b-1}>2^{b-1}-1\geq m$. So we only have two choices, $m=0$ or $m=2^{b-2}$. So $x=1$ or $x=2^{b-1}+1$.\\
If $m$ is odd, then we have $2^{b-2}\mid m+1$. By previous argument, we know $m\leq 2^{b-1}-1$. So we also have two choices. If $m+1=2^{b-2}$, then $m=2^{b-2}-1$. If $m+1=2^{b-1}$, then $m=2^{b-1}-1$. Hence, $x=2^{b-1}-1$ or $x=2^{b}-1$.\\
So we have four solution in total for $b>2$.
\\\qed\\
\begin{center}
    \textbf{Chapter 4}
\end{center}
\textbf{Problem 13:} Notice that an element of $G$ is a generator if and only if it has order $n$. Suppose $g^k\in G$, where $1\leq k\leq n$. Then the order of $g^k$ is the minimal positive integer $m$ such that $(g^k)^m=g^{km}=1$. But this happens if and only if $n\mid km$. If $\gcd(k,n)=1$, then we have $n\mid m$, hence $m=n$. On the other hand, if $m=n$, and $m$ is minimal, we want to show $\gcd(k,n)=1$. Suppose $\gcd(k,n)=d>1$ towards contradiction. Then we have $\frac{n}{d}\in\Z$ and $(g^k)^\frac{n}{d}=1$ because $n\mid \frac{kn}{d}=\frac{k}{d}\cdot n$ and $\frac{n}{d}<n$. This contradicts to the fact that $m=n$ is minimal. So we must have $\gcd(k,n)=1$.
\\\qed\\
\textbf{Problem 14:} Claim: If $\ord(a)=m$ and $\ord(b)=n$, then $\ord(ab)=lcm(m,n)$.\\
Since $A$ is abelian, we have $(ab)^k=a^kb^k$. If $(ab)^k=1$, then we have $a^k=1$ and $b^k=1$. Hence, we have $m\mid k$ and $n\mid k$. Since $\ord(ab)=\min\{k\in\Z\mid (ab)^k=1\}$, we can rewrite it as 
\[\ord(ab)=\min\{k\in\Z: m\mid k,n\mid k\}=lcm(m,n)\] 
But this is the same as the definition of least common multiple of $m,n$. If $\gcd(m,n)=1$, then we have 
\[\ord(ab)=lcm(m,n)=\frac{mn}{\gcd(m,n)}=mn\]
\qed\\
\textbf{Problem 15:} Suppose $|G|=n$, then we denote $\Psi(d)=card\{g\in G\mid \ord(g)=d \}$. Also, we denote $H:=\{g^d=1\mid g\in G\}$. We claim that $(H,\cdot)$ is a subgroup of $(G,\cdot)$. First, $H\neq \emptyset$ since $1\in H$. If $g,h\in H$, then we have $(gh^{-1})^d=g^dh^{-d}=1\cdot 1=1$. So we have $gh^{-1}\in H$. So $H\leq G$. Now, we want to prove $\psi(d)\leq \phi(d)$, where $\phi$ is the Euler totient function.\\
If $\psi(d)=0$, then it is trivial. If $\psi(d)\neq 0$, then there exists some $g\in G$ such that $\ord(g)=d$. It is obvious that $g\in H$. Hence, we have $\langle g\rangle \leq H$. So we have $\mid H\mid \geq \mid \langle g\rangle \mid =n$. On the other hand, the subgroup $H$ is the solution set of $x^d-1\in k[x]$. So we have $\mid H\mid \leq n$. So we have $H=\langle g\rangle$. So any other element with order $d$ in $G$ is a generator of $\langle g\rangle$. By problem 13, we know the generator of $\langle g\rangle$ has order $k$, where $\gcd(k,d)=1$. Hence, we have 
\[\psi(d)=\phi(d),\ \forall d\mid n\]
Hence, we have 
\[n=|G|=\sum_{d\mid n}\psi(d)\leq \sum_{d\mid n}\phi(d)=n\]
Hence, we must have $\psi(d)=\phi(d)$ for all $d\mid n$. In particular, we have $\psi(n)=\phi(n)\geq 1$. So there exists some element of $g'\in G$ such that $\ord(g')=n$. So $G$ is a cyclic group.
\\\qed\\
\textbf{Problem 20:}
Let's first show this is a subgroup of $(\Z/p\Z)^\times$. Denote the set as 
\[G:=\{([m]_p)^d\mid [m]_p\in (\Z/p\Z)^\times \}\]
Then for $([m_1]_p)^d,([m_2]_p)^d\in G$, we have 
\[([m_1]_p)^d\cdot ([m_2]_p)^{-d}=([m_1m_2^{-1}]_p)^d\]
Since $(\Z/pZ)^\times$ is a group, so $[m_1m_2^{-1}]_p\in (\Z/p\Z)^\times $. So we have $$([m_1]_p)^d\cdot ([m_2]_p)^{-d}=([m_1m_2^{-1}]_p)^d\in G$$
Hence, $G$ is a subgroup of $(\Z/p\Z)^\times $. Then we have a well-defined surjective group homomorphism.
\[f: (\Z/p\Z)^\times \to G\]
\[[m]_p\mapsto [m^d]_p\]
It is well-defined since finite powers is well-defined in the ring $\Z/p\Z$; hence on its multiplicative group. It is surjective by the definition of $G$. Notice that $\ker f=\{[m]_p\in (\Z/p\Z)^\times \mid [m^d]_p=[1]_p\}$. So we have $|\ker f|=\sum_{c\mid d}\psi(c)$, where $\psi(d)$ is the number of element in $(\Z/p\Z)^\times $ with order $d$. By previous problem, we have showed that $\psi(d)=\phi(d)$ for all $d\mid p-1$. Hence, we have 
\[|\ker f|=\sum_{c\mid d}\psi(c)=\sum_{c\mid d}\phi(c)=d\]
where $\phi$ is the Euler totient function. Now, by the first isomorphism theorem, we have 
\[|G|=\frac{|(\Z/p\Z)^\times |}{|\ker f|}=\frac{p-1}{d}\] 
If $p=11$ and $d=5$, then we have 
\[|G|=\frac{11-1}{5}=2\]
Notice that $2^5=32\equiv -1\pmod{11}$, so we have 
\[G=\langle [2^5]_{11}\rangle\]
If $p=17$ and $d=4$, then we have 
\[|G|=\frac{17-1}{4}=4\]
Also, we have $3^4=81\equiv 13\pmod {17}$. And notice that 
\[13^2\equiv (-4)^2\equiv 16\pmod {17},\ 13^3\equiv (-4)^3\equiv (-1)(-4)\equiv 4\pmod{17}\]
And 
\[13^4\equiv (-4)^4\equiv (-1)^2\equiv 1\pmod {17}\]
So we have $G=\langle[3^4]_{17}\rangle$.\\
If $p=19$, $d=6$, then we have 
\[|G|=\frac{19-1}{6}=3\]
Notice that $2^6=64\equiv 7\pmod{19}$ and we have 
\[7^2\equiv 11\pmod{19},\ 7^3\equiv 11\cdot 7=77\equiv 1\pmod{19}\]
So we have 
\[G=\langle[2^6]_p\rangle\]
\qed\\
\textbf{Problem 23:} If $x$ is a solution for $x^2\equiv -1\pmod p$, then we have $x^4\equiv 1\pmod p$. In other words, we know $\ord(x)\mid 4$. Since $x\neq 1$, and $x^2\neq 1$, we know $\ord(x)=4$. So it is equivalent to prove $(\Z/p\Z)^\times$ has an element of order $4$ if and only if $p\equiv 1\pmod4$.\\
If $(\Z/p\Z)^\times$ has an element of order $4$, then we know $4\mid p-1$. Hence, $p-1=4k$ for some $k\in\Z$. Hence, we have $p\equiv1\pmod 4$.\\
If $p\equiv 1\pmod 4$, then there are $4k$ element in $(\Z/p\Z)^\times$, where $k$ is some positive integer.
Then by Wilson's theorem, we have 
\[(p-1)!\equiv -1\pmod p\]
On the other hand, we have 
\[(p-1)!=\prod_{i=1}^{p-1}i=\prod_{i=1}^{4k}i\equiv(\prod_{i=1}^{2k}i)(\prod_{i=1}^{2k}(-i))=(-1)^{2k}((2k)!)^2=((2k)!)^2\equiv -1\pmod p\]
where we take $2k+1\equiv -2k$, $2k+2\equiv -2k+1$, $\cdots$, $4k\equiv -1 \pmod p$.\\
We use similar trick to prove the second part. A solution of $x^4\equiv -1$ has order $8$ since $x,x^2,x^4\neq 1\pmod p$ and $x^8\equiv (-1)^2\equiv 1\pmod p$. So we want to show $(\Z/p\Z)^\times$ has an element of order $8$ if and only if $p\equiv 1\pmod 8$.\\
If we have a solution $x$, then $8\mid p-1$. So we have $p-1=8k$ for some $k\in\Z_{>0}$. So we have $p\equiv 1\pmod 8$.\\
On the other hand, if $p\equiv 1\pmod 8$, then we suppose $p-1=8k$. Since $(\Z/p\Z)^\times$ is a cyclic group of order $8k$, then suppose $g$ is a generator of it. Then $g^k$ is an element of order $8$. Hence, we have $x=g^k$ is a solution of $x^4\equiv -1 \pmod p$.
\\\qed\\ 
\end{document}