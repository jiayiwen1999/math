
\documentclass[12pt]{amsart}
\usepackage{amsmath,epsfig,fancyhdr,amssymb,subfigure,setspace,fullpage,mathrsfs,upgreek,float, tabularx, tikz}
\usepackage[utf8]{inputenc}

\newcommand{\R}{\mathbb{R}}
\newcommand{\Q}{\mathbb{Q}}
\newcommand{\C}{\mathbb{C}}
\newcommand{\Z}{\mathbb{Z}}
\newcommand{\N}{\mathbb{N}}
\newcommand{\G}{\mathcal{N}}
\newcommand{\F}{\mathbb{F}}
\newcommand{\A}{\mathcal{A}}
\newcommand{\sB}{\mathscr{B}}
\newcommand{\sC}{\mathscr{C}}
\newcommand{\sd}{{\Sigma\Delta}}
\newcommand{\Orbit}{\mathcal{O}}
\newcommand{\normal}{\triangleleft}

\begin{document}
\title{Homework 2 - 202B}
\maketitle
\begin{center}
    Jiayi Wen\\
    A15157596
\end{center}
\textbf{Problem 1:} Consider the permutation representation of $S_n$. Let $V=\C^n$ be the asscoiated $S_n-$module. Let $\chi:S_n\to \C^\times $ be the character of this $S_n$-module. Then we have $\chi(\pi)=\mathrm{fix}(\pi)$. Then we have
\[\langle \chi,\chi\rangle=\frac{1}{|S_n|}\sum_{\pi\in S_n}\chi(\pi)\overline{\chi(\pi)}=\frac{1}{n!}\sum_{\pi\in S_n}\mathrm{fix}(\pi)^2\]
Now, we just need to show $\langle \chi,\chi\rangle=2$. We claim that $V\cong V_1\oplus V_2$, where
\[V_1=\{(x_1,x_2,\cdots,x_n)\in\C^n\mid \sum_{i=1}^nx_i=0\}\]
\[V_2=\{c(1,1,\cdots,1)\in\C^n\mid c\in \C\}\]
From last homework, we know $V_1$ is irreducible. Since $\dim V_2=1$, $V_2$ must be irreducible because the only vector subspaces of $V_2$ are $0$ and itself. Also, they have trivial intersection since if $(c,c,\dots, c)\in V_1$, then we have $\sum_{i=1}^n=cn=0$ implies $c=0$. Now, we want to show $V=V_1+V_2$. For any $(x_1,\cdots,x_n)\in V$, we denote
\[\lambda=\sum_{i=1}^n x_i\]
Then we can write
\[(x_1,\cdots, x_n)=(x_1-\frac{\lambda}{n},\cdots, x_n-\frac{\lambda}{n})+(\frac{\lambda}{n},\cdots,\frac{\lambda}{n})\in V_1+V_2\]
So we have $V\subseteq V_1+V_2$. The other direction is given by vector space axiom (closed under addition). So we decompose $V$ into irreducibles as the direct sum of $V_1$ and $V_2$.
So we have $\langle\chi,\chi\rangle=1^2+1^2=2$. The proof completes.
\\\qed\\
\textbf{Problem 2:} Since $Z(Q_8)=\{\pm1\}$, then center of $Q_8$, $1$ and $-1$ are two conjugacy classes of $Q_8$. Also, we have
\[j^{-1}ij=-jij=-(-k)j=kj=-i\]
\[k^{-1}ik=-kik=-jk=-i\]
So $\{\pm i\}$ is a conjugacy class. Similarly, we have two more conjugacy classes. They are $\{\pm j\}$ and $\{\pm k\}$. So $Q_8$ has 5 irreducible character. Let's call them $\chi_1,\chi_2,\chi_3,\chi_4,\chi_5$, where $\chi_1$ is the character of trivial representation. If we denote $d_i$ as the dimension of $\chi_i$, then we have $\sum_{i=1}^5 d_i^2=\mid Q_8\mid =8$. The only possibility is $1,1,1,1,2$, which gives us the first column (character of 1).  Then the table looks like
\begin{table}[H]
    \centering
    \begin{tabular}{l|cccccl}
        $Q_8$    & 1 & -1 & $\pm i$ & $\pm j$ & $\pm k$ \\ \hline
        $\chi_1$ & 1 & 1  & 1       & 1       & 1       \\
        $\chi_2$ & 1                                    \\
        $\chi_3$ & 1                                    \\
        $\chi_4$ & 1                                    \\
        $\chi_5$ & 2
    \end{tabular}
\end{table}
\noindent For the second column, we look at degree 1 character first. Since $\chi:Q_8\to \C^\times$ is a group homomorphism, we have $(\chi(-1))^2=\chi((-1)^2)=\chi(1)=1$. Hence, $\chi(-1)=\pm 1$. But since
\[(ij)(i^{-1}j^{-1})=kk=-1\]
and $C^\times$ is abelian, we have
\[\chi(-1)=\chi(iji^{-1}j^{-1})=\chi(i)\chi(j)\chi(-i)\chi(-j)=\chi(i)\chi(-i)\chi(j)\chi(-j)=1\]
Then by character orthogonality, we have $\chi_5(-1)=-2$.\\
Similarly, we can conclude the 1-dim character of $i,j,k$ with same trick. Since $i^2=j^2=k^2=-1$, we have
\[\chi(i)^2=\chi(j)^2=\chi(k)^2=\chi(-1)=1\]
So we have two option 1, -1 for each. Also, if we determine the character of two of them, the character of the third one is determined. So we have

\begin{table}[H]
    \centering
    \begin{tabular}{l|cccccl}
        $Q_8$    & 1 & -1 & $\pm i$ & $\pm j$ & $\pm k$ \\ \hline
        $\chi_1$ & 1 & 1  & 1       & 1       & 1       \\
        $\chi_2$ & 1 & 1  & 1       & -1      & -1      \\
        $\chi_3$ & 1 & 1  & -1      & 1       & -1      \\
        $\chi_4$ & 1 & 1  & -1      & -1      & 1       \\
        $\chi_5$ & 2 & -1 & 0       & 0       & 0
    \end{tabular}
\end{table}
where the last line is by character orthogonality.
\\\qed\\
\textbf{Problem 3:} Let's find the conjugacy classes first. We claim that there are $\frac{n}{2}+3$ conjugacy classes. Since $D_n$ is generated by $r,s$, so the conjugacy classes is determined by conjugation of $r,s$ and the character is determined the character of $r,s$. By the presentation, we have $srs=r^{-1}$. Hence, we have
\[sr^ks=(srs)^k=r^{-k}=r^{n-k}\]
If $k=0$, then $r^0=e$ commutes with all elements in $D_n$. If $k=\frac{n}{2}$, then it commutes with $s$ and all power of $r$. Hence, $r^{\frac{n}{2}}\in Z(D_n)$ as well.
If $k\neq \frac{n}{2},0$, then we have $r^k,r^{n-k}$ are in the same conjugacy classes for all $k=1,2,\cdots \frac{n}{2}-1$. Since $r^k$ commutes with any power of $r$, we have $r^k,r^{n-k}$ is a conjugacy class. So far, we have $\frac{n}{2}+1$ conjugacy classes.\\
Notice that $s(sr^i)s=r^{i}s=sr^{-i}$ and $r(sr^i)r^{-1}=(rs)r^{i-1}=sr^{-1}r^{i-1}=sr^{i-2}$. So the conjugacy classes of elements in the form of $sr^i$ are determined by the pairty of $i$. So there are two conjugacy classes, which are $\{s,sr^2,\cdots, sr^n\}$ and $\{sr,sr^3,\cdots, sr^{n-1}\}$.\\
Let's consider irreducible 1-dim character first. Denote it by $\chi$. Then we have
\[\chi(s)^2=\chi(s^2)=\chi(e)=1\]
So $\chi(s)=\pm 1$. Also, we have
\[\chi(r^{-1})=\chi(srs)=\chi(s)\chi(r)\chi(s)=\chi(s)^2\chi(r)=\chi(r)\]
So $\chi(r)\in\R$ since $\chi(r^{-1})=\overline{\chi(r)}=\chi(r)$. Since $\chi(r)^n=\chi(r^n)=\chi(e)=1$, we have $\chi(r)=\pm 1$ as well. So we have 4 different choice of 1-dim irreducible character. So these part of character table looks like
\begin{table}[H]
    \centering
    \begin{tabular}{l|ccccccl}
        $D_n$    & $e$ & $s$ & $r$ & $r^2$ & $\cdots$ & $r^\frac{n}{2}$    & $sr$ \\ \hline
        $\chi_1$ & 1   & 1   & 1   & 1     & $\cdots$ & 1                  & 1    \\
        $\chi_2$ & 1   & 1   & -1  & $1$   & $\cdots$ & $(-1)^\frac{n}{2}$ & -1   \\
        $\chi_3$ & 1   & -1  & 1   & 1     & $\cdots$ & 1                  & -1   \\
        $\chi_4$ & 1   & -1  & -1  & 1     & $\cdots$ & $(-1)^\frac{n}{2}$ & 1
    \end{tabular}
\end{table}
\noindent For the rest, we claim there are $\frac{n}{2}-1$ pairwise nonisomorphic irreducible 2-dim representations. The idea comes from the format of defining representation of $D_n$.
Let $\zeta=e^{\frac{2\pi }{n}i}$. Then for $1\leq j \leq \frac{n}{2}-1$, we can define
$$\rho_j:D_n\to GL_2(\C)$$
\[r\mapsto\begin{pmatrix}
        \zeta^j & 0          \\
        0       & \zeta^{-j}
    \end{pmatrix},\ s\mapsto \begin{pmatrix}
        0 & 1 \\
        1 & 0
    \end{pmatrix}\]
They are irreducible because if $1\leq j\leq \frac{n}{2}-1$, then
\[\zeta^{-j}=e^{\frac{-2j\pi}{n}i}=e^{\frac{(2n-2j)\pi}{n}i}\neq e^{\frac{2j\pi}{n}i}=\zeta^j\]
So no 1-dimensional vector subspace is fixed by $r$. Hence, it is irreducible. Notice that being isomorphic is the same as the corresponding matrices are similar. Since similar matrices has same Jordan canonical form, and $\rho_j(r)$ is also in Jordan canonical form, we can conclude that $\rho_j$ and $\rho_j'$ are nonisomorphic for all $j\neq j'$. So we have $\frac{n}{2}-1$ 2-dim irreducible characters. Together with $4$ irreducible 1-dim characters, we have found all irreducible character of $D_n$. Let's denote $\chi_{\rho_j}$ as the character of $\rho_j$. Then the character table looks like
\begin{table}[H]
    \centering
    \begin{tabular}{l|ccccccl}
        $D_n$                         & $e$ & $s$ & $r$                                            & $r^2$                      & $\cdots$ & $r^\frac{n}{2}$                                                        & $sr$ \\ \hline
        $\chi_1$                      & 1   & 1   & 1                                              & 1                          & $\cdots$ & 1                                                                      & 1    \\
        $\chi_2$                      & 1   & 1   & -1                                             & $1$                        & $\cdots$ & $(-1)^\frac{n}{2}$                                                     & -1   \\
        $\chi_3$                      & 1   & -1  & 1                                              & 1                          & $\cdots$ & 1                                                                      & -1   \\
        $\chi_4$                      & 1   & -1  & -1                                             & 1                          & $\cdots$ & $(-1)^\frac{n}{2}$                                                     & 1    \\
        $\chi_{\rho_1} $              & 2   & 0   & $\zeta+\zeta^{-1} $                            & $\zeta^{2}+\zeta^{-2}$     & $\cdots$ & $\zeta^\frac{n}{2}+\zeta^\frac{-n}{2}$                                 & 0    \\
        $\vdots$                                                                                                                                                                                                           \\
        $\chi_{\rho_{\frac{n}{2}-1}}$ & 2   & 0   & $\zeta^{\frac{n}{2}-1}+\zeta^{-\frac{n}{2}+1}$ & $\zeta^{n-2}+\zeta^{-n+2}$ & $\cdots$ & $\zeta^{\frac{n^2}{4}-\frac{n}{2}}+\zeta^{-\frac{n^2}{4}+\frac{n}{2}}$ & 0
    \end{tabular}
\end{table}
\qed\\
\textbf{Problem 4:} Since $X$ is a finite set, we assume $X=\{x_1,\cdots ,x_n\}$. The action by $g$ is asscoiate with an $n\times n$ matrix, let's call it $A_g$. If we pick the basis as $\{x_1,\cdots ,x_n\}$, then the matrix $A_g$ has only 0,1 on all of its entries. Having a 1 on $ij$th entry means that $g\cdot x_j=x_i$. So having 1's on the diagonal simply means $g$ fix that element. So we have
\[\chi(g)=\#\text{ number of fixed points by }g=|\{x\in X\mid g\cdot x =x \}\]
\qed\\
\textbf{Problem 5:} By problem 3, we have the character table as the following 

\begin{table}[H]
    \centering
    \begin{tabular}{l|cccccl}
        $D_6$                         & $e$ & $s$ & $r$                                            & $r^2$                      &  $r^3$                                                        & $sr$ \\ \hline
        $\chi_1$                      & 1   & 1   & 1                                              & 1                          & 1                                                                      & 1    \\
        $\chi_2$                      & 1   & 1   & -1                                             & $1$                         & $-1$                                                     & -1   \\
        $\chi_3$                      & 1   & -1  & 1                                              & 1                          & 1                                                                      & -1   \\
        $\chi_4$                      & 1   & -1  & -1                                             & 1                          & $-1$                                                     & 1    \\
        $\chi_{\rho_1} $              & 2   & 0   & $\zeta+\zeta^{-1} $                            & $\zeta^{2}+\zeta^{-2}$     & $\zeta^3+\zeta^3$                                 & 0    \\
        $\chi_{\rho_{2}}$ & 2   & 0   & $\zeta^2+\zeta^2$ & $\zeta^{4}+\zeta^{-4}$  & $\zeta^6+\zeta^{-6}$ & 0
    \end{tabular}
\end{table}
\noindent Let $V=\C^{15}$ be the corresponding $D_6$-module and we denote the character of $V$ by $\chi$. By problem 4, the character is the same as number of fixed sides/diagonals under the action. Notice that the sides and diagonals of the hexagon can be viewed as the edges of a complete graph $K_6$. And the edges can be encoded with the vertices. We number the vertices by $[6]$, starting from top left corner and going clockwise for convenience. If an edge is fixed under $g\in D_6$, then the corresponding pair of vertices either form a single orbit or splits as two singleton orbits under $g$-action. So we have $\chi(e)=15$ since $e$ fixed all vertices; hence all edges. We have $\chi(r)=\chi(r^2)=0$ because $r$ acts transtively on the vertices and the orbit of $r^2$ has size 3. For $r^3$, we have three $2-$orbits, and they are $\{1,4\}$, $\{2,5\}$, $\{3,6\}$. So $\chi(r^3)=3$. Similarly, we have three $2-$orbits for $s$, and they are $\{1,2\}$, $\{4,5\}$, $\{3,6\}$. And two $2-$orbits and two singletons for $sr$, and they are $\{1\},\ \{4\},\ \{2,6\}$ and $\{3,5\}$. So we have 
\[\chi(r^3)=\chi(s)=\chi(sr)=3\]
Then we have 
\[\langle \chi,\chi_1\rangle=\frac{1}{12}(15+3\cdot 3+3+3\cdot 3)=\frac{1}{12}\cdot 36=3 \]
\[\langle \chi,\chi_2\rangle=\frac{1}{12}(15+3\cdot 3-3=3\cdot 3)=\frac{1}{12}\cdot 12=1  \]
\[\langle \chi,\chi_3\rangle=\frac{1}{12}(15-3\cdot 3+3-3\cdot 3)=\frac{1}{12}\cdot 0=0 \]
\[\langle \chi,\chi_4\rangle=\frac{1}{12}(15-3\cdot 3-3+3\cdot 3)=\frac{1}{12}\cdot 12=1 \]
\[\langle \chi,\chi_{\rho_1}\rangle=\frac{1}{12}(30+3(\zeta^3+\zeta^{-3}))=\frac{1}{12}(30-6)=2 \]
\[\langle \chi,\chi_{\rho_2}\rangle=\frac{1}{12}(30+3(\zeta^6+\zeta^{-6}))=\frac{1}{12}(30+6)=3 \]
If we denote the corresponding $G$-module of $\chi_i$ by $V_i$ and $\chi_{\rho_i}$ by $W_i$, we have 
\[V\cong 3V_1\oplus V_2\oplus V_4\oplus 2W_1\oplus 3W_2\]
\qed\\
\textbf{Problem 6:} We use the same notation as above. Notice that the endomorphism algebra of $V$ is isomorphic to the commutant algebra of $V$. So we have 
\[Com(V)\cong Com(3V_1\oplus V_2\oplus V_4\oplus 2W_1\oplus 3W_2)\cong Mat_3(\C)\oplus \C\oplus \C\oplus Mat_2(\C)\oplus Mat_3(\C)\]
And the dimension of $Com(V)$ as a $\C$-vector space is the sum of dimension of these components. So we have $\dim Com(V)=3^2+1+1+2^2+3^2=9+1+1+4+9=24$.
\\\qed\\
\textbf{Problem 7:} By Maschke's Theorem, we suppose $V\cong \oplus_{i=1}^rm_iV_i$ and $W\cong \oplus_{i=1}^sn_iW_i$, where $m_i,n_i\geq 1$, $V_i$'s are pairwises nonisomorphic irreducible $G-$modules, and $W_i$'s are pairwise nonisomorphic irreducible $G-$modules. Let 
$\sum_{i=1}^rm_i=\alpha$ and $\sum_{i=1}^sn_i=\beta$
Then an element $\phi\in Hom(V,W)$ can be written as a block matrix
\[\phi=
    \begin{pmatrix}
    M_{11}&M_{12}&\cdots & M_{1\alpha}\\
    M_{21}&M_{22}&\cdots & M_{2\alpha}\\
    \vdots\\
    M_{\beta 1 }&M_{\beta 2}&\cdots & M_{\beta\alpha}  
    \end{pmatrix}
    \]
If $\sum_{i=1}^pm_i\leq a\leq \sum_{i=1}^{p+1}m_i$, and $\sum_{i=1}^qn_i\leq b\leq \sum_{i=1}^{q+1}n_i$, then $M_{ba}$ is the matrix that represents the linear transformation from the $(a-\sum_{i=1}^pm_i)$-th copy of $V_{p+1}$ to the $(b-\sum_{i=1}^{q}n_i)$-th copy of $W_{q+1}$.  Now, by Schur's lemma, we know $M_{ba}$ is either trivial or complex scalar multiplication. Notice that nonzero complex scalar multiplication is an isomorphism because $\C$ is a field (or follows from Baby Schur's lemma). So the dimension of $Hom(V,W)$ depends on how many isomorphic pairs $V_i$ with $W_j$. Since we can rearrange the direct sum, we suppose the first $k$ distinct irreducible $G$-submodules of $V$ and $W$ are isomorphic (i.e. $V_1\cong W_1$, $\cdots, \ V_k\cong W_k$). And there is no other isomorphic pairs of submodules. So we have 
\[Hom(V,W)\cong \oplus_{i=1}^k Mat_{n_i\times m_i}(\C)\]
\[\dim(Hom(V,W))=(\sum_{i=1}^kn_im_i)\]
On the other side, we have 
\[\langle\sum_{i=1}^rm_i\chi_{V_i},\sum_{i=1}^sn_i\chi_{W_i}\rangle=\sum_{i=1}^r\sum_{j=1}^sm_in_j\langle\chi_{V_i},\chi_{W_j}\rangle=(\sum_{i=1}^kn_im_i)\]
By character orthogonality, we have $\langle\chi_{V_i},\chi_{W_j}\rangle=1$ if and only if $V_i\cong W_j$ if and only if $1\leq i=j\leq k$. And $\langle\chi_{V_i},\chi_{W_j}\rangle=0$ otherwise.
So we have $\dim (Hom(V,W))=\langle\chi_V,\chi_W\rangle$.
\\\qed\\
\textbf{Problem 8:} Notice that the conjugacy classes of symmetric group is given by the cycle type. So each conjugacy class of $A_4$ is either a refinement of the original conjugacy classes in $S_4$ or the same as the original one. Since $8\nmid |A_4|=12$, the conjugacy class of $S_4$ with 3-cycle splits. By the following calculations,
\[(123)(12)(34)(132)=(14)(23),\  (234)(12)(34)(243)=(13)(24)\]
\[(12)(34)(123)(34)(12)=(142)\]
\[(13)(24)(123)(24)(13)=(134)\]
\[(14)(23)(123)(23)(14)=(243)\]
\[(12)(34)(132)(34)(12)=(124)\]
\[(13)(24)(132)(24)(13)=(143)\]
\[(14)(23)(132)(23)(14)=(234)\]
So we have four conjugacy classes. And the only possibility for $d_1^2+d_2^2+d_3^2+d_4^2=12$ is $1,1,1,3$, where $d_1=1$ is the dimension of trivial representation. 
Hence, we have 
\begin{table}[H]
    \centering
    \begin{tabular}{l|ccccl}
        $A_4$    & e & (12)(34) & (123) & (132)  \\ \hline
        $\chi_1$ & 1 & 1  & 1       & 1            \\
        $\chi_2$ & 1 &            \\
        $\chi_3$ & 1 &             \\
        $\chi_4$ & 3 &             
    \end{tabular}
\end{table}
For dimension 1, since $(12)(34)$ has order 2, we have $\chi((12)(34))=\pm $1. Similarly, since $(123),(132)$ has order 3, we have $\chi((123))=1,\zeta,\zeta^2$ and $\chi((132))=1,\zeta,\zeta^2$, where $\zeta=e^{\frac{2\pi }{3}i}$.
Since $(123)(12)(34)=(134)$, we have 
\[\chi(123)\chi((12)(34))=\chi(132)=\chi(123)\]
So we have $\chi((12(34)))=1$. If $\chi(123)=\zeta$, then by character orthogonality, we have 
\[\frac{1}{12}(1+3+4\bar{\zeta}+4\overline{\chi(132)})=0\]
\[\chi(132)=-\zeta-1=-\cos(\frac{2\pi}{3})-1-\sin(\frac{2\pi}{3})i=-\frac{1}{2}+\sin(-\frac{2\pi}{3})i=\cos(\frac{4\pi}{3})+\sin(\frac{4\pi}{3})i=\zeta^2\]
If $\chi(123)=\zeta^2$, then we have 
\[\chi(132)=-\zeta^2-1=\zeta\]
So we have the first three rows.
\begin{table}[H]
    \centering
    \begin{tabular}{l|ccccl}
        $A_4$    & e & (12)(34) & (123) & (132)  \\ \hline
        $\chi_1$ & 1 & 1  & 1       & 1            \\
        $\chi_2$ & 1 & 1  & $\zeta$ & $\zeta^2$         \\
        $\chi_3$ & 1 & 1  & $\zeta^2$ & $\zeta$          \\
        $\chi_4$ & 3 & -1 & 0 &0           
    \end{tabular}
\end{table}
where the last row follows from the character orthogonality.
\\\qed
\end{document}