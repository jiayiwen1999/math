\documentclass[12pt]{amsart}
\usepackage{amsmath,epsfig,fancyhdr,amssymb,subfigure,setspace,fullpage,mathrsfs,upgreek,float, tabularx, tikz-cd}
\usepackage[utf8]{inputenc}
\usepackage[boxsize=1.25em, centerboxes]{ytableau}

\newcommand{\R}{\mathbb{R}}
\newcommand{\Q}{\mathbb{Q}}
\newcommand{\C}{\mathbb{C}}
\newcommand{\Z}{\mathbb{Z}}
\newcommand{\N}{\mathbb{N}}
\newcommand{\G}{\mathcal{N}}
\newcommand{\F}{\mathbb{F}}
\newcommand{\A}{\mathcal{A}}
\newcommand{\sB}{\mathscr{B}}
\newcommand{\sC}{\mathscr{C}}
\newcommand{\sd}{{\Sigma\Delta}}
\newcommand{\Orbit}{\mathcal{O}}
\newcommand{\ord}{\mathrm{ord}}
\newcommand{\normal}{\triangleleft}

\begin{document}
\title{Homework 3 - 202B}
\maketitle
\begin{center}
    Jiayi Wen\\
    A15157596
\end{center}
\textbf{Problem 1:} Denote the map as 
\[f:W\to W\]
\[w\mapsto \sum_{g\in G}\overline{\chi(g)}g\cdot w\]
Now, we want to show this is a $G$-module homomorphism. For any $w_1,w_2\in W$ and $c\in \C$, we have 
\[f(w_1+w_2)=\sum_{g\in G}\overline{\chi(g)}g\cdot (w_1+w_2)=\sum_{g\in G}\overline{\chi(g)}g\cdot w_1+\sum_{g\in G}\overline{\chi(g)}g\cdot w_2=f(w_1)+f(w_2)\]
\[f(cw_1)=\sum_{g\in G}\overline{\chi(g)}g\cdot cw_1=c\sum_{g\in G}\overline{\chi(g)}g\cdot w=cf(w_1)\]
For any $h\in G$, we have 
\[f(h\cdot w_1)=\sum_{g\in G}\overline{\chi(g)}g\cdot h\cdot w_1=\sum_{hgh^{-1}\in G}\overline{\chi(hgh^{-1})}hgh^{-1} h\cdot w_1=h\cdot\sum_{hgh^{-1}\in G}\overline{\chi(g)}g\cdot w_1=h\cdot f(w_1)\]
Since, we have all niceness assumption here, we can apply Schur's lemma. Then $f(w)=cw$ for some $c\in \C$. Then we have $tr(f)=c\dim(W)$. On the other hand, if $\psi$ is the character of $W$, then we have 
\[tr(f)=\sum_{g\in G}\overline{\chi(g)}\psi(g)=|G|\langle\chi,\psi\rangle \]
Since both $V$ and $W$ are irreducible, the inner product is either 0 or 1. So if $W\cong V$, then we have 
\[c=\frac{tr(f)}{\dim(W)}=\frac{|G|}{\dim(W)}\]
If $W\ncong V$, then we have 
\[c=\frac{tr(f)}{\dim(W)}=0\]
\qed\\
\textbf{Problem 2:} Since $M^\lambda\cong \C[S_6/S_\lambda]$, we just need to decomposite $M^\lambda$. By Young's Rule, we have 
\[M^\lambda\cong_{S_6} \bigoplus_{\mu\vdash 6}K_{\mu\lambda}S^\mu\]
Then we have 7 possibilities.
\[\begin{ytableau}
    1&1&1&2&2&3
\end{ytableau}
\  \begin{ytableau}
    1&1&1&2&2\\3
\end{ytableau}  
\  \begin{ytableau}
    1&1&1&2&3\\2
\end{ytableau}  
\]\qed\\
\[\begin{ytableau}
    1&1&1&2\\
    2&3
\end{ytableau}
\  \begin{ytableau}
    1&1&1&3\\
    2&2
\end{ytableau}  
\  \begin{ytableau}
    1&1&1&2\\
    2\\
    3
\end{ytableau}  
\  \begin{ytableau}
    1&1&1\\
    2&2&3
\end{ytableau}  
\  \begin{ytableau}
    1&1&1\\
    2&2\\3
\end{ytableau}  
\]
So we have 
\[M^\lambda\cong S^{(6)}\oplus 2S^{(5,1)}\oplus 2S^{(4,2)}\oplus S^{(4,1,1)}\oplus  S^{(3,3)}\oplus S^{(3,2,1)}\]\qed\\
\textbf{Problem 3:} Let's compute $g_{\lambda,\lambda,(n)}$ first. Notice that this is the same as computing the multiplicity of $S^{(n)}$ in $S^{\lambda}\otimes S^\lambda$. And we know that $S^{(n)}$ is just the trivial representation. So we have 
\[\langle\chi_{S^\lambda\otimes S^\lambda},\chi_{triv}\rangle=\frac{1}{|S_n|}\sum_{g\in S_n}\chi^\lambda(g)\chi^\lambda(g)\overline{\chi_{triv}}(g)=\frac{1}{n!}\sum_{g\in S_n}(\chi^\lambda(g))^2\]
We claim that $g_{\lambda,\lambda',(1^n)}$ has the same formula. Notice that $S^{(1^n)}$ is just the sign permuation of $S_n$. We denote it as $S^{(1^n)}=sign_n$. Since $S^{\lambda'}\cong S^\lambda\otimes sign_{n}$, we have $\chi^{\lambda'}=\chi^\lambda\chi_{sign}$. So we have 
\[g_{\lambda,\lambda',(1^n)}=\langle\chi_{S^\lambda\otimes S^{\lambda'}},\chi_{sign}\rangle=\frac{1}{|S_n|}\sum_{g\in S_n}\chi^\lambda(g)\chi^\lambda(g)\chi_{sign}(g)\overline{\chi_{sign}}(g)\]
Since $\chi_{sign}(g)=sign(g)$, which is either 1 or -1, we have $\chi_{sign}(g)\overline{\chi_{sign}}(g)=(sign(g))^2=1$ for all $g\in S_n$. Hence,
\[g_{\lambda,\lambda',(1^n)}=\frac{1}{n!}\sum_{g\in S_n}(\chi^\lambda(g))^2\]
\qed\\
\textbf{Problem 4:} It is sufficient to show the irreducible character of symmetric groups has real value (actually rational). Suppose $g\in S_n$, and $\chi: S_n\to \C$ is an irreducible character. Then we have 
\[\chi(g)^{\ord(g)}=1\]
So $\chi(g)\in \Q(\zeta_{\ord(g)})$, where $\zeta_{\ord(g)}=e^{\frac{2\pi i }{\ord(g)}}$. If we decomposite $g$ into disjoint cycles, then we know the order of $g$ is the lcm of the length of each cycle. If $\gcd(k,\ord(g))=1$, then $g^k$ has the same cycle type as $g$ since $k$ doesn't kill any cycle in $g$. Then we have $\chi(g)=\chi(g^k)$. But notice that the Galois group of $\Q(\zeta_\ord(g))\cong (\Z/\ord(g)\Z)^\times$. So $\chi(g)$ is fixed under the group action by the Galois group. So we have $\chi(g)\in \Q$. Thus we have 
\[g_{\lambda,\mu,\nu }=\langle\chi^\lambda\chi^\mu,\chi^\nu\rangle=\sum_{g\in S_n}\frac{1}{n!}\chi^\lambda(g)\chi^\mu(g)\overline{\chi^\nu(g)}=\sum_{g\in S_n}\frac{1}{n!}\chi^\lambda(g)\chi^\mu(g)\chi^\nu(g)\]
So $g_{\lambda,\mu,\nu }$ is invariant under the perumation of $\lambda,\mu,\nu$.\\\qed\\
\textbf{Problem 5:} We can do this inductively using Littlewood-Richardson Rule. 
\[S^{(3)}\otimes S^{(1,1,1)}\uparrow_H^{S_6}\cong \bigoplus_{\lambda\vdash 6}c_{(3),(1,1,1)}^\lambda S^\lambda \]
First observation is that since we have content $(1,1,1)$, so we have 3 distinct numbers. Now, if the skew shape has more than one boxes at each row, then we will never get a reserve ballot since the smaller number is after some larger number in the reading word. So we have $c^\lambda_{(3),(1,1,1)}=0$ for all $\lambda\vdash 6$ except $\lambda=(4,1,1)$ and we have exactly one possible way to fill in the content.
\[\begin{ytableau}
    \times&\times &\times &1\\
    2\\3
\end{ytableau} \]
So we have $S^{(3)}\otimes S^{(1,1,1)}\uparrow_H^{S_6}\cong S^{(4,1,1)}$. Now, we do the same step.
\[S^{(4,1,1)}\otimes S^{(1)}\uparrow_{S_6\times S_1}^{S_7}\cong \bigoplus_{\lambda\vdash 7}c_{(4,1,1),(1)}^\lambda S^\lambda \]
This is much simple since we only have two possible choices of $\lambda$ as the following
\[
    \begin{ytableau}
    \times&\times &\times &\times &1\\
    \times \\\times 
\end{ytableau}
\     \begin{ytableau}
    \times&\times &\times &\times \\
    \times &1\\\times
\end{ytableau}
 \]
 So we have $S^{(4,1,1)}\otimes S^{(1)}\uparrow_{S_6\times S_1}^{S_7}\cong S^{(5,1,1)}\oplus S^{(4,2,1)}$. Thus, we have 
 \[End(V)\cong Mat_1(\C)\oplus Mat_1(\C)\cong \C^2\]
 So $End(V)$ has dimension 2.\\ \qed\\
\textbf{Problem 6:}
Let's do this by direction calculation. First, we know the irreducible character of $S_3\times S_2$ arise as the tensor product of irreducibles. And we already know the character table of $S_3$ and $S_2$. So we just need to understand $\chi^{(3,2)}\downarrow_{S_3\times S_2}$. We do the computation by Murnaghan-Nalcayama Rule. Notice that we only have five cycle types in $S_3\times S_2$. They are $(1,1,1,1,1),\ (2,1,1,1),\ (2,2,1),\ (3,1,1),$ and $(3,2)$. Notice that only $(1,1,1,1,1)$ has 5 different ways to remove the blocks to empty, the rest has a unique way to do so. 
\[\chi^{(3,2)}_{(1,1,1,1,1)}=5\]
\[\chi^{(3,2)}_{(2,1,1,1)}=(-1)^0(-1)^0(-1)^0(-1)^0=1\]
\[\chi^{(3,2)}_{(2,2,1)}=(-1)^0(-1)^0(-1)^0=1\]
\[\chi^{(3,2)}_{(3,1,1)}=(-1)^1(-1)^0(-1)^0=-1\]
\[\chi^{(3,2)}_{(3,2)}=(-1)^1(-1)^1=1\]
So we have 
\[\langle\chi^{(3,2)}\downarrow_{S_3\times S_2},\chi^{triv}\chi^{triv}\rangle=\frac{1}{12}(5+4+3-2+2)=1\]
\[\langle\chi^{(3,2)}\downarrow_{S_3\times S_2},\chi^{sign}\chi^{triv}\rangle=\frac{1}{12}(5+1\cdot 1\cdot1\cdot 1 +3\cdot1\cdot (-1)\cdot 1+3\cdot 1\cdot (-1)\cdot 1+2\cdot (-1)\cdot 1\cdot 1+2)=0\]
\[\langle\chi^{(3,2)}\downarrow_{S_3\times S_2},\chi^{def}\chi^{triv}\rangle=\frac{1}{12}(1\cdot 5\cdot 2+1\cdot 1\cdot2\cdot 1 +3\cdot1\cdot 0\cdot 1+3\cdot 1\cdot 0\cdot 1+2\cdot (-1)\cdot (-1)\cdot 1+2\cdot 1\cdot (-1)\cdot 1)=1\]
\[\langle\chi^{(3,2)}\downarrow_{S_3\times S_2},\chi^{triv}\chi^{sign}\rangle=\frac{1}{12}(5+3-1-3-2-2)=0\]
\[\langle\chi^{(3,2)}\downarrow_{S_3\times S_2},\chi^{sign}\chi^{sign}\rangle=\frac{1}{12}(5+1\cdot 1\cdot1\cdot( -1) +3\cdot1\cdot (-1)\cdot 1+3\cdot 1\cdot (-1)\cdot (-1)+2\cdot (-1)\cdot 1\cdot 1+2\cdot 1\cdot 1\cdot (-1))=0\]
\[\langle\chi^{(3,2)}\downarrow_{S_3\times S_2},\chi^{def}\chi^{sign}\rangle=\frac{1}{12}(1\cdot 5\cdot 2+1\cdot 1\cdot2\cdot (-1) +3\cdot1\cdot 0\cdot 1+3\cdot 1\cdot 0\cdot (-1)+2\cdot (-1)\cdot (-1)\cdot 1+2\cdot 1\cdot (-1)\cdot (-1))=1\]
So we have $S^{(3,2)}\downarrow_{S_3\times S_2}\cong S^{(3)}\otimes S^{(2)}\oplus S^{(2,1)}\otimes S^{(2)}\oplus S^{(2,1)}\otimes S^{(1,1)}$
\qed\\
\end{document}