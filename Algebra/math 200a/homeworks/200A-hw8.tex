\documentclass[12pt]{amsart}
\usepackage{amsmath,epsfig,fancyhdr,amssymb,subfigure,setspace,fullpage,mathrsfs,upgreek}
\usepackage[utf8]{inputenc}

\newcommand{\R}{\mathbb{R}}
\newcommand{\Q}{\mathbb{Q}}
\newcommand{\C}{\mathbb{C}}
\newcommand{\Z}{\mathbb{Z}}
\newcommand{\N}{\mathbb{N}}
\newcommand{\G}{\mathcal{N}}
\newcommand{\A}{\mathcal{A}}
\newcommand{\sB}{\mathscr{B}}
\newcommand{\sC}{\mathscr{C}}
\newcommand{\sd}{{\Sigma\Delta}}
\newcommand{\Orbit}{\mathcal{O}}
\newcommand{\normal}{\triangleleft}

\begin{document}
\title{Homework7 - 200A}
\maketitle
\begin{center}
    Jiayi Wen\\
    A15157596
\end{center}
From now, the map composition will be from right to left as usual. For example, $f\circ g$ denote the map do $g$ first, then do $f$.\\
\textbf{Problem 1:} WLOG, we can assume that $I$ cannot be generated by any proper subset of $\{r_1,\dots,r_n\}$. Otherwise, we can take a smaller generating set instead. The idea of the proof is to use Zorn's Lemma on a poset. We define a poset $$P:=\{J\subsetneq R\mid \text{ proper ideals, with } r_i\notin J \text{ for some } 1\leq i\leq n \}$$
where the partial order is defined by set inclusion. It is obvious that the ideal $(r_1,r_2,\dots,r_{n-1})$ doesn't contain $r_n$ by our assumption. Hence, $(r_1,r_2,\dots,r_{n-1})\in P$. So $P$ is nonempty.\\
Let $B$ be a chain in $P$. Since $B$ has a total order, we can assume $B=\{I_\alpha\mid \alpha\in A, r_n\notin I_\alpha\}$, where $A$ is an index set. Otherwise, the total order will force some element in $B$ to contain $I$. Then we can consider the set $J=\cup_{\alpha\in A}I_\alpha$. We claim $J$ is an ideal in $P$.\\
If $x,y\in J$, then we have $x\in I_\alpha$ and $y\in I_\beta$ for some $\alpha,\beta\in A$. WLOG, we can assume $I_\alpha\leq I_\beta$. Then we have $x\in I_\beta$ and $x-y\in I_\beta$ because $I_\beta$ is a group with respect to addition. Hence, $x-y\in J$. So $J$ is a group with respect to addition.\\
If $x\in J$ and $r\in R$, then there exists some $\alpha\in A$ such that $x\in I_\alpha$. So we have $xr\in I_\alpha$ because $I_\alpha$ is an ideal. Hence, $xr\in J$. So $J$ is an ideal. Because $r_n\notin I_\alpha$ for all $\alpha\in A$, we have $r_n\notin J$. Hence, $J\in P$. And $I_\alpha\subseteq J$ for all $I_\alpha\in B$ because $J$ is the union of all element in $B$. So $B$ has an upper bound in $P$. Then by Zorn's Lemma, $P$ has a maximal element because $P$ is a nonempty poset where every chain in $P$ has an upper bound. The proof completes.
\\\phantom{qed}\hfill$\square$\\
\textbf{Problem 2:}
Let $P$ be poset that consists of all prime ideal of $R$ and the partial order is defined by set inclusion. It is obvious that $P$ is a nonempty set because we proved in the lecture that every $R$ has a maximal ideal, which is prime. Suppose $B=\{I_\alpha\mid \alpha\in A\}$ is a chain in $P$, where $A$ is an index set. Then we claim $J=\cap_{\alpha\in A} I_\alpha$ is an ideal in $P$. Because the intersection of subgroups is still a subgroup, we know $J$ is a subgroup of $R$ with respect to addition. For any $s\in J$ and $r\in R$, we have $s\in I_\alpha$ for any $\alpha\in A$. Then we have $sr\in I_\alpha$ for any $\alpha\in A$. Hence, $sr\in J$. So $J$ is an ideal of $R$.  Also, we have $J\subseteq I_\alpha$ for any $\alpha\in A$. So all we need to do is show $J$ is prime.\\
We suppose the opposite. So there exists some $a,b\in R$ such that $ab\in J$ but $a\notin J$ and $b\notin J$. Hence, there exists some $\alpha,\beta\in A$ such that $a\notin I_\alpha$ and $b\notin I_\beta$. WLOG, we can assume $I_\alpha\subseteq I_\beta$. Then we know $b\notin I_\alpha$. However, Since $J\subseteq I_\alpha$, we have $ab\in I_\alpha$. This contradicts to the fact that $I_\alpha$ is a prime ideal. Hence, $J$ must be a prime ideal. So $J\in P$. So $B$ has a lower bound in $P$. So does every chain in $P$. Then, by Zorn's Lemma, we know $P$ has a minimal element. So $R$ has a minimal prime ideal.
\\\phantom{qed}\hfill$\square$\\
\pagebreak\\
\textbf{Problem 3:}\\
(a): If $I=\cap_{i=1}^n M_i$, where $M_i$ are maximal ideals of $R$. WLOG, we can assume they are distinct. Then we have $M_i,M_j$ are comaximal if $i\neq j$ because $M_i+M_j$ is an ideal that properly contains $M_i$, which forces $M_i+M_j=R$. Then by Chinese Remainder Theorem, we have 
\[R/I\cong R/M_1\times R/M_2\times \dots\times R/M_n\]
Notice that $R/M_i$ is a field because $M_i$ is maximal. So we have $R/I$ is isomorphic to the direct product of finitely many fields.\\
Conversely, if $R/I$ is isomorphic to the direct product of finitely many fields, then we can assume 
\[R/I\cong F_1\times F_2\times \dots \times F_n\]
Then we can consider the canonical projection $\pi_i: F_1\times F_2\times \dots \times F_n\to F_i$ such that $\pi_i(f_1,\dots,f_n)=f_i$. Then if $\varphi:R\to F_1\times F_2\times \dots \times F_n$ is a surjective homomorphism, whose kernel is $I$, then we have a surjective ring homomorphism $\phi_i:R\to F_i$ such that $\phi_i=\pi_i\circ \varphi$. Then by the first isomorphism, we have $R/ker\phi_i\cong F_i$ is a field. Hence, $ker\phi_i$ is a maximal ideal of $R$. And we have an isomorphism $\gamma_i:R/ker\phi_i\to F_i$ such that $\gamma_i\circ p_i=\phi_i$, where $p_i(r)=r+ker\phi_i$ is the canonical mapping. Then we have an isomorphism 
\[\gamma:R/ker\phi_1\times R/ker\phi_2\times \dots\times R/ker\phi_n \to F_1\times F_2\times \dots \times F_n\]
\[\gamma(r_1,\dots,r_n)=\big(\gamma_1(r_1),\dots,\gamma_n(r_n)\big)\]
It is obvious this is a bijection because $\gamma_i$ is a bijection for all $i$. We have $\gamma(r_1,\dots,r_n)=(0,\dots,0)$ if and only if $\gamma_i(r_i)=0 $ if and only if $r_i=0$. And $\forall (f_1,\dots,f_n)\in F_1\times F_2\times \dots \times F_n$, we have $\gamma_i(r_i)=f_i$ for all $i$. So $\gamma(r_1,\dots,r_n)=(f_1,\dots,f_n)$. It is a homomorphism because
\begin{align*}
    \gamma\big((r_1,\dots,r_n)+(s_1,\dots,s_n)\big)&=\gamma\big(r_1+s_1,\dots,r_n+s_n\big)\\
    &=\big(\gamma_1(r_1+s_1),\dots,\gamma_n(r_n+s_n)\big)\\
    &=\big(\gamma_1(r_1),\dots,\gamma_n(r_n)\big)+\big(\gamma_1(s_1),\dots,\gamma_n(s_n)\big)\\
    &=\gamma\big(r_1,\dots,r_n\big)+\gamma\big(s_1,\dots,s_n\big)
\end{align*}
\begin{align*}
    \gamma\big((r_1,\dots,r_n)\cdot(s_1,\dots,s_n)\big)&=\gamma\big(r_1s_1,\dots,r_ns_n\big)\\
    &=\big(\gamma_1(r_1s_1),\dots,\gamma_n(r_ns_n)\big)\\
    &=\big(\gamma_1(r_1),\dots,\gamma_n(r_n)\big)\cdot\big(\gamma_1(s_1),\dots,\gamma_n(s_n)\big)\\
    &=\gamma\big(r_1,\dots,r_n\big)\cdot\gamma\big(s_1,\dots,s_n\big)
\end{align*}
Then we claim $\varphi=\gamma\circ p$, where $p:R\to R/ker\phi_1\times R/ker\phi_2\times \dots\times R/ker\phi_n$ is the canonical morphism $p(r)=(r+ker\phi_1,\dots,r+ker\phi_n)$. Notice that 
\[\gamma\circ p=(\gamma_1\circ p_1,\dots,\gamma_n\circ p_n)=(\pi_1\circ \varphi, \dots,\pi_n\circ \varphi)=\varphi \]
So we have $ker\varphi=kerp$ because $\gamma$ is an isomorphism. Hence, we have 
\[I=ker\varphi=kerp=\cap_{i=1^n}ker\phi_i\]
So $I$ is the intersection of finitely many maximal ideals of $R$.
\\(b): We prove by induction. For $n=1$, we suppose $I=M_1$. If we also have $I=\cap_{i=1}^kM'_i$, where $M'_i$ are distinct maximal ideals and $k\geq 2$, then we know $I\subsetneq M'_i$ for any $1\leq i\leq k$. But this contradicts to the fact that $I=M_1$ is a maximal ideal. So we must have $k=1$. Therefore, we have $M_1=I=M'_1$.\\
Then we suppose for $n\leq k-1$ the statement is true. Then we can suppose $I=\cap_{i=1}^k M_i$, distinct maximal ideals. If we have $I=\cap_{i=1}^l M'_i$ for some $l\in \N$, we can assume $l\geq k$ because if $l<k$, then the induction hypothesis says the choice of maximal ideals are unique up to rearrangement and this contradicts to the assumption that $M_i$ are distinct.\\
Then we can consider the intersection of all them.
\[I=I\cap I=(\cap_{i=1}^k M_i)\cap(\cap_{i=1}^l M'_i)\]
By set theory, for any $1\leq i \leq l$ we have 
\[I=(\cap_{i=1}^k M_i)\cap(\cap_{i=1}^l M'_i)\subseteq(\cap_{i=1}^k M_i)\cap M'_i\subseteq \cap_{i=1}^k M_i= I \]
So we have $I\cap M'_i=(\cap_{i=1}^k M_i)\cap M'_i=I$. If $M'_i\neq M_j$ for all $1\leq j\leq k$. Then by Chinese Remainder Theorem, we have
\[R/M_1\times \dots \times R/M_k\cong R/I\cong R/M_1\times \dots \times R/M_k\times R/M'_i\]
But this is impossible if we count the ideals. To claim this, we need two lemmas about the ideals.\\
\textbf{Lemma 1:} If $R,S$ are commutative rings, then the ideals of $R\times S$ is in the form of $I\times J$, where $I$ and $J$ are ideals of $R$ and $S$, respectively.\\
\textbf{Proof of Lemma 1:} If we consider $R\times S$ in the category of abelian groups, then we know the subgroups of $R\times S$ is in the form of $I\times J$, where $I$ and $J$ are subgroups of $R$ and $S$, respectively. Now, all we need to show is $I$ and $J$ are ideals. Since $I\times J$ is an ideal of $R\times S$, then we have for any $(r,s)\in R\times S$ and $(x,y)\in I\times J$, we have 
\[(r,s)\cdot (x,y)=(rx,sy)\in I\times J\]
Hence, we have $rx\in I$ and $sy\in J$. So $I$ and $J$ are ideals. This completes the proof of the lemma.\\
Notice that the lemma stays true for finite direct product by induction.\\
\textbf{Lemma 2:} If $\varphi: R\to S$ is a ring isomorphism, then for any subgroup $I\leq R$ is an ideal if and only if $\varphi(I)$ is an ideal of $S$.\\
\textbf{Proof of Lemma 2:} If $I$ is an ideal, then for any $s\in S$ and $y\in \varphi(I)$, we have $\varphi(r)=s$ and $\varphi(x)=y$ for some $r\in R$ and $x\in I$. Then we have $rx\in I$ and
\[sy=\varphi(r)\varphi(x)=\varphi(rx)\in \varphi(I)\]
Conversely, if $J=\varphi(I)$ is an ideal of $S$, for any $r\in R$ and $x\in I$, we have $\varphi(x)\in J$. Hence, $\varphi(rx)=\varphi(r)\varphi(x)\in J=\varphi(I)$. But this implies $rx\in I$. So $I$ is an ideal of $R$. So the isomorphism of two rings gives an 1-1 correspondence between ideals.\\
Now we come back to prove the problem. We know $ R/M_1\times \dots \times R/M_k$ has $2^k$ ideals because each of $R/M_i$ is a field, which has two ideals ($0$ and $R/M_i$). However, $ R/M_1\times \dots \times R/M_k\times R/M'_i$ has $2^{k+1}$ ideals. It contradicts. Hence, $M'_i=M_j$ for some $j$. Hence, we should have $l=k$ by pigeonhole principle and $M'_i$'s are just rearrangement of $M_i$'s.  
\\(c): A simple example can be found in the Klein four-group. Let $K_4=\{1,i,j,k\}$. Then we have $\langle i\rangle\cong \langle j\rangle\cong\langle k\rangle\cong \Z/2\Z$ are maximal subgroups of $K_4$. Then we consider the trivial subgroup, we have 
$$1=\langle i\rangle\cap\langle j\rangle=\langle i\rangle\cap\langle k\rangle=\langle j\rangle\cap\langle k\rangle$$
\\\phantom{qed}\hfill$\square$\\
\textbf{Problem 4:}\\
(a): As we proved in last homework, $f=\sum_{n=0}^\infty a_nx^n$ is a unit in $F[[x]]$ if and only if $a_0\neq 0$. And notice that $F[[x]]\subseteq F((x))$ by definition. So $f$ is also a unit in $F((x))$. So we just need to show that $f\in F((x))-\{0\}$ is a unit for all $f$ with zero constant term or $f$ with negative power of $x$. If $f=\sum_{n=1}^\infty a_nx^n$, we assume the smallest non-zero coefficient of $f$ is $a_i$, then we can rewrite $f=\sum_{n=i}^\infty a_nx^n$. Take $g=x^{-i}$, which is a unit in $F((x))$ because $gx^i=1$. We have 
\[fg=\sum_{n=i}^\infty a_nx^{n-i}=\sum_{n=0}^\infty a_{n+i}x^n\]
We know $fg$ has non-zero constant term $a_i$. So $fg$ is a unit. Then there exists $h\in F((x))$ such that $fgh=1$. So $f$ is a unit.\\
Similarly, if $f=\sum_{n=N}^\infty a_nx^n$ for some $N\in \Z$ and $N<0$, we can assume $a_N\neq 0$. Then $fx^{-N}=\sum_{n=0}^\infty a_{n+N}x^n$ has non-zero constant term. Hence, $fx^{-N}$ is a unit. So, there exists some $h\in F((x))$ such that $fx^{-N}h=1$. So $f$ is a unit. So $F((x))$ is a field.\\
(b): Let $X=F[[x]]-\{0\}$, then the field of fraction of $F[[x]]$ is $F[[x]]X^{-1}$. Suppose $i:F[[x]]\to F[[x]]X^{-1}$ is the canonical embedding defined by $i(f)=[\frac{f}{1}]$. Then we can consider another embedding $\varphi:F[[x]]\to F((x))$ by $\varphi(f)=f$. Then we have $\varphi$ is injective and $F((x))$ is a field. Hence, $\varphi(f)$ is a unit for all $f\in X$. Then there exists a unique homomorphism $\theta:F[[x]]X^{-1}\to F((x))$ such that $\theta\circ i=\varphi$ and $\theta([\frac{r}{x}])=\varphi(r)\varphi(x)^{-1}$ by the universal property of localization. Now, we want to prove $\theta$ is an isomorphism. Because $F[[x]]X^{-1}$ is a field, so $\theta$ is injective. \\
For any $f\in F((x))$, we assume $f=\sum_{n=N}^\infty a_nx^n$ and $a_N\neq 0$. If $N\geq 0$, then we have $\varphi(f)=f$. Hence, we have $\theta\circ i(f)=\theta ([\frac{f}{1}])=f$. If $N<0$, then we have $fx^{-N}\in F[[x]]$ and $x^{-N}\in X$ because $-N> 0$. and we have 
$$\theta([\frac{fx^{-N} }{x^{-N}}])=\varphi(fx^{-N} )\varphi(x^{-N} )^{-1}=fx^{-N} (x^{-N})^{-1}=fx^{-N} x^N=f $$
So, we have $\theta$ is surjective. Hence, $\theta$ is an isomorphism. Hence, $F((x))\cong F[[x]]X^{-1}$.\\
(c): It is obvious that the field of fraction $\Z[[x]]X^{-1}\subseteq \Q((x))$ because $\Z[[x]]\subseteq \Q((x))$ is a subring, which gives a canonical inclusion map $i:\Z[[x]]\to \Q((x))$, then the universal property of localization implies the existence of a field homomorphism from $\Z[[x]]X^{-1}$ to $\Q((x))$. Consider $e^x=\sum_{n=0}^\infty\frac{x^n}{n!}\in \Q((x))$. We suppose $Q((x))$ is the field of fraction of $\Z[[x]]$ towards contradiction. Then there exists some $f,g\in\Z[[x]]-\{0\}$ such that $e^x=[\frac{f}{g}]$. Hence, we have $f=e^xg$. Suppose $f=\sum_{n=0}^\infty a_nx^n$ and $g=\sum_{n=0}^\infty b_nx^n$, where $a_n,b_,n\in\Z$. We can assume $a_0,b_0$ are not both 0 because if $a_0=b_0=0$, we can factor $f=f'x$ and $g=g'x$ for some $f'=\sum_{n=0}^\infty a_{n+1}x^n,g'=\sum_{n=0}^\infty b_{n+1}x^n$ and $\frac{f}{g}=\frac{f'}{g'}$. Then for any $k\in\N$, we can look at the $k$-th coefficient of both sides.
\[a_k=\sum_{j=0}^k \frac{b_{k-j}}{j!}\]
If $k=p$ for some prime $p$, then we have 
\[a_p=\sum_{j=0}^p \frac{b_{p-j}}{j!}=\sum_{j=0}^p \frac{b_{j}}{(p-j)!}\]
\[p!a_p=\sum_{j=0}^p b_{j}\frac{p!}{(p-j)!}=b_0+p\sum_{j=1}^p b_{j}\frac{(p-1)!}{(p-j)!}\]
Notice that $\frac{(p-1)!}{(p-j)!}\in \Z$ because $1\leq j\leq p$. So we have $p\mid b_0$ because $p\mid p!a_p$ and $p\mid p\sum_{j=1}^p b_{j}\frac{(p-1)!}{(p-j)!}$. And this holds for all prime $p$ in $\Z$. Hence, $b_0=0$. This is a result from elementary number theory by the unique prime factorization of integers. Then we have 
\[a_0=\sum_{j=0}^0 \frac{b_{0-j}}{j!}=\frac{b_0}{0!}=0\]
But this contradicts to our choice of $f,g$. Hence, such pair of $f,g$ doesn't exist in $\Z[[x]]$. So the field of fraction is properly contained in $\Z[[x]]$.
\\\phantom{qed}\hfill$\square$\\
\textbf{Problem 5:}
Since $R$ is an Euclidean domain, $R$ is a unique factorization domain. We claim that every irreducible factor in $X$ has an inverse in the localization.\\
\textbf{Claim:} If $p\in R$ is irreducible and $p\mid x$ for some $x\in X$, then $\frac{1}{p}\in S$.\\
The proof is easy. We suppose $x=pq_1q_2\dots q_k$ is a factorization of $x$ in $R$. Then we have $q_1q_2\dots q_k\in R$. Hence, $\frac{q_1q_2\dots q_k}{x}=\frac{1}{p}\in S$.\\
So for any $\frac{r}{x}\in S$, we can write $\frac{r}{x}=\frac{sy}{x}$ for some $s\in X$ and  $y\in R$ cannot be divided by any irreducible factor in $X$.
Now, we suppose $d:R\to \N$ is a norm function of $R$. And we assume $d(a)\leq d(ab)$ for all $b\neq 0$. We define the norm function 
\[d_0:S\to \N\]
\[d_0(\frac{r}{x})=d_0(\frac{sy}{x})=d(y)\]
We want to show $d_0$ is well-defined. If $\frac{r}{x}=\frac{s_1y_1}{x_1}=\frac{s_2y_2}{x_2}$, then we have $s_1y_1x_2=s_2y_2x_1$. If we factor through irreducible elements, then we have $y_1=p_1p_2\dots p_k$ and $s_1x_2=q_1q_2\dots q_n$. Since $y_1,y_2$ cannot be divided by any irreducible factor in $X$, we know $p_i\mid y_2$ for all $i$. Similarly, $q_j\mid s_2x_1$ for all $j$. So we have $y_2=y_1$ and $s_1x_2=s_2x_1$. Hence, $d_0(\frac{r}{x})=d(y_1)=d(y_2)$ is well-defined.\\
Next, we want to show $d_0$ is a norm function on $S$. For any $a,b\in S$ and $b\neq 0$, we can write it in the form of $a=\frac{s_1y_1}{x_1}$ and $b=\frac{s_2y_2}{x_2}$, where $s_i,x_i\in X$ and $y_i\in R$ cannot be divided by any irreducible factor in $X$. Then in $R$, we have $y_1s_1x_2=qy_2+r$ where $d(r)<d(y_2)$. Then we have 
\[\frac{y_1s_1x_2}{x_1x_2}=\frac{qy_2}{x_1x_2}+\frac{r}{x_1x_2}\]
\[a=\frac{q}{x_1s_2}b+\frac{r}{x_1x_2}\]
Now, we suppose $r=sy$ for some $s\in X$ and $y\in R$ where $y$ cannot be divided by any irreducible factor of $X$.
Notice that $d_0(\frac{r}{x_1x_2})=d(y)\leq d(sy)=d(r)<d(y_2)=d_0(b)$. So $S$ is an Euclidean domain with norm function $d_0$.
\\\phantom{qed}\hfill$\square$\\
\textbf{Problem 6:}\\
(a): For any $a,b\in \mathcal{O}$ and $b\neq 0$, let $z=ab^{-1}\in \Q(\sqrt{D})$ because $\Q(\sqrt{D})$ is a field. If $z\in \mathcal{O}$, then we are done and $r=0$, $q=z$. If not, we choose $q\in\mathcal{O}$ such that $|N(z-q)|<1$. Let $r=a-qb\in \mathcal{O}$. Then we have 
\[d(r)=|N(a-qb)|=|N(zb-qb)|=|N(z-q)N(b)|<|N(b)|=d(b)\]
So we have $a=qb+r$ for some $q,r\in \mathcal{O}$ such that $d(r)<d(b)$ or $r=0$.\\
(b): In each case, we want to prove the assumption in part a is true. So we just need to find $q\in \mathcal{O}$ such that $|N(q-z)|<1$ for any fixed $z\in \Q(\sqrt{D})$.\\
If $D=2$, then $\mathcal{O}=\{a+b\sqrt{2}|a,b\in\Z\}$. For any $z=x+y\sqrt{2}\in \Q(\sqrt{2})$, we can choose $c,d\in\Z$ such that $|c-x|\leq \frac{1}{2}$ and $|d-y|\leq \frac{1}{2}$.
Then we have 
\[|N(c+d\sqrt{2}-z)|=|(c-x)^2-2(d-y)^2|\leq (c-x)^2+2(d-y)^2\leq \frac{1}{4}+2\frac{1}{4}=\frac{3}{4}<1\]
So $q=c+d\sqrt{2}$ is the element we want.\\
If $D=-2$, then similarly, we let $q=c+d\sqrt{-2}$, where $|c-x|\leq \frac{1}{2}$ and $|d-y|\leq \frac{1}{2}$. Then we have 
\[|N(c+d\sqrt{2}-z)|=|(c-x)^2+2(d-y)^2|= (c-x)^2+2(d-y)^2\leq \frac{1}{4}+2\frac{1}{4}=\frac{3}{4}<1\]
If $D=-3,-7,-11$, then we have $\mathcal{O}=\{a+b\omega|a,b\in\Z\}$ and $\omega=\frac{1+\sqrt{D}}{2}$ because $-3\equiv-7\equiv-11\equiv 1 \pmod{4}$. Then for any $z=x+y\sqrt{D}\in \Q(\sqrt{D})$, $y\in[\frac{n}{2},\frac{n+1}{2}]$ for some $n\in \Z$. Hence, we can $d=n$ if $|y-\frac{n}{2}|\leq |y-\frac{n+1}{2}|$ or $d=n+1$ if $|y-\frac{n}{2}|>|y-\frac{n+1}{2}|$. Then we have $|y-\frac{d}{2}|\leq \frac{1}{4}$. Similarly, if $x\in [\frac{m}{2},\frac{m+1}{2}]$ for some $m\in \Z$, we choose $c=\frac{m}{2}-\frac{d}{2}$ if $|x-\frac{m}{2}|\leq |x-\frac{m+1}{2}|$ or $c=\frac{m+1}{2}-\frac{d}{2}$ if $|x-\frac{m}{2}|>|x-\frac{m+1}{2}|$. Then $|c+\frac{d}{2}-x|\leq \frac{1}{4}$. Now, we let $q=c+d\omega=(c+\frac{d}{2})+\frac{d}{2}\sqrt{D}$. We have 
\[|N(q-z)|=|(c+\frac{d}{2}-x)^2-D(\frac{d}{2}-y)^2|=(c+\frac{d}{2}-x)^2-D(\frac{d}{2}-y)^2\leq \frac{1}{16}-D\frac{1}{16}=\frac{1-D}{16}\]
Since $D=-3,-7,-11$, we have $1-D\leq 12$. Hence, $|N(q-z)|\leq \frac{12}{16}<1$.\\
So $\mathcal{O}$ is a Euclidean Domain if $D=-2,2,-3,-7,-11$.
\end{document}