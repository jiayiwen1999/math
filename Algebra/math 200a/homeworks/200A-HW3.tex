\documentclass[12pt]{amsart}
\usepackage{amsmath,epsfig,fancyhdr,amssymb,subfigure,setspace,fullpage,mathrsfs}
\usepackage[utf8]{inputenc}

\newcommand{\R}{\mathbb{R}}
\newcommand{\C}{\mathbb{C}}
\newcommand{\Z}{\mathbb{Z}}
\newcommand{\N}{\mathbb{N}}
\newcommand{\G}{\mathcal{N}}
\newcommand{\A}{\mathcal{A}}
\newcommand{\sB}{\mathscr{B}}
\newcommand{\sC}{\mathscr{C}}
\newcommand{\sd}{{\Sigma\Delta}}

\begin{document}
    \title{Homework3 - 200A}
    \maketitle
    \begin{center}
        Jiayi Wen\\
        A15157596
    \end{center}
    \indent For the notation of function composition, we follow with the one in the text book. For example, $f \circ g$ means do $f$ first then do $g$. For the identity element of a group, we will use either the notation $e$ or $1$ or $1_G$.\\
\noindent \textbf{Problem 1:} Let $X=\{a,b\}$ and let $\phi: X\rightarrow G$ be a function defined as $\phi(b)=f $ and $\phi(a)=g$. Since $F(X)$ is the free group on the set $X$, there exists a unique homomorphism $\hat{\phi}: F(X)\rightarrow G$ commutes the diagram by the universal property. Since the presentation $\langle a,b|b^2=1,ba=a^{-1}b\rangle $ stands for $F(X)/N$ where $N$ is the smallest normal subgroup containing the relations, there exists a unique homomorphism $\varphi:F(X)/N \to G$ such that $i\circ \varphi =\hat{\phi}$ where $i$ is the canonical mapping from $F(X)$ to $F(X)/N$ (the universal property of quotient groups).\\
Next, we show $\varphi$ is an isomorphism. Surjective simply follows from the fact that $G$ is generated by the subset $\{f,g\}$ and $\varphi(Na)=g$, $\varphi(Nb)=f$.\\
To show $\varphi$ is injective, we show every elements in $F(X)/N$ has the format $Na^ib^j$ for $i\in \Z$ and $j\in\{0,1\}$. It is obvious that the power of $b$ is either 0 or 1 since $Nb^2=1_{F(X)/N}$. We can also move all $b$ to the right since for any $Nba^i\in F(X)/N$, we have 
\[Nba^i=Nba\cdot Na^i=Na^{-1}ba^{i-1}=Na^{-2}ba^{i-2}=\dots =Na^{-i}b\]
Therefore, $F(X)/N=\{Na^ib^j|i\in\Z, j\in\Z/2\Z\}$. Next, we show $\varphi(Na^ib^j)=1_G$ if and only if $i=j=0$.\\
If $\varphi(Na^ib^j)=1_G$, then for any $x\in \Z$, we have 
\[\varphi(Na^ib^j)=g^if^j(x)=f^j(x+i)=(-1)^j(x+i)=x\]
If $j=1$, then the only $x$ satisfies the equation is $-\frac{i}{2}$. It contradicts to arbitrary choices of $x$. Hence, $j=0$. But this implies $i=0$ as well.\\
Conversely, if $i=j=0$, then $\varphi(Na^0b^0)=\varphi(1_{F(X)/N})=1_G$. So $\varphi$ is injective. Hence, we have $\varphi$ is an isomorphism and $F(X)/N\cong G$.
\\\phantom{qed}\hfill$\square$\\
\textbf{Problem 2:}  Let $X=\{a,b\}$ and let $\phi: X\rightarrow Q_8$ be a function defined as $\phi(a)=i $ and $\phi(b)=j$. Since $F(X)$ is the free group on the set $X$, there exists a unique homomorphism $\hat{\phi}: F(X)\rightarrow Q_8$ commutes the diagram by the universal property. Since the presentation $\langle a,b|a^2=b^2,a^{-1}ba=b^{-1}\rangle $ stands for $F(X)/N$ where $N$ is the smallest normal subgroup containing the relations, there exists a unique homomorphism $\varphi:F(X)/N \to Q_8$ such that $\pi\circ \varphi =\hat{\phi}$ where $\pi$ is the canonical mapping from $F(X)$ to $F(X)/N$ (the universal property of quotient groups).\\
Similarly, we want to show $\varphi$ is an isomorphism. Surjective simply follows from the fact that $Q_8$ is generated by the subset $\{i,j\}$ and $\varphi(Na)=i$, $\varphi(Nb)=j$.\\
Next, we want to show $F(X)/N=\{Na^ib^j|0\leq i\leq 1,0\leq j\leq 3\}$. Since $Na^{-1}ba=Nb^{-1}$, we have $Nba=Nab^{-1}$. Therefore, we have 
\[Nb^ja^i=Nab^{-j}a^{i-1}=\dots=Na^ib^{(-1)^ij}\]
So, every elements are in the form of $Na^ib^j$ for some $i,j\in \Z$. Next, we want to show $i,j\in\Z/4\Z$. Since $Na^{-1}ba=Nb^{-1}$, we have $Na^{-1}bab=N$. So, $Nbab=Na$.
\[Na^2=Nbabbab=Nbab^2ab=Nbaa^2ab=Nba^4b=Na^4b^{(-1)^4}b=Na^4b^2=Na^4a^2\]
Hence, we have $N=Na^4$. Also, since $Nb^4=Nb^2Nb^2=Na^2Na^2=Na^4=N$, we have $i,j\in\Z/4\Z$. But since $Na^2=Nb^2$ and $Na^3=Naa^2=Nab^2$, we have $0\leq i \leq 1$. Hence, we have at most 2 choices of $i$ and $4$ choices of $j$. So $|F(X)/N|\leq 2\times 4=8$. Notice that $|Q_8|=8$ and we already proved $\varphi$ is surjective, hence $|F(X)/N|=8$ and $\varphi$ is a bijection as a consequence. Then, $\varphi$ is an isomorphism and the proof is complete.
\\\phantom{qed}\hfill$\square$\\
\textbf{Problem 3:} Notice that the notation of $x,y$ are symmetric because the two relations are only different by the notations. Hence, if we can prove $y=e$, then the proof of $x=e$ will be identical. We first show the two equality in the hint. The calculations are just to use the relation $xy^2=y^3x$ several times.
\begin{align*}
    x^2y^8x^{-2}&=x(y^3x)y^6x^{-2}\\
    &=(y^3x)y(y^3x)y^4x^{-2}\\
    &=y^3(y^3x)y^2(y^3x)y^2x^{-2}\\
    &=y^6(y^3x)y^3(y^3x)x^{-2}\\
    &=y^9(y^3x)y^4x^{-1}\\
    &=y^{12}(y^3x)y^2x^{-1}\\
    &=y^{15}(y^3x)x^{-1}\\
    &=y^{18}
\end{align*}
Then we use the result to show the second one.
\begin{align*}
    x^3y^8x^{-3}&=x(y^{18})x^{-1}\\
    &=x(y^2)^{9}x^{-1}\\
    &=(y^3)^9xx^{-1}\\
    &=y^{27}
\end{align*}
Since $yx^2=x^3y$, we have $x^{-2}y^{-1}=y^{-1}x^{-3}$. Then we can show $y^{27}=y^{18}$.
\begin{align*}
    y^{27}&=x^{3}y^8x^{-3}\\
    &=(yx^2)y^7(yy^{-1})x^{-3}\\
    &=yx^2y^8(x^{-2y^{-1}})\\
    &=y(y^{18})y^{-1}\\
    &=y^{18}
\end{align*}
Therefore, we have $y^9=e$. Hence, we have $x^2y^8x^{-2}=(y^9)^2=e^2=e$. So $y^8=x^{-2}x^2=e$.
Therefore, we have $y=ye=yy^8=y^9=e$. Hence, $x=e$. Therefore, $G$ is generated by the identity element, which means $G$ is trivial.
\\\phantom{qed}\hfill$\square$\\
\textbf{Problem 4:} We will follow the hint and the idea is to construct a bijection between the sets and an isomorphism between $H$ and a free group. Let's denote the sets as 
\[W:=\{w_0,w_1,w_2\dots\}\]
\[X:=\{b,a^{-1}ba,a^{-2}ba^2,\dots\}\]
Then there is a map $f:W\to X$, $f(w_i)=a^{-i}ba^i$ with an inverse $g:X\to W$, $g(a^{-i}ba^i)=w_i$. It is obvious that $f$ is a bijection since $f\circ g(w_i)=g(a^{-i}ba^i)=w_i$ for any $w_i\in W$ and $g\circ f(a^{-i}ba^i)=f(w_i)=a^{-i}ba^i$ for any $a^{-i}ba^i\in X$. Since $H=\langle X\rangle$ is a group and $G'=F(W)$ is a free group, the function $f\circ i_X$ induced a homomorphism $\phi:G'\rightarrow H$, where $i_X$ is the canonical mapping from $X$ into $H$. Next, we want to show $\phi$ is an isomorphism.\\
Surjectivity is obvious since $X$ are the generators of $H$ and $\phi^{-1}(a^{-i}ba^i)=(f\circ i_X)^{-1}(a^{-i}ba^i)=g(a^{-i}ba^i)=w_i$ for any $a^{-i}ba^i\in X$.\\
To show $\phi$ is injective, we assume $1\neq w\in F(W)$ such that $\phi(w)=1_H$ towards contradiction. Since $w$ is a nontrivial reduced word, we can assume $w=w_{i_1}^{e_1}w_{i_2}^{e_2}\dots w_{i_n}^{e_n}$, where $w_{i_j}\in W$ and $e_i\in\{\pm 1\}$ for any $1\leq j\leq n$ and $w_{i_j}^{e_{j}}w_{i_{j+1}}^{e_{j+1}}\neq 1_{F(W)}$ for any $1\leq j\leq n-1$. Therefore, for any $1\leq j\leq n-1$, we have 
\[\phi(w_{i_j}^{e_{j}}w_{i_{j+1}}^{e_{j+1}})=(a^{-i_j}ba^{i_j})^{e_j}(a^{-i_{j+1}}ba^{i_{j+1}})^{e_{j+1}}\]
Since $w_{i_j}^{e_{j}}w_{i_{j+1}}^{e_{j+1}}\neq 1_{F(W)}$, we have $e_ji_j+e_{j+1}i_{j+1}\neq 0$. Also,since $a,b$ are two variables in the free group, we have $a,b\neq 1_G$ and $a,b$ does not commute. Therefore, we have 
\[\phi(w_{i_j}^{e_{j}}w_{i_{j+1}}^{e_{j+1}})=a^{-e_ji_j}b^{e_j}a^{e_ji_j+e_{j+1}i_{j+1}}b^{e_{j+1}}a^{e_{j+1}i_{j+1}}\neq 1_{G}=1_{H}\]
Since this is true for any $1\leq j\leq n-1$, then $\phi(w)$ will be some reduced word in $H$ with non-zero powers for $b$. Hence $\phi(w)\neq 1_H$. It contradicts. Hence $\phi$ has trivial kernel. Therefore, $\phi$ is an isomorphism.\\
Verify $H$ is a free group on $X$: For any group $K$, any function from $\tau:X\to K$, we have a function from $W$ to $K$ given by $f\circ \tau$. and an induced homomorphism 
$\widehat{f\circ \tau}:G'=F(W)\to K$. Therefore, we have a homomorphism $\phi^{-1}\circ\widehat{f\circ \tau} $ from $H$ to $K$ such that $\phi^{-1}\circ\widehat{f\circ \tau}|_X= \widehat{f\circ \tau}|_W=f\circ \tau|_W=\tau|_X$. And the uniqueness of induced homomorphism is given by the uniqueness of $\widehat{f\circ \tau} $. Hence, $H$ is a free group on $X$ by definition.\\
\phantom{qed}\hfill$\square$\\
\textbf{Problem 5:} Since the universal property tells us the existence and the uniqueness of induced homomorphism, it also tells us we can construct the set $Hom(F(a,b),D_{2n})$, which consists all homomorphism from $F(a,b)$ to $D_{2n}$, from the set consists of all functions from $\{a,b\}$ to $D_{2n}$. Because given any homomorphism, we can restrict it to $\{a,b\}$ to get a function and the universal property tells us the existence of an induced homomorphism. At the same time, the uniqueness tells us the induced homomorphism is the same as the original one. Therefore, in order to find the automorphism group of $D_{2n}$, we can start from the set $Hom(F(a,b),D_{2n})$ and find out all surjective homomorphism $\sigma$ with kernel satisfies $F(a,b)/ker\sigma=D_{2n}$.\\
Let $\sigma':\{a,b\}\to D_{2n}$ and $\sigma$ be the induced homomorphism. Notice that $\sigma(a)\neq 1 $ and $\sigma(b)\neq 1$ since if either of $a,b$ is identity in $F(a,b)/ker\sigma$, then $F(a,b)/ker\sigma$ is generated by at most one element, which makes it cyclic (or trivial). We first show $\sigma(a)=a^i$ for some $1\leq i\leq n$ and $gcd(i,n)=1$. We will prove by contradiction.\\
If $\sigma(a)=a^d$ for some $gcd(d,n)>1$, then we have $\sigma(a^{\frac{n}{gcd(d,n)}})=(a^d)^{\frac{n}{gcd(d,n)}}=(a^n)^{\frac{d}{gcd(d,n)}}=1$. Hence, $a^{\frac{n}{gcd(d,n)}}\in ker\sigma $. However, in $D_{2n}$, $ord(a)=n>\frac{n}{gcd(d,n)}$. It contradicts. Similarly, if $\sigma(a)=a^ib$ for some $0\leq i\leq n-1$, then by the relation $ba=a^{-1}b$, we know $ba^i=a^{-i}b$. Hence, 
\[\sigma(a^2)=(a^ib)(a^ib)=a^i(ba^i)b=(a^ia^{-i})(bb)=1\cdot 1=1\]
Since $n\geq 3$, it contradicts since $ord(a)=n\geq 3>2$.
Hence, $\sigma(a)=a^i$ for some $1\leq i\leq n$ and $gcd(i,n)=1$.\\
Next, we show $\sigma(b)=a^{i}b$ for some $1\leq i\leq n$. First $\sigma(b)\neq a^j$ for any $i$. Since if $\sigma(b)=a^j$, and we already showed that $\sigma(a)=a^i$ for $i$ that is coprime with $n$. Hence, $\sigma(F(a,b))\subseteq \langle a\rangle$. Therefore, $\sigma$ is not surjective. It contradicts. Hence, $\sigma(b)=a^ib$ for some $1\leq i\leq n$.\\
So far, we showed that every automorphism should $\sigma(a)=a^i$ for some $1\leq i\leq n$ and $gcd(i,n)=1$ and $\sigma(b)=a^jb$ for some $1\leq j\leq n$. Last, we want to verify all of them are indeed isomorphisms between $D_{2n}$ and $D_{2n}$. Suppose $\sigma$ is a homomorphism such that $\sigma(a)=a^i$ and $\sigma(b)=a^jb$ where $gcd(i,n)=1$ and $1\leq i,j\leq n$. Since $gcd(i,n)=1$, $\langle a^i\rangle=\langle a\rangle$. Hence there exists some $1\leq k\leq n$ such that $(a^i)^k=a$. Therefore, we have $\sigma(a^k)=(a^i)^k=a$. Also, there exists a $1\leq l\leq n$ such that $(a^i)^l=a^{-j}$. Then we have 
$\sigma(a^lb)=(a^{i})^la^jb=a^{-j}a^jb=b$. Hence, $\sigma$ is surjective. 
The injectivity is easy since if $g=a^lb^m\in ker\sigma$, then we have 
\[\sigma(g)=a^{il}(a^jb)^m=1\]
Since $ord(a^jb)=2$, we can assume $m=0,1$. If $m=0$, then $n\mid il$ implies $n\mid l$ since $gcd(i,n)=1$. Then we have $g=a^l=a^n=1$. If $m=1$, then we have $\sigma(g)=a^{il+j}b=1$. This contradicts since $b\notin \langle a\rangle$. Therefore, $m$ cannot be 1. Hence, $\sigma$ is injective. So $\sigma$ is indeed an isomorphism.\\
To count the automorphisms, we have $n$ choices for the image of $b$ and $\varphi(n)$ choices for the image of $a$ because the euler $\varphi$-function counts number of coprimes of $n$ in $\Z/n\Z$. Hence, $|Aut(D_{2n})|=n\varphi(n)$. 
\\\phantom{qed}\hfill$\square$\\
\textbf{Problem 6:} Since $Q_8$ is generated by $i,j$, we can define homomorphism based on the image of $i,j$. We want to show that $\sigma$ is an automorphism if and only if $\sigma(i)\in \{\pm i,\pm j,\pm k\}$ and $\sigma(j)\in \{\pm i,\pm j,\pm k\}\setminus \{\pm \sigma(i)\}$.\\
If $\sigma\in Aut(Q_8)$, then $\sigma(i)\neq 1$ since $\sigma$ has trivial kernel. Also, if $\sigma(i)=-1$, then $\sigma(-1)=\sigma(i^2)=(-1)^2=1$ contradicts to the assumption that $\sigma$ has trivial kernel. Similar argument works for $j$ since $j^2=i^2=-1$ in $Q_8$. Therefore, $\sigma(i),\sigma(j)\in \{\pm i,\pm j,\pm k\}$. If $\sigma(j)=\sigma(i)$, then $\sigma(Q_8)=\langle \sigma(i)\rangle \neq Q_8$ since $Q_8$ is not cyclic. If $\sigma(j)=-\sigma(i)$, then we have $\sigma(Q_8)=\langle -1,\sigma(i)\rangle$. However, $-1$ is not a generator of $Q_8$ since $(\pm i)^2=(\pm j)^2=(\pm k)^2=-1$. So $\langle -1,\sigma(i)\rangle=\langle\sigma(i)\rangle\neq Q_8$. Hence, $\sigma(j)\in \{\pm i,\pm j,\pm k\}\setminus \{\pm \sigma(i)\}$.\\
Conversely, if $\sigma(i)\in \{\pm i,\pm j,\pm k\}$ and $\sigma(j)\in \{\pm i,\pm j,\pm k\}\setminus \{\pm \sigma(i)\}$. The universal property tells us $\sigma$ is a homomorphism since $\sigma|_{\{i,j\}}$ is a function that induced a homomorphism $f$ from $F(i,j)$ to $Q_8$, and then there exists a unique homomorphism $g=\sigma$ from $Q_8$ to $Q_8$ such that $f=i\circ g$ where $i$ is the canonical mapping from $F(i,j)$ to the quotient group $Q_8$ (universal property of quoitent group). Next, we need to check it is an isomorphism.\\
Surjectivity follows directly from our choice of the image. Since $Q_8$ can be generated by any two element in the group as long as one is not contained in the cyclic group generated by the another.\\
We show $\sigma$ is injective. From Problem 2, we showed that any element in $Q_8$ are in the format $i^aj^b$ where $0\leq a\leq 3$ and $0\leq j\leq1$. It is not hard to verify all elements in $Q_8$. If $b=0$, then we have $\sigma(i^a)=\sigma(i)^a$. Since $\sigma(i)\in \{\pm i,\pm j,\pm k\}$ and $ord(\pm i)=ord(\pm j)=ord(\pm k)=4$, we have $\sigma(i)^a=1$ if and only if $a\mid 4$. Since $0\leq a \leq 3$, we know $\sigma(i^a)=\sigma(i)^a=1$ if and only if $a=0$. If $b=1$, then we have $\sigma(i^aj)=\sigma(i)^a\sigma(j)$. If $i^aj\in ker\sigma$, then we have $\sigma(j)=(\sigma(i)^a)^{-1}$. It contradicts to $\sigma(j)\notin \langle \sigma(i)\rangle$. Hence, $i^aj\notin ker\sigma$ for any $0\leq a \leq 3$. Therefore, $\ker\sigma=\{1\}$. So $\sigma$ is an isomorphism. The proof is complete.
\\\phantom{qed}\hfill$\square$\\
\end{document}