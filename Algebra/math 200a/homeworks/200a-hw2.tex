\documentclass[12pt]{amsart}
\usepackage{amsmath,epsfig,fancyhdr,amssymb,subfigure,setspace,fullpage,mathrsfs}
\usepackage[utf8]{inputenc}

\newcommand{\R}{\mathbb{R}}
\newcommand{\C}{\mathbb{C}}
\newcommand{\Z}{\mathbb{Z}}
\newcommand{\N}{\mathbb{N}}
\newcommand{\G}{\mathcal{N}}
\newcommand{\A}{\mathcal{A}}
\newcommand{\sB}{\mathscr{B}}
\newcommand{\sC}{\mathscr{C}}
\newcommand{\sd}{{\Sigma\Delta}}

\begin{document}
    \title{Homework2 - 200A}
    \maketitle
    \begin{center}
        Jiayi Wen\\
        A15157596
    \end{center}
    \indent For the notation of function composition, we follow with the one in the text book. For example, $f \circ g$ means do $f$ first then do $g$. For the identity element of a group, we will use either the notation $e$ or $1$ or $1_G$.\\
    \textbf{Problem 2.7:} Let's denote the set of all non-generators of $G$ as $Y$. For trivial group $G$, the argument is true since $\Phi(G)=G$ and $X=G$ is the only possible choice for $\langle X\rangle$ to be a subgroup of $G$. Therefore $e$ is a nongenerator of $G$. Hence, $\Phi(G)=Y=G$.\\
    For nontrivial case: Since $G$ is finite, we can consider the cardinality of all proper subgroups, then the proper subgroup with maximal cardinality is a maximal subgroup. There exists at least one proper subgroup for any nontrivial group $G$, namely the trivial subgroup $\{e\}$. To show $\Phi(G)\subseteq Y$, we assume $g\in\Phi(G)$ but $g\notin Y$ towards contradiction. Then there exists some $X\subseteq G$ such that $\langle X\rangle\neq G$ and $\langle X\cup \{g\}\rangle =G$. Since $\langle X\rangle\leq G$, then $\langle X\rangle\leq H$ for some maximal subgroup $H$ in $G$. But by the definition of the Frattini subgroup, we know $\Phi(G)\leq H$ for every maximal subgroup $H$. Hence, $g\in H$ for every maximal subgroup $H$. Hence, $\langle X\cup \{g\}\rangle \leq H$ because $H$ is closed under group operation. It contradicts to the fact that $\langle X\cup\{g\}\rangle =G$. So $g\in Y$ and $\Phi(G)\subseteq Y$.\\
    Conversely, if $g\in Y$, we show $g\in \Phi(G)$ as well. The proof will also go with contradiction. Let's assume $g\notin H$ for some maximal subgroup $H$ of $G$. Then, take $X=H$ as a subset of $G$. And we have $\langle X\rangle=H$ because $H$ is a subgroup. Then, consider $\langle X\cup\{g\}\rangle\leq G$. It is obvious $H\subseteq  \langle X\cup\{g\}\rangle$ since $X\subset X\cup \{g\}$. And $H\neq \langle X\cup\{g\}\rangle$ since $g\in \langle X\cup\{g\}\rangle$ and $g\notin H$. Therefore, $\langle X\cup\{g\}\rangle=G$ since $H$ is a maximal subgroup. But this contradicts to the fact that $g\in Y$ is a nongenerator of $G$. So $g\in H$ for any maximal subgroup $H$. Therefore, $g\in \Phi(G)$ and $Y\subseteq \Phi(G)$. 
    \\\phantom{qed}\hfill$\square$\\
    \textbf{Problem 2.8:}\\
    \textbf{(a)} A set $T$ is a right transversal for $H$ in $G$ implies that each element in $T$ is a representative of right coset of $H$ in $G$. And for $t_1,t_2\in T$, $Ht_1=Ht_2$ if and only if $t_1=t_2$. Then for $s_1\neq s_2\in S$, we show $Hs_1\neq Hs_2$. Assume $Hs_1=Hs_2$ towards contradiction. Then we have $s_1s_2^{-1}\in H$. Since $S\subseteq K$, then $s_1,s_2\in K$. Because $K$ is a subgroup of $G$, we have $s_1s_2^{-1}\in K$. Hence, $(H\cap K)s_1s_2^{-1}=(H\cap K)$ and $(H\cap K)s_1=(H\cap K)s_2$. It contradicts to $S$ is a right transversal for $H\cap K$ in $K$. So elements in $S$ are representatives for distinct right cosets of $H$ in $G$. To get a right transversal $T$, we can add elements to $S$ by choosing exactly one elements from each distinct right coset of $H$ in $G$, which are not $Hs$ for all $s\in S$. Such $T$ is a right transversal by construction and $S\subseteq T$.\\
    \textbf{(b)} If $T=S$, then given any $g\in G$, there exists some $s\in S=T$ such that $g\in Hs$ since the right cosets partition $G$. And $g\in Hs$ implies that there exists some $h\in H$ such that $g=hs$. Notice that $S\subset K$, so $s\in K$. Hence $g\in HK$. So $G\subseteq HK$. It is obvious that $HK\subseteq G$ since $G$ is a group (closed under group operation). So $G=HK$.\\
    Conversely, if $G=HK$, then for any $g\in G$, there exists $h\in H$ and $k\in K$ such that $g=hk$. In order to show $T=S$, we want to show $g\in Hs$ for some $s\in S$. Since $S$ is transversal for $H\cap K$ in $K$, there exists some $s_1\in S$ such that $k\in (H\cap K)s_1$. Therefore, we have $k=ls_1$ for some $l\in H\cap K$. Then let $h'=hl\in H$ since $h\in H$ and $l\in H\cap K\subseteq H$. This implies $g\in Hs_1$. Since $g$ is arbitrary, we have $G\subseteq \bigsqcup_{s\in S}Hs$ (disjoint union). In part (a), we proved the other direction. Hence, $G=\bigsqcup_{s\in S}Hs$. So $S$ is a right transversal for $H$ in $G$. If $T$ is also a right transversal that contains $S$, then $S=T$ since a transversal contains exactly one element from each coset.\\
    \textbf{(c)} If part (a), we proved that $S\subseteq T$ and the cardinality of transversal counts the number of right cosets, which is the same as the index. Therefore, we have 
    \[|K:H\cap K|=|S|\leq |T|=|G:H|\]
    If $|G:H|< \infty$, then $S,T$ are finite set. Then the eqaulity is the same as $|S|=|T|$. But $|S|=|T|$ if and only if $S=T$ since $S\subseteq T$. From part (b), we know $S=T$ if and only if $G=HK$.\\
    \textbf{(d)} Since $|G|< \infty$, then $|G:H|\leq |G|<\infty$ and $K|\leq |G|<\infty$. By part (c), we have $G=HK$ if and only if $|G:H|=|K:H\cap K|$. Then by Lagrange's Theorem, we have 
    \[\frac{|G|}{|H|}=|G:H|=|K:H\cap K|=\frac{|K|}{|H\cap K|}\]
    \[|G|=\frac{|H||K|}{|H\cap K|}\]
    \\\phantom{qed}\hfill$\square$\\
    \textbf{Problem 2.9:} If $g\in K\cap HL$, then we have $g=hl\in K$ for some $h\in H$ and $l\in L$. Since $H\leq K$, $h^{-1}\in H\leq K$. So we have $l=(h^{-1}h)l=h^{-1}(hl)\in K$. So $l\in K\cap L$. Therefore, $g=hl\in H(K\cap L)$ since $h\in H$ and $l\in K\cap L$. So $(K\cap HL)\subseteq (H(K\cap L))$.\\
    Conversely, if $g\in H(K\cap L)$, then $g=hl$ for some $h\in H$ and $l\in K\cap L\subseteq L$. So $g\in HL$. Also, since $H\leq K$, we have $h\in K$. Therefore $g=hl\in K$ because $K$ is closed under group operation. So $g\in K\cap HL$. Therefore, $H(K\cap L)\subseteq K\cap HL$.\\
    Combine two arguments, we have $K\cap HL= H(K\cap L)$.
    \\\phantom{qed}\hfill$\square$\\
    \textbf{Problem 2.16:} Suppose $G/Z\cong\langle a \rangle$ for some $a\in G$. For any $g,h\in G$, we want to show $gh=hg$. Since the cosets partition $G$, we can assume $g\in Za^i$ and $h\in Za^j$ for some $i,j\in \Z$. Then we have $g=z_1a^i$ and $h=z_2a^j$ for some $z_1,z_2\in Z$. Since $Z\leq Z(G)$, we have 
    \[gh=(z_1a^i)(z_2a^j)=z_1(a^{i}a^j)z_2=z_1z_2a^{i+j}=z_2z_1a^{i+j}=z_2a^{i+j}z_1=z_2a^ja^iz_1=z_2a^jz_1a^i=hg\]
    So $G$ is an abelian group.
    \\\phantom{qed}\hfill$\square$\\
    \textbf{Problem 2.21:} We will follow the hints in the textbook.\\
    We first prove $A\subseteq N_G(A\cap H)$. For any $a\in A$, we split into 2 cases. If $a\in H$, then $a\in A\cap H\subseteq N_G(A\cap H)$. We are done. If $a\notin H$, then we show $(A\cap H)^a=A\cap H$. For any $g\in A\cap H$, we have $g\in A$. Since $A$ is abelian, $g^a=a^{-1}ga=a^{-1}ag=g$. Therefore, $(A\cap H)^a=A\cap H$. Hence $A\subseteq N_G(A\cap H)$.\\
    Then we prove $H\subseteq N_G(A\cap H)$. Similarly, for any $h\in H$, we also split into 2 cases. If $h\in A$, we are done since $h\in A\cap H$. If $h\notin A$, we show $(A\cap H)^h=A\cap H$. For any $g\in A\cap H$, we have $g\in A$ and $g\in H$. We have $h^{-1}gh\in H$ since $H$ is a subgroup of $G$, which is closed under group operation. Also, we have $h^{-1}gh\in A^h=A$ since $A\triangleleft G$ implies $A^h=A$. So $(A\cap H)^h\subseteq (A\cap H)$. Conversely, if $g\in A\cap H$, then we want to show there exists some $g'\in A\cap H$ such that $g=(g')^h$. It is the same as showing $hgh^{-1}\in A\cap H$. Since $g\in H and h\in H$, we have $hgh^{-1}\in H$. Since $g\in A$ and $A\triangleleft G$, we have $hgh^{-1}\in A^{h^{-1}}=A$. Therefore, $hgh^{-1}\in A\cap H$. Hence, $A\cap H\subseteq (A\cap H)^h$. Hence, $A\cap H= (A\cap H)^h$ for any $h\in H$. So $H\subseteq N_G(A\cap H)$.\\
    Next, we want to show $G=N_G(A\cap H)$. It is obvious that $N_G(A\cap H)\subseteq G$ by the definition of the normalizer. For the other direction, since $G=AH$, it is the same as showing $AH=G\subseteq N_G(A\cap H)$. Notice that for any $g\in G=AH$, there exists some $a\in A$ and $h\in H$ such that $g=ah$. By the argument we just proved, we know $a\in N_G(A\cap H)$ and $h\in N_G(A\cap H)$ since $A,H$ are subsets of $N_G(A\cap H)$. Therefore, $g=ah\in N_G(A\cap H)$ because $N_G(A\cap H)$ is a subgroup of $G$, which is closed under group operation. So we have $G=AH\subseteq N_G(A\cap H)$. Hence, $G=N_G(A\cap H)$, which implies $(A\cap H)\triangleleft G$.
    \\\phantom{qed}\hfill$\square$\\
    \textbf{Problem 3.1:} Consider the surjective homomorphism $\phi: G\rightarrow \text{Inn}(G)$ defined by $\phi(g)=\theta_g$, where $\theta_g$ is the inner automorphism induced by $g$. The map is well-defined and it is surjective since for any $\theta_g\in \text{Inn}(G)$, we have $\phi(g)=\theta_g$. we show $\phi$ is a homomorphism. For any $g,h\in G$, $\phi(gh)=\theta_{gh}$. Then for any $x\in G$, the induced inner automorphism $\theta_{gh}(x)=(gh)^{-1}x(gh)=h^{-1}g^{-1}xgh=h^{-1}(g^{-1}xg)h=(x^g)^h=\theta_g\circ \theta_h(x)$. Therefore, we have $$\phi(gh)=\theta_{gh}=\theta_g\circ \theta_h=\phi(g)\circ \phi(h)$$.
    So $\phi$ is a surjective homomorphism. Then by the first isomorphism theorem, we have $G/ker\phi\cong \text{Inn}(G)$. Since $g\in ker(\phi)$ if and only if $\theta_g=\theta_e$ is the identity map on $G$, we have $\theta_g(x)=g^{-1}xg=x$ for every $x\in G$. Then $xg=(gg^{-1})xg=g(g^{-1}xg)=gx$ for every $x\in G$. This happens if and only if $g\in Z(G)$. So we have $g\in ker\phi$ if and only if $g\in Z(G)$. Therefore, $ker\phi=Z(G)$. So $G/Z(G)\cong \text{Inn}G$.\\
    If Inn$(G)$ is a cyclic group, then $G/Z(G)$ is a cyclic group. By problem 2.16, we know this implies $G$ is abelian (Take $Z=Z(G)$). But the center of an abelian group is the whole group, which means $\text{Inn}(G)\cong G/Z(G)=G/G\cong \{e\}$ is a trivial cyclic group. Therefore, we conclude that Inn$(G)$ cannot be a nontrivial cyclic group.
    \\\phantom{qed}\hfill$\square$\\
    \textbf{Problem 3.2:} We prove by contradiction. Suppose $G/Z(G)\cong Q_8$. Since $Q_8=\langle i, j\rangle$, then $Q_8$ has two cyclic subgroups $\langle i \rangle$ and $\langle j\rangle$ with index 2, which are obviously abelian. Consider the canonical homomorphism: $\phi: G \rightarrow G/Z(G)$ that sends elements in $G$ into the corresponding cosets. Then by the third isomorphism theorem, we know there is an 1-1 correspondence between $\{K: Z(G)\leq K\leq G\}$ and $\{L: L\leq G/Z(G)\}$ given by $\phi$. Therefore, we have $\phi^{-1}(\langle i\rangle)$ and $\phi^{-1}(\langle j \rangle)$ are two subgroups of $G$ with index 2 because $|G:\phi^{-1}(\langle i\rangle)|=|G/Z(G):\langle i \rangle|$ and $|G:\phi^{-1}(\langle j\rangle)|=|G/Z(G):\langle j \rangle|$. They are abelian because $Z(G)\subseteq Z(\phi^{-1}(\langle i \rangle))$ and $\phi^{-1}(\langle i \rangle)/Z(G)\cong \langle i \rangle$ is cyclic (problem 2.16, same reason for $\phi^{-1}(\langle j \rangle))$. \\
    Let $g\in G$ such that $\phi(g)=-1$. We want to show $g\in Z(G)$. Since $-1\in \langle i\rangle$ and $-1\in \langle j \rangle$, So $g\in \phi^{-1}(\langle i\rangle)$ and $g\in \phi^{-1}(\langle j\rangle)$. Therefore, $g$ commute with all elements in $\phi^{-1}(g\in \phi^{-1}(\langle i\rangle))$ and $\phi^{-1}g\in \phi^{-1}(\langle j\rangle)$. For any $h\in G$, $h$ is in some cosets of $Z(G)$. Suppose $\phi(i')=i$ and $\phi(j')=j$ for some $i',j'\in G$. Then we have $\phi(i'^\alpha)=(\phi(i'))^\alpha=i^\alpha$ for any $\alpha \in \Z$ and same for $j'$. Then we can suppose $h=zi'^\alpha j'^\beta$ for some $0\leq\alpha,\beta \leq 3$ and $z\in Z(G)$. Then $g$ commutes with $i'$ and $j'$ since $i'\in \phi^{-1}(\langle i\rangle)$ and $j'\in \phi^{-1}(\langle j\rangle)$. So we have 
    \[gh=gzi'^\alpha j'^\beta =zgi'^\alpha j'^\beta =zi'^\alpha gj'^\beta=zi'^\alpha j'^\beta g =hg\]
    So $g\in Z(G)$. But this contradicts to $\phi(g)=-1$ since $Z(G)=ker\phi$ and $\phi(g)=-1\neq 1$. So $G/Z(G)\ncong Q_8$.
    \\\phantom{qed}\hfill$\square$\\
    \textbf{Problem 3.9:}
    We claim that $N_G(H)/C_G(H)\cong \{\theta_g\in\text{Inn}(G):g\in N_G(H)\}\leq Aut(H)$. The isomorphism is simply defined by the inner automorphism restrict to $H$. 
    \[\phi:N_G(H)\rightarrow Aut(H)\]
    \[g\mapsto \theta_g|_H\]
    Notice that $\theta_g|_H$ is indeed a automorphism of $H$ because $\theta_g(H)=g^{-1}Hg=H$ given by the definition of $N_G(H)$. Also, $\phi$ is a well-defined homomorphism because $\phi$ is the homomorphism from $G$ to Inn$(G)$ restricted to $N_G(H)$. Therefore, $\phi(N_G(H))=\{\theta_g\in\text{Inn}(G):g\in N_G(H)\}$. Last, we want to show $ker\phi=C_G(H)$. Since $\theta_g|H=1_{Aut(G)}$ if and only if $\theta_g|H(h)=g^{-1}hg=h$ for any $h\in H$, this is equivalent to say $gh=hg$ for any $h\in H$. Hence, $\theta_g|H=1_{Aut(G)}$ if and only if $g\in C_G(H)$. So $ker\phi=C_G(H)$. Then by the first isomorphism theorem, we proved the claim.
    \\\phantom{qed}\hfill$\square$\\
    \textbf{Problem 3.12:} Since $\varphi$ is surjective, then there exists some $x\in G$ such that $\varphi(x)=h$. Since $G$ is a finite group, then $g$ has finite order. Suppose $ord(x)=\alpha$ and $ord(h)=p^\beta$ for some $\alpha,\beta\in \Z$. Since $\varphi(x)=h$, $\varphi(x^i)=\varphi(x)^i=h^i\in\langle h\rangle$ for any $i\in\Z$. Then we have 
    \[1_H=\varphi(1_G)=\varphi(x^\alpha)=h^\alpha\]
    And $h^\alpha=1_H$ if and only if $p^\beta=ord(h)|\alpha$. Suppose $m\in\Z$ is the largest $p-$power divides $\alpha$ (i.e. $p^m\mid \alpha$ and $p^{m+1}\nmid \alpha$).  By the structure of finite cyclic group, $\langle x\rangle$ has exactly one subgroup with order $p^\beta$, namely $\langle x^{\frac{\alpha}{p^m}}\rangle$. Then any element in $\langle x^{\frac{\alpha}{p^m}}\rangle$ has $p$-power order since $|\langle x^{\frac{\alpha}{p^m}}\rangle|=p^m$ and $p$ is a prime number. Next, we want to show $\varphi(x^{\frac{\alpha}{p^m}})=h^n$ for some $n\in\Z$ and $p\nmid n$. Since $\varphi(x^\frac{\alpha}{p^m})=h^{\frac{\alpha}{p^m}}=h^n$. This implies $\frac{\alpha}{p^m}-n\equiv0(\text{mod }p^\beta)$. Therefore, $\frac{\alpha}{p^m}-n\equiv0(\text{mod }p)$. If $p\mid n$, then we will have $p\mid \frac{\alpha}{p^m}$. It contradicts to the choice of $m$. So $p\nmid n$. So $gcd (n,p^\beta )=1$. This implies that $\langle h^n\rangle=\langle h\rangle$. Asssume $(h^n)^d=h$, then choose $g=x^\frac{\alpha d}{p^m}\in \langle x^{\frac{\alpha}{p^m}}\rangle$, we have $\varphi(g)=\varphi(x^{\frac{\alpha}{p^m}})^d=h^{nd}=h$ and $g$ has $p-$power order.
    \\\phantom{qed}\hfill$\square$\\
    \textbf{Problem 3.13:} We will follow the hint and prove by induction on $|G|$. In order to have some prime number divides the cardinality of $G$, we can assume $G$ is nontrivial.
    Then for $|G|=2$, $2$ is the only prime number divides $|G|$ and the nonidentity element in $G$ has order 2.\\
    Suppose the statement is true for any group with order less than $n$(strong induction). Then for $|G|=n$ and any $p\mid n$, consider a nonidentity element $g\in G$. If $p\mid ord(g)$, then by the structure of cyclic group, we know $ord(g^{\frac{ord(g)}{p}})=p$. So $g^{\frac{ord(g)}{p}}$ is the element we want. If $p\nmid ord(g)$, then we consider the quotient group $G/\langle g\rangle$. The quotient group makes sense here because $G$ is abelian. By Lagrange's Theorem, we have $$p\mid |G|=|\langle g\rangle|\cdot |G/\langle g\rangle|$$
    \[p\mid |G/\langle g\rangle|\]
    because $p\nmid |\langle g\rangle|$. Then there exists an element $\bar{h}=h\langle g\rangle$ in $G/\langle g\rangle$ such that $ord(\bar{h})=p$ by induction since $|\langle g\rangle|>1$. Then we have $h^p\in\langle g\rangle$ ($h$ here is an element in $G$). And $h^d\in \langle g\rangle$ if and only if $\bar{h}^d=\bar{e}=\langle g\rangle$ if and only if $p\mid d$ since $ord(\bar{h})=p$. Therefore, $p\mid ord(h)$ since $h^{ord(h)}=e$ implies $h^{ord(h)}\in \langle g\rangle$. Then we do the same step. The element $h^{\frac{ord(h)}{p}}$ is one we want. This completes the induction. So every finite abelian group has an element of order $p$ if $p$ is a prime divides the cardinality.
    \\phantom{qed}\hfill$\square$\\
\end{document}