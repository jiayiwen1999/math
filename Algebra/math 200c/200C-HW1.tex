\documentclass[12pt]{amsart}
\usepackage{amsmath,epsfig,fancyhdr,amssymb,subfigure,setspace,fullpage,mathrsfs,upgreek,tikz-cd,mathtools}
\usepackage[utf8]{inputenc}

\newcommand{\R}{\mathbb{R}}
\newcommand{\F}{\mathbb{F}}
\newcommand{\Q}{\mathbb{Q}}
\newcommand{\C}{\mathbb{C}}
\newcommand{\Z}{\mathbb{Z}}
\newcommand{\N}{\mathbb{N}}
\newcommand{\G}{\mathcal{N}}
\newcommand{\A}{\mathcal{A}}
\newcommand{\sB}{\mathscr{B}}
\newcommand{\sC}{\mathscr{C}}
\newcommand{\sd}{{\Sigma\Delta}}
\newcommand{\Orbit}{\mathcal{O}}
\newcommand{\normal}{\triangleleft}
\newcommand{\Aut}[0]{\operatorname{Aut}}
\newcommand{\Hom}[0]{\operatorname{Hom}}
\newcommand{\End}[0]{\operatorname{End}}
\newcommand{\Gal}[0]{\operatorname{Gal}}
\newcommand{\Mor}[0]{\operatorname{Mor}}
\newcommand{\Ann}[0]{\operatorname{Ann}}
\newcommand{\Ob}[0]{\operatorname{Ob}}
\newcommand{\catname}[1]{{\normalfont\textbf{#1}}}
\newcommand{\Set}{\catname{Set}}
\newcommand{\Ring}{\catname{Ring}}
\newcommand{\CRing}{\catname{CRing}}
\newcommand{\Grp}{\catname{Grp}}
\newcommand{\Top}{\catname{Top}}
\newcommand{\Ab}{\catname{Ab}}



\begin{document}
\title{Homework 1 - 200C}
\maketitle
\begin{center}
    Jiayi Wen\\
    A15157596
\end{center}
If the category that we are working on is clear, we will omit the subscript for the morphism set. I.e. $\Mor(X,Y)$ means the set of morphisms from $X$ to $Y$.
\section*{Problem 1.} Since $U_2$ is a final object in the category $\mathcal{C}$, then we have a unique morphism $i\in \Mor(U_1,U_2)$. Similarly, since $U_1$ is a final object in the category $\mathcal{C}$, then we have a unique morphism $j\in \Mor(U_2,U_1)$. Then we have
$i\circ j\in \End(U_1)$ and $j\circ i \in \End(U_2)$. Now, by the definition of final objects, we know $\End(U_1), \ \End(U_2)$ are singleton set. And by the identity axiom of category, we know each endomorphism set of objects in $\mathcal{C}$ has an identity element. Thus, we have $i\circ j= Id_{U_1},\ j\circ i =Id_{U_2}$. Hence, $i:U_1\to U_2$ is an isomorphism. Since it is the only element in $\Mor(U_1,U_2)$, so it is the unique isomorphism.
\\\qed\\
\section*{Problem 2.}
\noindent\textbf{(a): } In \Set, we have $\emptyset$ is the inital object since for any other set $A\in\Ob$(\Set), the set of morphism $\Mor(\emptyset,A) $ only contains the empty function (even if $A=\emptyset$).\\
On the other hand, a singleton set, $\{\ast\}$, is the final object in \Set. It is clear that $\Mor(\emptyset,\{\ast\})$ is a singleton set by the fact that $\emptyset$ is the inital object. For any other nonempty set $A$, we have $\Mor(A,\{\ast\})$ only contains constant function. So $\{\ast\}$ is the final object.\\
\textbf{(b):} In \Ring, we have the initial object $(\Z,+,\cdot)$. Since for any other ring $R$ with a unit, the morphism $f:\Z\to R$ must satisfies $f(0)=0_R$ and $f(1)=1_R$. But notice that $\Z$ is generated by $1$, thus the ring homomorphism is determined by the image of $1\in\Z $. For any other morphism $g:\Z\to R$ must satisfies $g(0)=0_R$ and $g(1)=1_R$, which coincides with $f$ on the generating set. Thus $g=f$. So $f$ is the unique element in $\Mor(\Z,R)$.\\
The final object is the zero ring, let's simply denote as 0. Notice that \Ring \  is a subcategory of \Set. Thus, there is at most one morphism in the set $\Mor(R,0)$ for any ring $R$ because $0$ is a singleton set, which is the final object in \Set. The constant map from $c:R\to 0$ is indeed a ring homomorphism since we have $c(0_R)=0$, $c(1_R)=0$, $c(r+s)=0=c(r)+c(s)$ and $c(rs)=0=c(r)c(s)$.\\
\textbf{(c):} We denote the category of commutative ring as \CRing. First, notice that \CRing \ is a subcategory of \Ring. And we have both $\Z$ and $0$ are objects in \CRing. So it has the same universal objects as \Ring.\\
\textbf{(d):} In \Top, we give the trivial topology to both the empty set and the singleton set. Then for any topological space $X$, both $\Mor(\emptyset, X)$ and $\Mor(X,\{\ast\})$ have at most 1 morphism. For the empty function, $f:\emptyset\to X$, for any open set $U\subset X$, we have $f^{-1}(U)=\emptyset$ is open. So $f$ is continuous. So $\emptyset $ is the initial object in \Top. On the other hand, we know the constant function is continuous; hence, $\{\ast\}$ is the final object in \Top.\\
\textbf{(e):} Another way to view \Top$(X)$ is to view it as a poset, where the partial order is set inclusion. Thus, the empty set is the intial object since every open set in $X$ contains the empty set. Also, we have $X$ is the final object since open sets in $X$ are subset of $X$.\\\qed\\
\section*{Problem 3.}
\noindent \textbf{(a):}  Consider an arbitrary collection of abelian groups $\{G_i\}_{i\in I}$. Now, we want to show $\bigoplus_{i\in I}G_i$ and $\prod_{i\in I} G_i$ are abelian groups. First, we have $(0)_{i\in I}\in \prod_{i\in I} G_i$ and $(0)_{i\in I}\in \bigoplus_{i\in I} G_i$. For any $(g_i)_{i\in I}\in \prod_{i\in I} G_i$, we have $(-g_i)_{i\in I}\in \prod_{i\in I} G_i$. Also, for any $(h_i)_{i\in I}\in \prod_{i\in I} G_i$, we have
\[(g_i)_{i\in I}+(h_i)_{i\in I}=(g_i+h_i)_{i\in I}=(h_i+g_i)_{i\in I}=(h_i)_{i\in I}+(g_i)_{i\in I}\]
And $(g_i)_{i\in I}+(h_i)_{i\in I}=(g_i+h_i)_{i\in I}\in \prod_{i\in I}G_i$. So, $ \prod_{i\in I}G_i$ is an abelian group. \\
Notice that $\bigoplus_{i\in I} G_i$ is a subset of $\prod_{i\in I}G_i$. Now, we just need to check the subgroup test. Given any element $(g_i)_{i\in I}\bigoplus_{i\in I}G_i$ and $(h_i)_{i\in I}\bigoplus_{i\in I}G_i$, we have
\[(g_i)_{i\in I}+(-h_i)_{i\in I}=(g_i-h_i)_{i\in I}\in \bigoplus_{i\in I}G_i\]
This has finitely many nonzero coordinates because $(g_i)_{i\in I}$ and $(-h_i)_{i\in I}$ have finitely many nonzero coordinates.\\
So $\bigoplus_{i\in I} G_i$ is an abelian group.\\
\textbf{(b):} Let's check the module axioms. We check the direct product first. For any $r,s\in R$ and $(m_i)_{i\in I},\ (n_i)_{i\in I}\in \prod_{i\in I}M_i$, we have
\[r\cdot (s\cdot (m_i)_{i\in I})=r\cdot (s\cdot m_i)_{i\in I}=(rs\cdot m_i)_{i\in I}=(rs)\cdot (m_i)_{i\in I}\]
\[1\cdot (m_i)_{i\in I}=(1\cdot m_i)_{i\in I}=(m_i)_{i\in I}\]
\[(r+s)\cdot(m_i)_{i\in I}=((r+s)\cdot m_i)_{i\in I}=(r\cdot m_i)_{i\in I}+(s\cdot m_i)_{i\in I}=r\cdot (m_i)_{i\in I}+s\cdot (m_i)_{i\in I}\]
\[r\cdot (m_i+n_i)_{i\in I}=(r\cdot (m_i+n_i))_{i\in I}=(r\cdot m_i)_{i\in I}+(r\cdot n_i)_{i\in I}=r\cdot (m_i)_{i\in I}+r\cdot (n_i)_{i\in I}\]
So $\prod_{i\in I}M_i$ is an $R$-mod.\\
Now, if $(m_i)_{i\in I},\ (n_i)_{i\in I}\in \bigoplus_{i\in I}M_i$, then all equality holds since they are just special cases (finite nonzero coordinates) of the general situation. \\
\textbf{(c):} By part b, we already showed that both $\prod_{i\in I}M_i$ and $\bigoplus_{i\in I}M_i$ are $R$-mods. Thus, they are the direct product and the direct sum of $R$-mods, respectively, by definition.\\
\textbf{(d):} Consider the mapping $\varphi: \Hom(\bigoplus_{i\in I}M_i, N)\to \prod_{i\in I}\Hom(M_i,N)$, where $\varphi(f)=(f|_{M_i})_{i\in I}$. Since $R$ is commutative, we have an $R$-mod structure on the hom space. Also, we have $M_i$ is a submodule of $\bigoplus_{i\in I}M_i$ for all $i\in I$. Hence, $f|_{M_i}$ is again an $R$-module homomorphism. Now, we want to show $\varphi$ is an $R$-mod homomorphism. For any $f,g\in \Hom(\bigoplus_{i\in I}M_i, N)$ and $r\in R$, we have
\[\varphi(f+g)=\Big((f+g)|_{M_i}\Big)_{i\in I}=(f|_{M_i})_{i\in I}+(g|_{M_i})_{i\in I}=\varphi(f)+\varphi(g)\]
\[\varphi(r\cdot f)=\Big((r\cdot f)|_{M_i}\Big)_{i\in I}=\Big(r\cdot f|_{M_i}\Big)_{i\in I}=r\cdot (f|_{M_i})_{i\in I} =r\cdot\varphi(f) \]
It is injective since if $\varphi(f)=\varphi(g)$, then we have $f|_{M_i}=g|_{M_i}$ for all $i\in I$. Hence, we have $f=g$. Also, for any $(f_i)_{i\in I}\in \prod_{i\in I}\Hom(M_i,N)$, using the universal property of direct sum, we can define a $R$-mod homomorphism $f$ from $\bigoplus_{i\in I}M_i$ to $N$ such that $f\circ j_i=f_i$ for all $i\in I$, where $j_i$ is the embedding of $M_i$ into the direct sum. Notice that precomposing the inclusion is the same as restricting the morphism; hence, we have $\varphi(f )=(f_i)_{i\in I}$. So $\varphi$ is surjective. So we have an isomorphism between $\Hom(\bigoplus_{i\in I}M_i, N)\cong \prod_{i\in I}\Hom(M_i,N)$.\\
Now, we prove the second one. Similarly, we define a mapping $\phi: \Hom(N,\prod_{i\in I}M_i)\to \prod_{i\in I}\Hom(N,M_i)$ such that $\phi(f)=(f_i)_{i\in I}$, where $f_i$ is the $i$-th coordinates of $f$. The proof is similarly since for each coordinate function $f_i$ of $f$, it is a $R$-mod homomorphism from $N$ to $M_i$. Then for any $f,g\in \Hom(N,\prod_{i\in I}M_i)$ and $r\in R$, we have
\[\phi(f+g)=\Big((f+g)_i\Big)_{i\in I}=(f_i)_{i\in I}+(g_i)_{i\in I}=\phi(f)+\phi(g)\]
\[\phi(r\cdot f)=\Big((r\cdot f)_i\Big)_{i\in I}=\Big(r\cdot f_i\Big)_{i\in I}=r\cdot (f_i)_{i\in I} =r\cdot\phi(f) \]
So $\phi$ is a $R$-mod homomorphism. It is injective since $\phi(f)=\phi(g)$ is the same as saying $f_i=g_i$ for all $i\in I$, which is the same as saying $f=g$. It is surjective since for any element in $(f_i)_{i\in I}\in \prod_{i\in I}\Hom(N,M_i)$, we can define $f(n)=(f_i(n))_{i\in I}$ for any $n\in N$. Thus, we have $\phi(f)=(f_i)_{i\in I}$. So $\phi$ is an isomorphism.
\\\textbf{(e):} We want to use the universal property of tensor product here. Let's consider the mapping
\[\phi: N\times \bigoplus_{i\in I} M_i\to \bigoplus_{i\in I} N\otimes_R M_i\]
\[\Big(n,(m_i)_{i\in I}\Big)\mapsto (n\otimes m_i)_{i\in I}\]
Notice that the image of $\phi$ is indeed in the direct sum of tensor products since only finitely many $m_i$ is nonzero; hence, only finitely many $n\otimes m_i$ is nonzero. Now, we want to show this map is bilinear. For any $n_1,n_2\in N$, $(m_i)_{i\in I}, (s_i)_{i\in I}\in \bigoplus_{i\in I} M_i$, and $r_1,r_2\in R$, we have
\begin{align*}
    \phi\Big(r_1n_1+r_2n_2,(m_i)_{i\in I}\Big) & =\Big((r_1n_1+r_2n_2)\otimes m_i\Big)_{i\in I} \\ &=r_1(n_1\otimes m_i)_{i\in I}+r_2(n_2\otimes m_i)_{i\in I}\\ &=r_1\phi(n_1,(m_i)_{i\in I})+r_2\phi(n_2,(m_i)_{i\in I})
\end{align*}
\begin{align*}
    \phi\Big(n_1,r_1(m_i)_{i\in I}+r_2(s_i)_{i\in I}\Big) & =\Big(n_1\otimes (r_1m_i+r_2s_i)\Big)_{i\in I} \\ &=r_1(n_1\otimes m_i)_{i\in I}+r_2(n_1\otimes s_i)_{i\in I}\\ &=r_1\phi(n_1,(m_i)_{i\in I})+r_2\phi(n_1,(s_i)_{i\in I})
\end{align*}
So there exists a unique $R$-mod homomorphism $\sigma: N\otimes_R \bigoplus_{i\in I} M_i\to \bigoplus_{i\in I} N\otimes_R M_i$ commutes the diagram.
\begin{center}
    \begin{tikzcd}
        N\times \bigoplus_{i\in I} M_i \arrow[rd, "\phi"] \arrow[r]  & N\otimes_R \bigoplus_{i\in I} M_i \arrow[d,"\exists ! \ \sigma", dashed] \\ & \bigoplus_{i\in I} N\otimes_R M_i
    \end{tikzcd}
\end{center}
Now, we want to show $\sigma$ is a bijection by constructing an explicit inverse. Notice that we have $R$-mod homomorphisms $\tau_i=1_N\otimes j_i: N\otimes_R M_i\to N\otimes_R \bigoplus_{i\in I} M_i$, where $j_i$ is the inclusion of direct sum. Now, by the universal property of direct sum, we have an $R$-mod homomorphism $\tau:\bigoplus_{i\in I} N\otimes_R M_i\to N\otimes_R \bigoplus_{i\in I} M_i$ such that $\tau=\sum_{i\in I}\tau_i$. We claim that $\tau$ is the inverse of $\sigma$.
\[\tau\circ \sigma\Big(n\otimes (m_i)_{i\in I}\Big)=\tau(n\otimes m_i)_{i\in I}=n\otimes (m_i)_{i\in I} \]
So we have $\tau\circ \sigma=Id_{N\otimes_R \bigoplus_{i\in I} M_i}$.
Also, we have
\[\sigma\circ \tau(n_i\otimes m_i)_{i\in I}=\sigma(\sum_{i\in I}\tau_i(n_i\otimes m_i))=\sum_{i\in i}\sigma\circ \tau_i(n_i\otimes m_i)=(n_i\otimes m_i)_{i\in I}\]
So we have $\sigma\circ \tau= Id_{\bigoplus_{i\in I} N\otimes_R M_i }$. So we have $\bigoplus_{i\in I} N\otimes_R M_i\cong  N\otimes_R \bigoplus_{i\in I}M_i$.
\\\textbf{(f):} No. We take $R=\Z$ and consider the infinite direct product of $\Z/p\Z$, where $p$ is prime. Then we know $\Z/p\Z \otimes \Q=0$ since for any nonzero element in $\Q$, we can divide it by $p$ in $Q$, which is an annihilator of $\Z/p\Z$. However, we have
\[\Ann_\Z(\prod \Z/p\Z)=0\]
In particular, we have $0\neq (1,1,\cdots)\otimes 2\in(\prod\Z/p\Z)\otimes_\Z\Q$. So it is different from $\prod(\Z/p\Z\otimes_\Z\Q)=\prod(0)=0$.\\
\textbf{(g):} Yes. It actually follows from part e. Suppose $M,N$ are free $R$-mods, then we assume $X$ is a basis of $M$ and $Y$ is a basis of $N$. Then we have $M\cong \bigoplus_{x\in X}R$ and $N\cong \bigoplus_{y\in Y}R$. Then we have
\[M\otimes_R N\cong (\bigoplus_{x\in X}R)\otimes_R N\cong \bigoplus_{x\in X}(R\otimes_R N)\cong \bigoplus_{x\in X}N\cong \bigoplus_{(x,y)\in X\times Y}R  \]
So $M\otimes_R N$ is a free $R$-mod.\\
\qed\\
\section*{Problem 4.}
\noindent\textbf{(a):} We consider the $(R_1\times R_2,\pi_1,\pi_2)$, where $\pi_i$ is the i-th coordinate projection. Then for any ring $S\in \Ob(\mathcal{R} )$ and mapping $f_1: S\to R_1$ and $f_2:S\to R_2$, we can define a new ring homomorphism $f:S\to R_1\times R_2$, where $f=(f_1,f_2)$. Since $\pi_1\circ f= f_1$ and $\pi_2\circ  f= f_2$, we have the following diagram commutes.
\begin{center}
    \begin{tikzcd}
        S \arrow[rd, "f" description] \arrow[rrd, "f_1"] \arrow[rdd, "f_2"'] &                                                     &     \\
        & R_1\times R_2 \arrow[r, "\pi_1"'] \arrow[d, "\pi_2"] & R_1 \\
        & R_2                                                 &
    \end{tikzcd}
\end{center}
If there is another homomorphism $g:S\to R_1\times R_2$ fits into the diagram, then we have
\[\pi_1\circ g(s)=f_1(s)=\pi_1\circ f(s)\]
So $g$ agrees with $f$ at the first coordinate. Similarly, we have
\[\pi_2\circ g(s)=f_1(s)=\pi_2\circ f(s)\]
So $g$ also agrees with $f$ at the second coordinate. Thus, $g=f$. So $f$ is unique. So $R_1\times R_2$ is the product of $R_1$ and $R_2$ in $\mathcal{R}$.\\
\textbf{(b):} Notice that by the definition of direct sum, we have $R_1\oplus R_2$ is a subring of $R_1\times R_2$, where only finitely many coordinate is nonzero. Since we are dealing with fintie direct sum, we have $R_1\oplus R_2\cong R_1\times R_2$ as rings. Now, we show it is not the coproduct of $R_1$ and $R_2$.\\
Suppose it is the coproduct, then we have the following counterexample. Let $R_1\cong R_2\cong \Z$. Then we have a unique ring homomorphism $f:\Z\to \Z\times \Z$ because $\Z$ is the initial object in $\mathcal{R}$, where $f(1)=(1,1)$. Then we have the following diagram
\begin{center}\begin{tikzcd}
        & \Z \arrow[d, "f"] \arrow[rdd, "Id_\Z"] &    \\
        \Z \arrow[r, "f"] \arrow[rrd, "Id_\Z"'] & \Z\times \Z \arrow[rd, "\phi" description]         &    \\
        &                                        & \Z
    \end{tikzcd}
\end{center}
But we have two choices of $\phi$. We can either use the first coordinate projection or the second coordinate projection. It contradicts with the definition of coproduct.\\\qed
\section*{Problem 5.}
\noindent If there exists a fiber product $(Z_1,\pi_1,\pi_2)$ of $f,g$ in $\mathcal{C}$, then we suppose there is the triple $(Z_2,\sigma_1,\sigma_2)$ is also a fiber product of $f,g$ in $\mathcal{C}$. Then by definition of the fiber product, we have unique morphisms $u:Z_1\to Z_2$ and $v: Z_2\to Z_1$ such that the diagram commutes
\[    \begin{tikzcd}
        Z_1 \arrow[rd, "u" description] \arrow[rrd, "\pi_1"] \arrow[rdd, "\pi_2"'] &                                                     &     \\
        & Z_2 \arrow[r, "\sigma_1"'] \arrow[d, "\sigma_2"] & X \arrow[d, "f"]\\
        & Y \arrow[r,"g"]                                                 & S
    \end{tikzcd}
    \begin{tikzcd}
        Z_2 \arrow[rd, "v" description] \arrow[rrd, "\sigma_1"] \arrow[rdd, "\sigma_2"'] &                                                     &     \\
        & Z_1 \arrow[r, "\pi_1"'] \arrow[d, "\pi_2"] & X \arrow[d, "f"]\\
        & Y \arrow[r,"g"]                                                 & S
    \end{tikzcd}
\]
Now, if we combine these two diagram, we get
\[
    \begin{tikzcd}
        Z_2 \arrow[rd, "v" description] \arrow[rrrdd, "\sigma_1"] \arrow[rrddd, "\sigma_2"'] &                                                                                                   &                                                  &                  \\
        & Z_1 \arrow[rd, "u" description] \arrow[rrd, "\pi_1" description] \arrow[rdd, "\pi_2" description] &                                                  &                  \\
        &                                                                                                   & Z_2 \arrow[r, "\sigma_1"'] \arrow[d, "\sigma_2"] & X \arrow[d, "f"] \\
        &                                                                                                   & Y \arrow[r, "g"]                                 & S
    \end{tikzcd}\]
But notice that we also have
\[ \begin{tikzcd}
        Z_2 \arrow[rd, "Id_{Z_2}" description] \arrow[rrd, "\sigma_1"] \arrow[rdd, "\sigma_2"'] &                                                     &     \\
        & Z_2 \arrow[r, "\sigma_1"'] \arrow[d, "\sigma_2"] & X \arrow[d, "f"]\\
        & Y \arrow[r,"g"]                                                 & S
    \end{tikzcd}\]
By uniqueness, we have $u\circ v= Id_{Z_2}$. Similarly, we have
\[
    \begin{tikzcd}
        Z_1 \arrow[rd, "u" description] \arrow[rrrdd, "\pi_1"] \arrow[rrddd, "\pi_2"'] &                                                                                                   &                                                  &                  \\
        & Z_2 \arrow[rd, "v" description] \arrow[rrd, "\sigma_1" description] \arrow[rdd, "\sigma_2" description] &                                                  &                  \\
        &                                                                                                   & Z_1 \arrow[r, "\pi_1"'] \arrow[d, "\pi_2"] & X \arrow[d, "f"] \\
        &                                                                                                   & Y \arrow[r, "g"]                                 & S
    \end{tikzcd}
    \begin{tikzcd}
        Z_1 \arrow[rd, "Id_{Z_1}" description] \arrow[rrd, "\pi_1"] \arrow[rdd, "\pi_2"'] &                                                     &     \\
        & Z_1 \arrow[r, "\pi_1"'] \arrow[d, "\pi_2"] & X \arrow[d, "f"]\\
        & Y \arrow[r,"g"]                                                 & S
    \end{tikzcd}
\]
So we have $v\circ u=Id_{Z_1}$ by the uniqueness. So we have $Z_1\cong Z_2$ in $\mathcal{C}$ and $\sigma_i,\pi_i$ are differ by isomorphisms $u,v$. So the fiber product is unique up to isomorphism. \\\qed
\section*{Problem 6.}
\noindent\textbf{(a):} First, we want to show this diagram commutes
\begin{center}
    \begin{tikzcd}
        X\times_S Y \arrow[r, "\pi_1"]\arrow[d,"\pi_2"] & X \arrow[d,"f"]\\
        Y\arrow[r, "g"] & S
    \end{tikzcd}
\end{center}
Where $\pi_i$ is the i-th coordinate projection. For any $(x,y)\in X\times_S Y$, we have 
\[f\circ \pi_1(x,y)=f(x)=g(y)=g\circ \pi_2(x,y)\]
Now, for any object $A\in \Ob(\Set)$ and functions $q_1:A\to X$ and $q_2:A\to Y$ such that $f\circ q_1=g\circ q_2$, we define a mapping $q: A\to X\times_S Y$  where $q(a)=(q_1(a),q_2(a))$. The image of $q$ is in the set $X\times_S Y$ since $f\circ q_1=g\circ q_2$, and it is well-defined since $q_1,q_2$ are well-defined maps. Also, we have 
\[\pi_1\circ q(a)=q_1(a)\]
\[\pi_2\circ q(a)=q_2(a)\]
So the diagram commutes.
\[
    \begin{tikzcd}
        A \arrow[rd, "q" description] \arrow[rrd, "q_1"] \arrow[rdd, "q_2"'] &                                                     &     \\
        & X\times_S Y \arrow[r, "\pi_1"'] \arrow[d, "\pi_2"] & X \arrow[d, "f"]\\
        & Y \arrow[r,"g"]                                                 & S
    \end{tikzcd}    
\]
If $r=(r_1,r_2):A\to X\times_S Y$ also commutes the diagram, then we have 
\[q_1(a)=\pi_1\circ r(a)=r_1(a)\]
\[q_2(a)=\pi_2\circ q(a)=r_2(a)\]
Hence, we have $r=q$. So the function $q$ is unique. So $X\times_S Y$ is the fiber product of $f,g$ in \Set.\\
\textbf{(b):} Intersection.
\\\qed
\end{document}